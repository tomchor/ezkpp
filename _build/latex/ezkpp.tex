% Generated by Sphinx.
\def\sphinxdocclass{report}
\newif\ifsphinxKeepOldNames \sphinxKeepOldNamestrue
\documentclass[letterpaper,10pt,openany,oneside]{sphinxmanual}
\usepackage{iftex}

\ifPDFTeX
  \usepackage[utf8]{inputenc}
\fi
\ifdefined\DeclareUnicodeCharacter
  \DeclareUnicodeCharacter{00A0}{\nobreakspace}
\fi
\usepackage{cmap}
\usepackage[T1]{fontenc}
\usepackage{amsmath,amssymb,amstext}
\usepackage[english]{babel}
\usepackage{times}
\usepackage[Bjarne]{fncychap}
\usepackage{longtable}
\usepackage{sphinx}
\usepackage{multirow}
\usepackage{eqparbox}


\addto\captionsenglish{\renewcommand{\figurename}{Fig.\@ }}
\addto\captionsenglish{\renewcommand{\tablename}{Table }}
\SetupFloatingEnvironment{literal-block}{name=Listing }

\addto\extrasenglish{\def\pageautorefname{page}}

\setcounter{tocdepth}{1}


\title{An easy guide for KPP}
\date{Nov 23, 2016}
\release{0.1}
\author{Tomas Chor}
\newcommand{\sphinxlogo}{\sphinxincludegraphics{wheat_field.jpg}\par}
\renewcommand{\releasename}{Release}
\makeindex

\makeatletter
\def\PYG@reset{\let\PYG@it=\relax \let\PYG@bf=\relax%
    \let\PYG@ul=\relax \let\PYG@tc=\relax%
    \let\PYG@bc=\relax \let\PYG@ff=\relax}
\def\PYG@tok#1{\csname PYG@tok@#1\endcsname}
\def\PYG@toks#1+{\ifx\relax#1\empty\else%
    \PYG@tok{#1}\expandafter\PYG@toks\fi}
\def\PYG@do#1{\PYG@bc{\PYG@tc{\PYG@ul{%
    \PYG@it{\PYG@bf{\PYG@ff{#1}}}}}}}
\def\PYG#1#2{\PYG@reset\PYG@toks#1+\relax+\PYG@do{#2}}

\expandafter\def\csname PYG@tok@gd\endcsname{\def\PYG@tc##1{\textcolor[rgb]{0.63,0.00,0.00}{##1}}}
\expandafter\def\csname PYG@tok@gu\endcsname{\let\PYG@bf=\textbf\def\PYG@tc##1{\textcolor[rgb]{0.50,0.00,0.50}{##1}}}
\expandafter\def\csname PYG@tok@gt\endcsname{\def\PYG@tc##1{\textcolor[rgb]{0.00,0.27,0.87}{##1}}}
\expandafter\def\csname PYG@tok@gs\endcsname{\let\PYG@bf=\textbf}
\expandafter\def\csname PYG@tok@gr\endcsname{\def\PYG@tc##1{\textcolor[rgb]{1.00,0.00,0.00}{##1}}}
\expandafter\def\csname PYG@tok@cm\endcsname{\let\PYG@it=\textit\def\PYG@tc##1{\textcolor[rgb]{0.25,0.50,0.56}{##1}}}
\expandafter\def\csname PYG@tok@vg\endcsname{\def\PYG@tc##1{\textcolor[rgb]{0.73,0.38,0.84}{##1}}}
\expandafter\def\csname PYG@tok@vi\endcsname{\def\PYG@tc##1{\textcolor[rgb]{0.73,0.38,0.84}{##1}}}
\expandafter\def\csname PYG@tok@mh\endcsname{\def\PYG@tc##1{\textcolor[rgb]{0.13,0.50,0.31}{##1}}}
\expandafter\def\csname PYG@tok@cs\endcsname{\def\PYG@tc##1{\textcolor[rgb]{0.25,0.50,0.56}{##1}}\def\PYG@bc##1{\setlength{\fboxsep}{0pt}\colorbox[rgb]{1.00,0.94,0.94}{\strut ##1}}}
\expandafter\def\csname PYG@tok@ge\endcsname{\let\PYG@it=\textit}
\expandafter\def\csname PYG@tok@vc\endcsname{\def\PYG@tc##1{\textcolor[rgb]{0.73,0.38,0.84}{##1}}}
\expandafter\def\csname PYG@tok@il\endcsname{\def\PYG@tc##1{\textcolor[rgb]{0.13,0.50,0.31}{##1}}}
\expandafter\def\csname PYG@tok@go\endcsname{\def\PYG@tc##1{\textcolor[rgb]{0.20,0.20,0.20}{##1}}}
\expandafter\def\csname PYG@tok@cp\endcsname{\def\PYG@tc##1{\textcolor[rgb]{0.00,0.44,0.13}{##1}}}
\expandafter\def\csname PYG@tok@gi\endcsname{\def\PYG@tc##1{\textcolor[rgb]{0.00,0.63,0.00}{##1}}}
\expandafter\def\csname PYG@tok@gh\endcsname{\let\PYG@bf=\textbf\def\PYG@tc##1{\textcolor[rgb]{0.00,0.00,0.50}{##1}}}
\expandafter\def\csname PYG@tok@ni\endcsname{\let\PYG@bf=\textbf\def\PYG@tc##1{\textcolor[rgb]{0.84,0.33,0.22}{##1}}}
\expandafter\def\csname PYG@tok@nl\endcsname{\let\PYG@bf=\textbf\def\PYG@tc##1{\textcolor[rgb]{0.00,0.13,0.44}{##1}}}
\expandafter\def\csname PYG@tok@nn\endcsname{\let\PYG@bf=\textbf\def\PYG@tc##1{\textcolor[rgb]{0.05,0.52,0.71}{##1}}}
\expandafter\def\csname PYG@tok@no\endcsname{\def\PYG@tc##1{\textcolor[rgb]{0.38,0.68,0.84}{##1}}}
\expandafter\def\csname PYG@tok@na\endcsname{\def\PYG@tc##1{\textcolor[rgb]{0.25,0.44,0.63}{##1}}}
\expandafter\def\csname PYG@tok@nb\endcsname{\def\PYG@tc##1{\textcolor[rgb]{0.00,0.44,0.13}{##1}}}
\expandafter\def\csname PYG@tok@nc\endcsname{\let\PYG@bf=\textbf\def\PYG@tc##1{\textcolor[rgb]{0.05,0.52,0.71}{##1}}}
\expandafter\def\csname PYG@tok@nd\endcsname{\let\PYG@bf=\textbf\def\PYG@tc##1{\textcolor[rgb]{0.33,0.33,0.33}{##1}}}
\expandafter\def\csname PYG@tok@ne\endcsname{\def\PYG@tc##1{\textcolor[rgb]{0.00,0.44,0.13}{##1}}}
\expandafter\def\csname PYG@tok@nf\endcsname{\def\PYG@tc##1{\textcolor[rgb]{0.02,0.16,0.49}{##1}}}
\expandafter\def\csname PYG@tok@si\endcsname{\let\PYG@it=\textit\def\PYG@tc##1{\textcolor[rgb]{0.44,0.63,0.82}{##1}}}
\expandafter\def\csname PYG@tok@s2\endcsname{\def\PYG@tc##1{\textcolor[rgb]{0.25,0.44,0.63}{##1}}}
\expandafter\def\csname PYG@tok@nt\endcsname{\let\PYG@bf=\textbf\def\PYG@tc##1{\textcolor[rgb]{0.02,0.16,0.45}{##1}}}
\expandafter\def\csname PYG@tok@nv\endcsname{\def\PYG@tc##1{\textcolor[rgb]{0.73,0.38,0.84}{##1}}}
\expandafter\def\csname PYG@tok@s1\endcsname{\def\PYG@tc##1{\textcolor[rgb]{0.25,0.44,0.63}{##1}}}
\expandafter\def\csname PYG@tok@ch\endcsname{\let\PYG@it=\textit\def\PYG@tc##1{\textcolor[rgb]{0.25,0.50,0.56}{##1}}}
\expandafter\def\csname PYG@tok@m\endcsname{\def\PYG@tc##1{\textcolor[rgb]{0.13,0.50,0.31}{##1}}}
\expandafter\def\csname PYG@tok@gp\endcsname{\let\PYG@bf=\textbf\def\PYG@tc##1{\textcolor[rgb]{0.78,0.36,0.04}{##1}}}
\expandafter\def\csname PYG@tok@sh\endcsname{\def\PYG@tc##1{\textcolor[rgb]{0.25,0.44,0.63}{##1}}}
\expandafter\def\csname PYG@tok@ow\endcsname{\let\PYG@bf=\textbf\def\PYG@tc##1{\textcolor[rgb]{0.00,0.44,0.13}{##1}}}
\expandafter\def\csname PYG@tok@sx\endcsname{\def\PYG@tc##1{\textcolor[rgb]{0.78,0.36,0.04}{##1}}}
\expandafter\def\csname PYG@tok@bp\endcsname{\def\PYG@tc##1{\textcolor[rgb]{0.00,0.44,0.13}{##1}}}
\expandafter\def\csname PYG@tok@c1\endcsname{\let\PYG@it=\textit\def\PYG@tc##1{\textcolor[rgb]{0.25,0.50,0.56}{##1}}}
\expandafter\def\csname PYG@tok@o\endcsname{\def\PYG@tc##1{\textcolor[rgb]{0.40,0.40,0.40}{##1}}}
\expandafter\def\csname PYG@tok@kc\endcsname{\let\PYG@bf=\textbf\def\PYG@tc##1{\textcolor[rgb]{0.00,0.44,0.13}{##1}}}
\expandafter\def\csname PYG@tok@c\endcsname{\let\PYG@it=\textit\def\PYG@tc##1{\textcolor[rgb]{0.25,0.50,0.56}{##1}}}
\expandafter\def\csname PYG@tok@mf\endcsname{\def\PYG@tc##1{\textcolor[rgb]{0.13,0.50,0.31}{##1}}}
\expandafter\def\csname PYG@tok@err\endcsname{\def\PYG@bc##1{\setlength{\fboxsep}{0pt}\fcolorbox[rgb]{1.00,0.00,0.00}{1,1,1}{\strut ##1}}}
\expandafter\def\csname PYG@tok@mb\endcsname{\def\PYG@tc##1{\textcolor[rgb]{0.13,0.50,0.31}{##1}}}
\expandafter\def\csname PYG@tok@ss\endcsname{\def\PYG@tc##1{\textcolor[rgb]{0.32,0.47,0.09}{##1}}}
\expandafter\def\csname PYG@tok@sr\endcsname{\def\PYG@tc##1{\textcolor[rgb]{0.14,0.33,0.53}{##1}}}
\expandafter\def\csname PYG@tok@mo\endcsname{\def\PYG@tc##1{\textcolor[rgb]{0.13,0.50,0.31}{##1}}}
\expandafter\def\csname PYG@tok@kd\endcsname{\let\PYG@bf=\textbf\def\PYG@tc##1{\textcolor[rgb]{0.00,0.44,0.13}{##1}}}
\expandafter\def\csname PYG@tok@mi\endcsname{\def\PYG@tc##1{\textcolor[rgb]{0.13,0.50,0.31}{##1}}}
\expandafter\def\csname PYG@tok@kn\endcsname{\let\PYG@bf=\textbf\def\PYG@tc##1{\textcolor[rgb]{0.00,0.44,0.13}{##1}}}
\expandafter\def\csname PYG@tok@cpf\endcsname{\let\PYG@it=\textit\def\PYG@tc##1{\textcolor[rgb]{0.25,0.50,0.56}{##1}}}
\expandafter\def\csname PYG@tok@kr\endcsname{\let\PYG@bf=\textbf\def\PYG@tc##1{\textcolor[rgb]{0.00,0.44,0.13}{##1}}}
\expandafter\def\csname PYG@tok@s\endcsname{\def\PYG@tc##1{\textcolor[rgb]{0.25,0.44,0.63}{##1}}}
\expandafter\def\csname PYG@tok@kp\endcsname{\def\PYG@tc##1{\textcolor[rgb]{0.00,0.44,0.13}{##1}}}
\expandafter\def\csname PYG@tok@w\endcsname{\def\PYG@tc##1{\textcolor[rgb]{0.73,0.73,0.73}{##1}}}
\expandafter\def\csname PYG@tok@kt\endcsname{\def\PYG@tc##1{\textcolor[rgb]{0.56,0.13,0.00}{##1}}}
\expandafter\def\csname PYG@tok@sc\endcsname{\def\PYG@tc##1{\textcolor[rgb]{0.25,0.44,0.63}{##1}}}
\expandafter\def\csname PYG@tok@sb\endcsname{\def\PYG@tc##1{\textcolor[rgb]{0.25,0.44,0.63}{##1}}}
\expandafter\def\csname PYG@tok@k\endcsname{\let\PYG@bf=\textbf\def\PYG@tc##1{\textcolor[rgb]{0.00,0.44,0.13}{##1}}}
\expandafter\def\csname PYG@tok@se\endcsname{\let\PYG@bf=\textbf\def\PYG@tc##1{\textcolor[rgb]{0.25,0.44,0.63}{##1}}}
\expandafter\def\csname PYG@tok@sd\endcsname{\let\PYG@it=\textit\def\PYG@tc##1{\textcolor[rgb]{0.25,0.44,0.63}{##1}}}

\def\PYGZbs{\char`\\}
\def\PYGZus{\char`\_}
\def\PYGZob{\char`\{}
\def\PYGZcb{\char`\}}
\def\PYGZca{\char`\^}
\def\PYGZam{\char`\&}
\def\PYGZlt{\char`\<}
\def\PYGZgt{\char`\>}
\def\PYGZsh{\char`\#}
\def\PYGZpc{\char`\%}
\def\PYGZdl{\char`\$}
\def\PYGZhy{\char`\-}
\def\PYGZsq{\char`\'}
\def\PYGZdq{\char`\"}
\def\PYGZti{\char`\~}
% for compatibility with earlier versions
\def\PYGZat{@}
\def\PYGZlb{[}
\def\PYGZrb{]}
\makeatother

\renewcommand\PYGZsq{\textquotesingle}

\begin{document}

\maketitle
\tableofcontents
\phantomsection\label{index::doc}


You can download this guide in pdf
\href{https://github.com/tomchor/ezkpp/raw/gh-pages/ezkpp.pdf}{here}.

You can acces the html version \href{https://tomchor.github.io/ezkpp/}{here}.

Contents:


\chapter{Introduction}
\label{README:introduction}\label{README:easy-guide-to-compiling-and-running-kpp}\label{README::doc}
This is an unofficial guide is aimed at providing additional information not
covered by the official KPP manual. We focus on the latest version of the
software, release 2.2.3, which can be freely downloaded at its \href{http://people.cs.vt.edu/~asandu/Software/Kpp/}{official
webpage}.

The source for the documentation can be found at its \href{https://github.com/tomchor/ezkpp}{github page} and the online html version can be
accessed \href{https://tomchor.github.io/ezkpp/}{here}.

We recommend any person reading this guide to keep a copy of the original
manual (which can be downloaded \href{http://people.cs.vt.edu/~asandu/Software/Kpp/}{here}) since this guide is not
meant to replace the original manual, but to supplement it.


\chapter{About bash}
\label{bash:about-bash}\label{bash::doc}
The official KPP manual is entirely based on Unix Shell, which is command
language which most of the Linux distrobutions use to interact with the system
without a Graphical User Interface. The manual, however, assumes a non-trivial
knowledge of this tool, which makes it difficult for users not experienced with
terminals and command line interfaces (which includes bash, C shell, MS-DOS,
PowerShell, ksh etc.) to install and run the simulations effectively. The
approach adopted in this guide will be to go through the steps necessary to
compile and run KPP, as stated in the manual, but taking the time to explain
them a little better how to do them, and what exactly it is that they do.

We will first go over a few basic notions necessary to understand what is going
to be done in the guide. If you are familiar with the concepts of system shells,
you may skip the next sections.


\section{What is Bash?}
\label{bash:what-is-bash}
But what is a shell? A system shell is the name what computer engineers use to
refer to the outter layer of an Operaional System (OS). It is said outter layer
because it separates the user (you) from the core of your OS. So it separates
you from the intricate group of codes that ultimately governs your machine and
lets you interact with your computer using a human-readable language (and not,
for example, binary!). So a shell is a bridge between you and your machine.

These shells can be either graphical shells (called Graphical User Interface,
GUI, just like what you use during mundane tasks such as browsing the web and
reading a PDF document) or text shells (also called terminals or Command Line
Interface, CLI). Graphical shells are easier and extremely intuitive (most
people use the mouse in a GUI and never needed to be told how to do it), but
they are very limited. Basically all you can do is click on buttons that were
previously programmed to to some task and input text.

Texts shells (terminals), however, are extremely powerful. You can do virtually
anything with your computer using them. That comes to the cost of terminals not
being intuitive at all. Since KPP is a complicated code for which there is no
graphical interface, we need to use a terminal to compile (``install'') and run
it, simply because this task requires a more powerful tool then your mouse.

Bash (acronym for Bourne Again Shell) is a kind of Unix Shell used by most of
the Linux systems and some Mac OSs. Some other shells can be used to perform
the same tasks (the KPP manual itself also gives some commands in C Shell,
which is another Unix Shell), but we focus on Bash here because it is the most
common and most easily accessible.  Most of Linux distributions use it, and
some Mac OSs use it as well. Furthermore, it can be nativelly installed into
Windows 10, as we will explain in the next section.


\section{Accessing Bash}
\label{bash:accessing-bash}
To access and use Bash, you either need a Bash emulator or to be in an
operational system that supports it natively. Various emulators exist (Cygwin,
cmder, MinGW, etc.) but they are not recommended because some of them contain
many bugs. If you would like to try those anyway, chances are that it'll work,
since we're going to be doing simple tasks and many people use those. However,
running it natively is a always a garantee of no bugs, so (in the spirit of
keeping it general) that's why it's the most recommended option for this guide.

We will briefly go through your options for each of the 3 most common
operational systems.


\subsection{From Windows}
\label{bash:from-windows}
Windows doesn't support Unix Shells natively by default, so here are the
options.

If you're using Windows 10, you can natively install the Ubuntu 14.04 inside
your Windows machine with the Windows 10 anniversary update, which is available
for every up-to-date Windows 10 computer. Directions to do this are very somple
and are given in many places (such as \href{http://www.howtogeek.com/249966/how-to-install-and-use-the-linux-bash-shell-on-windows-10/}{here})
so for now we will not explain them in detail. This will give you Bash running
natively on Windows. But only works for up-to-date Windows 10 computers.

If you do not have Windows 10, you can either install one of the many Bash
emulators for Windows or you can install a Linux virtual machine inside your
Windows computer.  You can do that using \href{https://www.virtualbox.org/wiki/Downloads}{Virtual Box} and installing a Ubuntu-based
distribution (we recommend installing either a recent version of Ubuntu or
Linux Mint 18 (or greater), since these two are most suited for beginners in
Linux). Again, directions on how to do this are straightfoward and exist all
over the internet, so we will not spend time on steps on how to do that.


\subsection{From Mac OS}
\label{bash:from-mac-os}
If you have a Mac, you might already have Bash natively installed, since all
Macs are based on Unix. To find out what your shell is, you need to open a
terminal application (generally under utilities). Then type the command \sphinxtitleref{echo
\$SHELL} and press enter. If the output of the shell is something ending in
Bash, like \sphinxtitleref{/bin/bash}, then you're already running Bash. If it ends in
something else, like \sphinxtitleref{/bin/ksh}, then you're running a different Unix Shell.
Most commands should be the same, but if you want to use this shell you might
have to translate some (which should be easy with the help of Google).

If you're running another terminal and would like to try Bash, you can either
get an Bash emulator for Mac, install a Linux virtual machine (as described in
the Windows section) or change your terminal to Bash. The most recomended here
is to change your Shell to Bash. Instructions on how to do this are easy and
can again be found in many places, including \href{http://osxdaily.com/2012/03/21/change-shell-mac-os-x/}{here}.


\subsection{From Linux}
\label{bash:from-linux}
If you're running Linux you can open a terminal and run the command \sphinxtitleref{echo
\$SHELL} to find out if you're running Bash or not. If you're not you can try to
keep going with your Shell (some commands may need to be translated) or you can
change your default Shell with the \sphinxtitleref{chsh} command. You can find more detailed
information on that in many places, such as \href{http://stackoverflow.com/questions/13046192/changing-default-shell-in-linux}{here}.


\chapter{Compiling KPP}
\label{compiling::doc}\label{compiling:compiling-kpp}
In this chapter we detail how to successfully download and compile KPP
on your system under the Bash environment.


\section{Downloading into your folder}
\label{compiling:downloading-into-your-folder}
One of the first things to be said is: most of the commands we will use will
only work if you're in the right directory (which we will always tell what it
is). So when you open a terminal, that terminal is ``running'' in some directory
in your computer.  You can find out which directory that is by entering the
command \sphinxcode{pwd} which stands for ``Print Working Directory''. That will show you
exactly where you are on your computer.

\begin{notice}{note}{Note:}
You can also use the \sphinxcode{ls} command, which will ``list'' everything you have on
that directory to get a better sense of where you are. Also, you can use the
command \sphinxcode{tree -d {}`pwd{}`}, which shows you your current directory on top, and
the subdirectories in it in a tree-like structure. Try it! This can also be used
to make you get a sense of where you are and what directories are ``around you''.
\end{notice}

To change directories, you can
use the command \sphinxcode{cd}, which stands for ``Change Directory''. So if you want to
go to your downloads directory, you can type \sphinxcode{cd Downloads}, or \sphinxcode{cd
/home/myuser/Downloads} depending on where you are on your terminal (the first
is a relative path (to your current location) and the second is an absolute or
full path; you can read more about relative and absolute paths \href{https://jeremywsherman.com/blog/2011/09/26/absolute-and-relative-paths/}{here}).

\begin{notice}{note}{Note:}
Throughout this document, we'll generally use \sphinxcode{myuser} to refer to
your username in the system. This generally comes right after \sphinxcode{/home/}
and you should change according to your case. So if your user name is \sphinxcode{john}
you'd replace \sphinxcode{/home/myuser} with \sphinxcode{/home/john} in every occasion.
\end{notice}

If you prefer to download KPP through its website manually and unpack it
somewhere, you'll have to go there with your terminal. So, if I unpack it in my
home directory, as soon as I open my terminal I'll have to use \sphinxcode{cd
/home/myuser/kpp-2.2.3}. This command will only work if the path is correct (it
might not work on Windows, for example, which does not have a \sphinxcode{/home} location.
If you're using Bash on Windows it's better to go with the following
alternative.

However, if you're insecure with navigating your directories using your
terminal, it's best to do everything via this second, more straightfoward,
option. It uses solely commands but it's easier. First, as soon as you open the
terminal run the following commands

\begin{Verbatim}[commandchars=\\\{\}]
wget http://people.cs.vt.edu/\PYGZti{}asandu/Software/Kpp/Download/kpp\PYGZhy{}2.2.3\PYGZus{}Nov.2012.zip
unzip kpp\PYGZhy{}2.2.3\PYGZus{}Nov.2012.zip
\PYG{n+nb}{cd} kpp\PYGZhy{}2.2.3
\end{Verbatim}

Line one will automatically download the software to your current directory and
line two will unpack it.  This will create a new directory with all the
contants of the \sphinxcode{.zip} file, so the last command line will move to the
recently-created directory.

Make sure you're in the correct directory by entering \sphinxcode{pwd}, which should show
you that you're on the \sphinxtitleref{kpp-2.2.3} directory. You can also type \sphinxtitleref{ls}, which should
show you a list of everything that was in the zip file:

\begin{Verbatim}[commandchars=\\\{\}]
\PYG{n}{cflags}        \PYG{n}{drv}       \PYG{n+nb}{int}                 \PYG{n}{Makefile}\PYG{o}{.}\PYG{n}{defs}  \PYG{n}{site}\PYG{o}{\PYGZhy{}}\PYG{n}{lisp}
\PYG{n}{cflags}\PYG{o}{.}\PYG{n}{guess}  \PYG{n}{examples}  \PYG{n+nb}{int}\PYG{o}{.}\PYG{n}{modified\PYGZus{}WCOPY}  \PYG{n}{models}         \PYG{n}{src}
\PYG{n}{doc}           \PYG{n}{gpl}       \PYG{n}{Makefile}            \PYG{n}{readme}         \PYG{n}{util}
\end{Verbatim}


\section{Making sure dependecies are installed}
\label{compiling:making-sure-dependecies-are-installed}
Now we are going to set-up the environment to compile KPP. The first step is to
make sure that you have the necessary software. These are called the
dependencies of a program: it is everything the program needs to be available
in the system (softwares, libraries, etc.) before it's installed.

Be sure that FLEX (which is a public domain \href{https://en.wikipedia.org/wiki/Lexical\_analysis}{lexical analizer}) is installed on your
machine. You can run \sphinxcode{flex -{-}version} and if it is installed you should see
something like \sphinxcode{flex 2.6.0}. If instead you see something like \sphinxcode{flex:
command not found} then it means that it is not installed and you're going to
have to install it by running \sphinxcode{sudo apt update \&\& sudo apt install flex} if
you're running Linux natively (depending on your Linux distribution) or by
manually installing downloading and installing the file if you're emulating
(with Cygwin, for example). A quick google search should tell you how to
install it easily. Note: if \sphinxcode{flex} isn't available for you, you might need to
install the Flex-dev package with \sphinxcode{sudo apt install flex-devel.x86\_64}.

Be also sure that \sphinxcode{yacc} and \sphinxcode{sed} are installed by typing \sphinxcode{which yacc}
and \sphinxcode{which sed}. If you see something like \sphinxcode{/usr/bin/sed} or
\sphinxcode{/usr/lib/yacc} then they are installed. If you see an error message, then
you're also going to have to install it manually. Again, a quick google search
should tell you how to do it, although it is very rare that these packages
aren't installed.

\begin{notice}{note}{Note:}
\sphinxcode{flex} and \sphinxcode{yacc} have to do with \href{https://en.wikipedia.org/wiki/Lexical\_analysis}{lexical analysis}
and it's not specially important to know exactly what they do. Suffices to
say that they are used internally by the compiler to generate the executable
file, but you will never have to use them directly when compiling/using KPP.
\sphinxcode{sed}, however, is a very useful \href{https://en.wikipedia.org/wiki/Sed}{text manipulation tool},
but you also won't need to use it
while running KPP.
\end{notice}


\section{Telling your system where KPP is}
\label{compiling:telling-your-system-where-kpp-is}
Now that we have the dependecies installed, we need to make sure that your
computer knows where KPP is in your system. We do that by altering a file
called \sphinxcode{.bashrc}. This file is a simple text file (so can you easily open and
read it, as you'll see) with some very simple commands. Every time you start a
terminal that file is ``read'' internally by the terminal and executed. So inside
that file you can put any command that you could type in the terminal. Thus,
generally, if you want to change something in your terminal so that the change
takes place every time you start it (so you don't have to re-change it over and
over again every time you open it), that's the place to do it. For our purposes
we simply need to add a couple of lines. We'll do that step by step.

\begin{notice}{note}{Note:}
If you're using a terminal other than Bash the \sphinxcode{.bashrc} file will probably
have a slightly different name (like \sphinxcode{.cshrc} e.g.) and the commands might
also differ a bit, but the process and the ideas are the same! You'll just have
to probably do some quick googling.
\end{notice}

First, in the directory where you unpacked KPP, run the command \sphinxcode{pwd} to
print the present working directory and copy its output. You'll need this for
the next steps.

Now you need to open and edit \sphinxcode{.bashrc} which can be done with many programs,
it really depends on what is installed for your (or what you would like to
install).  The best options would be an editor that runs with a GUI. For
Windows users the best option is probably \sphinxcode{notepad++}, while for Linux users
\sphinxcode{gedit} is generally the default GUI option. You can try these (and any other
GUI plain text editors you know) with the commands \sphinxcode{gedit \textasciitilde{}/.bashrc}, or
\sphinxcode{notepad++ \textasciitilde{}/.bashrc} and so forth with the others.

If any of those work, great!, you can edit the file in an intuitive GUI editor.
If not, you're either going to have to install a GUI text editor, or use Nano
by running the command \sphinxcode{nano \textasciitilde{}/.bashrc}. Nano is a very handy text editor
which runs on the terminal itself, however, it's not as eye-pleasing and not as
intuitive as the GUI-based ones.
\begin{figure}[htbp]
\centering
\capstart

\noindent\scalebox{0.900000}{\sphinxincludegraphics{{nano_example}.png}}
\caption{.bashrc example.}\label{compiling:nano-ex}\label{compiling:id1}\end{figure}

If you're forced to use Nano, you should probably read this very quick
\href{http://www.howtogeek.com/howto/42980/the-beginners-guide-to-nano-the-linux-command-line-text-editor/}{tutorial}
to learn how to open, save and close files. It's not as intuitive, but it's
very easy.

Once you open \sphinxcode{.bashrc}, you're going to see something like Fig.
{\hyperref[compiling:nano\string-ex]{\sphinxcrossref{\DUrole{std,std-ref}{.bashrc example.}}}} (in this case open with Nano). Don't worry about the lines of
code. They're probably going to be different for you and that's OK; it really
varies a lot from system to system. You can ignore all those codes and jump to
the last line of the file. After the last line you'll include the following

\begin{Verbatim}[commandchars=\\\{\}]
\PYG{n+nb}{export} \PYG{n+nv}{KPP\PYGZus{}HOME}\PYG{o}{=}\PYG{n+nv}{\PYGZdl{}HOME}/kpp\PYGZhy{}2.2.3
\PYG{n+nb}{export} \PYG{n+nv}{PATH}\PYG{o}{=}\PYG{n+nv}{\PYGZdl{}PATH}:\PYG{n+nv}{\PYGZdl{}KPP\PYGZus{}HOME}/bin
\end{Verbatim}

except that you should replace \sphinxcode{\$HOME/kpp} with the output of your \sphinxcode{pwd} command.
For example, if the output of \sphinxcode{pwd} was \sphinxcode{/home/myuser/Downloads/kpp-2.2.3} you
should write

\begin{Verbatim}[commandchars=\\\{\}]
\PYG{n+nb}{export} \PYG{n+nv}{KPP\PYGZus{}HOME}\PYG{o}{=}/home/user/Downloads/kpp\PYGZhy{}2.2.3
\PYG{n+nb}{export} \PYG{n+nv}{PATH}\PYG{o}{=}\PYG{n+nv}{\PYGZdl{}PATH}:\PYG{n+nv}{\PYGZdl{}KPP\PYGZus{}HOME}/bin
\end{Verbatim}

After this is done, you are going to save and exit. If you're using any option
with a GUI this should be straightforward. With Nano you can save and exit by
pressing control X, choosing the ``yes'' option (by hitting the ``y'' key) when it
asks you to save, and then pressing enter when asked to confirm to name of the
file to save to.

Now your terminal will know where KPP is the next times you start it. But for
the changes to make effect you need to close this terminal and open another
one. So just close the terminal you were working with, open a new one. Now, if
everything worked properly, you should be able to type \sphinxcode{cd \$KPP\_HOME} and go
automatically to your KPP directory. If this worked, we are ready for the next
step, which is telling your system how to compile KPP.


\section{Specifying how to compile}
\label{compiling:specifying-how-to-compile}
Now we actually compile (which is a way of installing) KPP. First, type
\sphinxcode{locate libfl.a} and save the output. If there is no output, use \sphinxcode{locate
libfl.sh} and save the output of that. These commands tell you where the Flex
library is, which we assured was installed somewhere in the system during the
last section. In my case the output was \sphinxcode{/usr/lib/x86\_64-linux-gnu/libfl.a}.

Now in your KPP directory, use the same text editor as before to open a file
called \sphinxcode{Makefile.defs}, which sets how Bash is going to make the executable
code for KPP (i.e., it only gives instructions to your computer on how to
compile it). So type \sphinxcode{gedit Makefile.defs}, or \sphinxcode{nano Makefile.defs} and so
on, depending on the editor you're using.

Once again, you'll see a lot of lines with comments, and the only lines that
matter are those that don't start with \sphinxcode{\#}. There should be 5 lines like this
in this file. The first one starts with \sphinxcode{CC}, which sets the C Compiler. In
this guide we will use the Gnu Compiler Collection, \sphinxcode{gcc}. So make sure that
the line which starts with \sphinxcode{CC} reads \sphinxcode{CC=gcc}.

Next, since we made sure that Flex was installed, make sure the next important
line reads \sphinxcode{FLEX=flex}. On the third step, set the next variable
(\sphinxcode{FLEX\_LIB\_DIR}) with the output we just saved without the last part. So in
my case the output saved was \sphinxcode{/usr/lib/x86\_64-linux-gnu/libfl.a}, so the line
will read \sphinxcode{FLEX\_LIB\_DIR=/usr/lib/x86\_64-linux-gnu}. You should, of course,
replace your line accordingly.

The next two items define some possible extra options for the compilation and
extra directories also to include in the compilation. We will don't have to
worry about those, unless maybe if we need to debug the program for some
reason. Now you can save and close/exit the file.

If we did everything correctly we can compile KPP simply by running the
\sphinxcode{make} command. Many warnings are going to appear on the screen (that's
normal), but as long as no error appears, the compilation will be successful.
You can be sure it was successsful by once again running \sphinxcode{ls} and seeing that
there is now one extra entry on the KPP directory called \sphinxcode{bin}:

\begin{Verbatim}[commandchars=\\\{\}]
bin           doc       gpl                 Makefile       readme     util
cflags        drv       int                 Makefile.defs  site\PYGZhy{}lisp
cflags.guess  examples  int.modified\PYGZus{}WCOPY  models         src
\end{Verbatim}

Now let's test it by running \sphinxcode{kpp test}. If the output is something like

\begin{Verbatim}[commandchars=\\\{\}]
\PYG{n}{This} \PYG{o+ow}{is} \PYG{n}{KPP}\PYG{o}{\PYGZhy{}}\PYG{l+m+mf}{2.2}\PYG{o}{.}\PYG{l+m+mf}{3.}

\PYG{n}{KPP} \PYG{o+ow}{is} \PYG{n}{parsing} \PYG{n}{the} \PYG{n}{equation} \PYG{n}{file}\PYG{o}{.}
\PYG{n}{Fatal} \PYG{n}{error} \PYG{p}{:} \PYG{n}{test}\PYG{p}{:} \PYG{n}{File} \PYG{o+ow}{not} \PYG{n}{found}
\PYG{n}{Program} \PYG{n}{aborted}
\end{Verbatim}

then we know it worked. This tells you the version of KPP and that it couldn't
find any file to work with, which is fine because we didn't give it any yet. If
this works, you can skip to the next section.

If, however you get an output similar to \sphinxcode{kpp: command not found...} then
chances are that \sphinxcode{bin} is a binary executable file, while it should be a
directory containing the binary file. This should not happen, according to the
manual, but for some reason it (very) often does. We need simply to rename that
executable file and put it a directory called \sphinxcode{bin}. This can be done with
the followinf command:

\begin{Verbatim}[commandchars=\\\{\}]
mv bin kpp \PYG{o}{\PYGZam{}\PYGZam{}} mkdir bin \PYG{o}{\PYGZam{}\PYGZam{}} mv kpp bin
\end{Verbatim}

Try this command and then try \sphinxcode{kpp test} again. You should get the correct
output this time, meaning that the system could find KPP successfully.
g


\chapter{Running KPP}
\label{running::doc}\label{running:running-kpp}
Now that KPP is properly compiled, we proceed to running the first test case
to make sure it works!


\section{The first test case}
\label{running:the-first-test-case}
We now follow the manual and begin running the Chapman stratospheric mechanism
as a test case. This will allow us to illustrate some key features when running
KPP.

In order to run a simulation on KPP, it needs three things:
\begin{itemize}
\item {} 
a \sphinxcode{.kpp} file (from the KPP directory, type \sphinxcode{ls examples} to see some examples of those)

\item {} 
a \sphinxcode{.spc} file (type \sphinxcode{ls models} to see some examples of those)

\item {} 
a \sphinxcode{.eqn} file (type \sphinxcode{ls models} to see some examples of those)

\end{itemize}

We begin by creating a directory to run this first test. Let's call this
directory \sphinxcode{test1} and create it with \sphinxcode{mkdir test1} (this new directory can
be created anywhere!). We then go to that directory with \sphinxcode{cd test1}. Let's
follow the manual and create a file called \sphinxcode{small\_strato.kpp} with the
following contents:

\begin{Verbatim}[commandchars=\\\{\}]
\PYG{c+cp}{\PYGZsh{}MODEL      small\PYGZus{}strato}
\PYG{c+cp}{\PYGZsh{}LANGUAGE   Fortran90}
\PYG{c+cp}{\PYGZsh{}INTEGRATOR rosenbrock}
\end{Verbatim}

You can do this by typing \sphinxcode{nano small\_strato.kpp} in the \sphinxcode{test1} directory,
if using Nano, or by using another editor of your choice (replace \sphinxcode{nano} with
\sphinxcode{notepad++} for example). Then just paste the content above in the file, save
it and exit it.

This file tells KPP what model to use (\sphinxcode{small\_strato.def}) and how to process
it (most importantly for us here, it tells KPP to generate a Fortran 90 code).
Many other options can be added to this file and you can learn more about them
in the KPP manual.

If our changes to \sphinxcode{.bashrc} are correct, then KPP should be able to find the
correct model, since the \sphinxcode{small\_strato} model (given by \sphinxcode{small\_strato.def})
is located in the \sphinxcode{models} directory, in the KPP home directory. We test this
by running KPP on our recently created file with

\begin{Verbatim}[commandchars=\\\{\}]
\PYG{n}{kpp} \PYG{n}{small\PYGZus{}strato}\PYG{o}{.}\PYG{n}{kpp}
\end{Verbatim}

You should see the following lines on your screen:

\begin{Verbatim}[commandchars=\\\{\}]
\PYG{n}{This} \PYG{o+ow}{is} \PYG{n}{KPP}\PYG{o}{\PYGZhy{}}\PYG{l+m+mf}{2.2}\PYG{o}{.}\PYG{l+m+mf}{3.}

\PYG{n}{KPP} \PYG{o+ow}{is} \PYG{n}{parsing} \PYG{n}{the} \PYG{n}{equation} \PYG{n}{file}\PYG{o}{.}
\PYG{n}{KPP} \PYG{o+ow}{is} \PYG{n}{computing} \PYG{n}{Jacobian} \PYG{n}{sparsity} \PYG{n}{structure}\PYG{o}{.}
\PYG{n}{KPP} \PYG{o+ow}{is} \PYG{n}{starting} \PYG{n}{the} \PYG{n}{code} \PYG{n}{generation}\PYG{o}{.}
\PYG{n}{KPP} \PYG{o+ow}{is} \PYG{n}{initializing} \PYG{n}{the} \PYG{n}{code} \PYG{n}{generation}\PYG{o}{.}
\PYG{n}{KPP} \PYG{o+ow}{is} \PYG{n}{generating} \PYG{n}{the} \PYG{n}{monitor} \PYG{n}{data}\PYG{p}{:}
    \PYG{o}{\PYGZhy{}} \PYG{n}{small\PYGZus{}strato\PYGZus{}Monitor}
\PYG{n}{KPP} \PYG{o+ow}{is} \PYG{n}{generating} \PYG{n}{the} \PYG{n}{utility} \PYG{n}{data}\PYG{p}{:}
    \PYG{o}{\PYGZhy{}} \PYG{n}{small\PYGZus{}strato\PYGZus{}Util}
\PYG{n}{KPP} \PYG{o+ow}{is} \PYG{n}{generating} \PYG{n}{the} \PYG{k}{global} \PYG{n}{declarations}\PYG{p}{:}
    \PYG{o}{\PYGZhy{}} \PYG{n}{small\PYGZus{}strato\PYGZus{}Main}
\PYG{n}{KPP} \PYG{o+ow}{is} \PYG{n}{generating} \PYG{n}{the} \PYG{n}{ODE} \PYG{n}{function}\PYG{p}{:}
    \PYG{o}{\PYGZhy{}} \PYG{n}{small\PYGZus{}strato\PYGZus{}Function}
\PYG{n}{KPP} \PYG{o+ow}{is} \PYG{n}{generating} \PYG{n}{the} \PYG{n}{ODE} \PYG{n}{Jacobian}\PYG{p}{:}
    \PYG{o}{\PYGZhy{}} \PYG{n}{small\PYGZus{}strato\PYGZus{}Jacobian}
    \PYG{o}{\PYGZhy{}} \PYG{n}{small\PYGZus{}strato\PYGZus{}JacobianSP}
\PYG{n}{KPP} \PYG{o+ow}{is} \PYG{n}{generating} \PYG{n}{the} \PYG{n}{linear} \PYG{n}{algebra} \PYG{n}{routines}\PYG{p}{:}
    \PYG{o}{\PYGZhy{}} \PYG{n}{small\PYGZus{}strato\PYGZus{}LinearAlgebra}
\PYG{n}{KPP} \PYG{o+ow}{is} \PYG{n}{generating} \PYG{n}{the} \PYG{n}{Hessian}\PYG{p}{:}
    \PYG{o}{\PYGZhy{}} \PYG{n}{small\PYGZus{}strato\PYGZus{}Hessian}
    \PYG{o}{\PYGZhy{}} \PYG{n}{small\PYGZus{}strato\PYGZus{}HessianSP}
\PYG{n}{KPP} \PYG{o+ow}{is} \PYG{n}{generating} \PYG{n}{the} \PYG{n}{utility} \PYG{n}{functions}\PYG{p}{:}
    \PYG{o}{\PYGZhy{}} \PYG{n}{small\PYGZus{}strato\PYGZus{}Util}
\PYG{n}{KPP} \PYG{o+ow}{is} \PYG{n}{generating} \PYG{n}{the} \PYG{n}{rate} \PYG{n}{laws}\PYG{p}{:}
    \PYG{o}{\PYGZhy{}} \PYG{n}{small\PYGZus{}strato\PYGZus{}Rates}
\PYG{n}{KPP} \PYG{o+ow}{is} \PYG{n}{generating} \PYG{n}{the} \PYG{n}{parameters}\PYG{p}{:}
    \PYG{o}{\PYGZhy{}} \PYG{n}{small\PYGZus{}strato\PYGZus{}Parameters}
\PYG{n}{KPP} \PYG{o+ow}{is} \PYG{n}{generating} \PYG{n}{the} \PYG{k}{global} \PYG{n}{data}\PYG{p}{:}
    \PYG{o}{\PYGZhy{}} \PYG{n}{small\PYGZus{}strato\PYGZus{}Global}
\PYG{n}{KPP} \PYG{o+ow}{is} \PYG{n}{generating} \PYG{n}{the} \PYG{n}{stoichiometric} \PYG{n}{description} \PYG{n}{files}\PYG{p}{:}
    \PYG{o}{\PYGZhy{}} \PYG{n}{small\PYGZus{}strato\PYGZus{}Stoichiom}
    \PYG{o}{\PYGZhy{}} \PYG{n}{small\PYGZus{}strato\PYGZus{}StoichiomSP}
\PYG{n}{KPP} \PYG{o+ow}{is} \PYG{n}{generating} \PYG{n}{the} \PYG{n}{driver} \PYG{k+kn}{from} \PYG{n+nn}{none}\PYG{n+nn}{.}\PYG{n+nn}{f90}\PYG{p}{:}
    \PYG{o}{\PYGZhy{}} \PYG{n}{small\PYGZus{}strato\PYGZus{}Main}
\PYG{n}{KPP} \PYG{o+ow}{is} \PYG{n}{starting} \PYG{n}{the} \PYG{n}{code} \PYG{n}{post}\PYG{o}{\PYGZhy{}}\PYG{n}{processing}\PYG{o}{.}

\PYG{n}{KPP} \PYG{n}{has} \PYG{n}{succesfully} \PYG{n}{created} \PYG{n}{the} \PYG{n}{model} \PYG{l+s+s2}{\PYGZdq{}}\PYG{l+s+s2}{small\PYGZus{}strato}\PYG{l+s+s2}{\PYGZdq{}}\PYG{o}{.}
\end{Verbatim}

If indeed you see this (or something similar) it means you were successful in
creating the model. Now if you type \sphinxcode{ls}, you'll see many new files:

\begin{Verbatim}[commandchars=\\\{\}]
\PYG{n}{Makefile\PYGZus{}small\PYGZus{}strato}           \PYG{n}{small\PYGZus{}strato}\PYG{o}{.}\PYG{n}{map}
\PYG{n}{small\PYGZus{}strato\PYGZus{}Function}\PYG{o}{.}\PYG{n}{f90}       \PYG{n}{small\PYGZus{}strato\PYGZus{}mex\PYGZus{}Fun}\PYG{o}{.}\PYG{n}{f90}
\PYG{n}{small\PYGZus{}strato\PYGZus{}Global}\PYG{o}{.}\PYG{n}{f90}         \PYG{n}{small\PYGZus{}strato\PYGZus{}mex\PYGZus{}Hessian}\PYG{o}{.}\PYG{n}{f90}
\PYG{n}{small\PYGZus{}strato\PYGZus{}Hessian}\PYG{o}{.}\PYG{n}{f90}        \PYG{n}{small\PYGZus{}strato\PYGZus{}mex\PYGZus{}Jac\PYGZus{}SP}\PYG{o}{.}\PYG{n}{f90}
\PYG{n}{small\PYGZus{}strato\PYGZus{}HessianSP}\PYG{o}{.}\PYG{n}{f90}      \PYG{n}{small\PYGZus{}strato\PYGZus{}Model}\PYG{o}{.}\PYG{n}{f90}
\PYG{n}{small\PYGZus{}strato\PYGZus{}Initialize}\PYG{o}{.}\PYG{n}{f90}     \PYG{n}{small\PYGZus{}strato\PYGZus{}Monitor}\PYG{o}{.}\PYG{n}{f90}
\PYG{n}{small\PYGZus{}strato\PYGZus{}Integrator}\PYG{o}{.}\PYG{n}{f90}     \PYG{n}{small\PYGZus{}strato\PYGZus{}Parameters}\PYG{o}{.}\PYG{n}{f90}
\PYG{n}{small\PYGZus{}strato\PYGZus{}Jacobian}\PYG{o}{.}\PYG{n}{f90}       \PYG{n}{small\PYGZus{}strato\PYGZus{}Precision}\PYG{o}{.}\PYG{n}{f90}
\PYG{n}{small\PYGZus{}strato\PYGZus{}JacobianSP}\PYG{o}{.}\PYG{n}{f90}     \PYG{n}{small\PYGZus{}strato\PYGZus{}Rates}\PYG{o}{.}\PYG{n}{f90}
\PYG{n}{small\PYGZus{}strato}\PYG{o}{.}\PYG{n}{kpp}                \PYG{n}{small\PYGZus{}strato\PYGZus{}Stoichiom}\PYG{o}{.}\PYG{n}{f90}
\PYG{n}{small\PYGZus{}strato\PYGZus{}LinearAlgebra}\PYG{o}{.}\PYG{n}{f90}  \PYG{n}{small\PYGZus{}strato\PYGZus{}StoichiomSP}\PYG{o}{.}\PYG{n}{f90}
\PYG{n}{small\PYGZus{}strato\PYGZus{}Main}\PYG{o}{.}\PYG{n}{f90}           \PYG{n}{small\PYGZus{}strato\PYGZus{}Util}\PYG{o}{.}\PYG{n}{f90}
\end{Verbatim}

Most of them end with a \sphinxcode{.f90} extension, which tells us they are Fortran 90
codes. These codes have to be compiled into an executable file which is what
will actually process and run the kinetic model. So the next step is to compile
every one of those code together into one executable and run it.

Let's focus for now on the \sphinxcode{Makefile\_small\_strato}. This is a text file that
tells your computer which Fortran compiler to use to compile, some options and
etc. Open the \sphinxcode{Makefile\_small\_strato} file and find where it says

\begin{Verbatim}[commandchars=\\\{\}]
\PYG{c+c1}{\PYGZsh{}COMPILER = G95}
\PYG{c+c1}{\PYGZsh{}COMPILER = LAHEY}
\PYG{n+nv}{COMPILER} \PYG{o}{=} INTEL
\PYG{c+c1}{\PYGZsh{}COMPILER = PGF}
\PYG{c+c1}{\PYGZsh{}COMPILER = HPUX}
\PYG{c+c1}{\PYGZsh{}COMPILER = GFORTRAN}
\end{Verbatim}

Each of the lines is a different Fortran compiler, and your computer is only
going to see the line that doesn't start with a \sphinxcode{\#} (we say that the lines
with \sphinxcode{\#} are commented and therefore the computer doesn't ``see'' them). So,
currently, these lines are telling the computer to use the Intel Fortran
compiler, \sphinxcode{ifort}.

If you are using \sphinxcode{ifort}, you should leave it as it is. Since \sphinxcode{ifort} is
paid, chances are you are using another compiler. If this is the case, put the
\sphinxcode{\#} in front of the \sphinxcode{INTEL} options and take it out of the line which has
the name of your compiler. If you don't know which compiler you have, chances
are you have gfortran, which is free and what we will use here. You can also
install gfortran with \sphinxcode{sudo apt install gfortran} (or the equivalent
installation command for your system).

Since gfortran is the most common compiler, we will assume here that you're
using it. So, for gfortran, you should make the above lines of code look like
the following:

\begin{Verbatim}[commandchars=\\\{\}]
\PYG{c+c1}{\PYGZsh{}COMPILER = G95}
\PYG{c+c1}{\PYGZsh{}COMPILER = LAHEY}
\PYG{c+c1}{\PYGZsh{}COMPILER = INTEL}
\PYG{c+c1}{\PYGZsh{}COMPILER = PGF}
\PYG{c+c1}{\PYGZsh{}COMPILER = HPUX}
\PYG{n+nv}{COMPILER} \PYG{o}{=} GFORTRAN
\end{Verbatim}

When doing that we say that we ``uncommented'' the gfortran line.
You can save and exit the file.

Now all you have to do is run \sphinxcode{make -f Makefile\_small\_strato}, which will
compile your Fortran code into an executable using the options we just set.
You should see a lot of lines appearing on screen starting with \sphinxcode{gfortran}
and if no error messages appear the compilation was successful.

Now you'll see many more new files, including one called \sphinxcode{small\_strato.exe},
which is your executable file (run \sphinxcode{ls} again to see that). This is the
executable that will actually calculate the concentrations using the model.

To test if it works, run \sphinxcode{./small\_strato.exe}, which will run the executable.
You should see some output on the screen with concentrations, like Fig. {\hyperref[running:test1\string-output]{\sphinxcrossref{\DUrole{std,std-ref}{Output concentrations of the first test case.}}}}
\begin{figure}[htbp]
\centering
\capstart

\noindent\scalebox{0.900000}{\sphinxincludegraphics{{test1_output}.png}}
\caption{Output concentrations of the first test case.}\label{running:test1-output}\label{running:id1}\end{figure}

If this is the case, then your run was successful and everything worked well!
You just calculated the concentrations of the compounds in the \sphinxcode{small\_strato}
model with the pre-defined initial conditions.


\section{Understanding the test case}
\label{running:understanding-the-test-case}
Now let's understand why our run of \sphinxcode{small\_strato.exe} was successful and what
happened. First, by running \sphinxcode{kpp small\_strato}, what we did was to tell KPP
to open a file called \sphinxcode{small\_strato.kpp}, in the current directory and do what that
file tells it to do. In the first line of the file there is the command

\begin{Verbatim}[commandchars=\\\{\}]
\PYG{c+c1}{\PYGZsh{}MODEL      small\PYGZus{}strato}
\end{Verbatim}

which tells KPP to look in the directory containing its models (located at
\sphinxcode{\$KPP\_HOME/models}, according to the changes we made before in the
\sphinxcode{.bashrc} file) for a file called \sphinxcode{small\_strato.def}.  Since the file is
there, KPP had no problems finding it. This file has the initial concentrations
you want to use in the model, the time step etc.. It also links two other files
(\sphinxcode{small\_strato.spc} and \sphinxcode{small\_strato.eqn}), which tell KPP with chemical
species and chemical equations to use.

After receiving all that information, KPP finally creates a Fortran 90 code
(because it says so in the \sphinxcode{small\_strato.kpp} we created) with our small
stratospheric model containing our pre-defined initial conditions, time step,
chemical reactions and so on.

The code, however, has to be compiled before run, so that is why we issued the
command \sphinxcode{make}, which compiles the code according to the file
\sphinxcode{Makefile\_small\_strato} (which is where we specified the Fortran compiler).
This step creates an executable file, which has the extension \sphinxcode{.exe} and is
ready to be run. By running the \sphinxcode{.exe} file we ran a program that got our
initial concentrations of the species we defined and, based on the chemical
reactions, calculated, step by step, their concentrations in each time step.

At each step, the model is not only printing the concentrations on screen, but
it is also writing them into a file called \sphinxcode{small\_strato.dat}, which is a
column-separated text file. This file can be used to see, plot, make
calculations with the data and so on. However, you should be careful because
the order of the concentrations that appear on screen isn't the same order KPP
uses for the \sphinxcode{.dat} file. You can check the correct order (and learn how to
change it) at page 7 of the KPP manual. You can also figure out what the first
line of the file looks like, since it should match the initial conditions you
set.

In the case of \sphinxcode{small\_strato} the order printed on the file is

\begin{Verbatim}[commandchars=\\\{\}]
\PYG{n}{time}\PYG{p}{,} \PYG{n}{O1D}\PYG{p}{,} \PYG{n}{O}\PYG{p}{,} \PYG{n}{O3}\PYG{p}{,} \PYG{n}{NO}\PYG{p}{,} \PYG{n}{NO2}\PYG{p}{,} \PYG{n}{M}\PYG{p}{,} \PYG{n}{O2}
\end{Verbatim}

The time is always going to be the first column, and it is always going to be
in hours since the start of the simulation. Since the solar forcing matters
here, we need to keep track of the time of day the day that the simulation
started. In this case it was at noon.

We can read that data in many ways. I present below a quick python script
to plot the concentrations as a function of the hour of the day

\begin{Verbatim}[commandchars=\\\{\}]
\PYG{k+kn}{import} \PYG{n+nn}{pandas} \PYG{k}{as} \PYG{n+nn}{pd}
\PYG{k+kn}{from} \PYG{n+nn}{matplotlib} \PYG{k}{import} \PYG{n}{pyplot} \PYG{k}{as} \PYG{n}{plt}
\PYG{n}{concs} \PYG{o}{=} \PYG{n}{pd}\PYG{o}{.}\PYG{n}{read\PYGZus{}csv}\PYG{p}{(}\PYG{l+s+s1}{\PYGZsq{}}\PYG{l+s+s1}{small\PYGZus{}strato.dat}\PYG{l+s+s1}{\PYGZsq{}}\PYG{p}{,} \PYG{n}{index\PYGZus{}col}\PYG{o}{=}\PYG{l+m+mi}{0}\PYG{p}{,} \PYG{n}{delim\PYGZus{}whitespace}\PYG{o}{=}\PYG{k+kc}{True}\PYG{p}{,} \PYG{n}{header}\PYG{o}{=}\PYG{k+kc}{None}\PYG{p}{)}\PYG{o}{.}\PYG{n}{apply}\PYG{p}{(}\PYG{n}{pd}\PYG{o}{.}\PYG{n}{to\PYGZus{}numeric}\PYG{p}{,} \PYG{n}{errors}\PYG{o}{=}\PYG{l+s+s1}{\PYGZsq{}}\PYG{l+s+s1}{coerce}\PYG{l+s+s1}{\PYGZsq{}}\PYG{p}{)}
\PYG{n}{concs}\PYG{o}{.}\PYG{n}{columns} \PYG{o}{=} \PYG{p}{[}\PYG{l+s+s1}{\PYGZsq{}}\PYG{l+s+s1}{O1D}\PYG{l+s+s1}{\PYGZsq{}}\PYG{p}{,} \PYG{l+s+s1}{\PYGZsq{}}\PYG{l+s+s1}{O}\PYG{l+s+s1}{\PYGZsq{}}\PYG{p}{,} \PYG{l+s+s1}{\PYGZsq{}}\PYG{l+s+s1}{O3}\PYG{l+s+s1}{\PYGZsq{}}\PYG{p}{,} \PYG{l+s+s1}{\PYGZsq{}}\PYG{l+s+s1}{NO}\PYG{l+s+s1}{\PYGZsq{}}\PYG{p}{,} \PYG{l+s+s1}{\PYGZsq{}}\PYG{l+s+s1}{NO2}\PYG{l+s+s1}{\PYGZsq{}}\PYG{p}{,} \PYG{l+s+s1}{\PYGZsq{}}\PYG{l+s+s1}{M}\PYG{l+s+s1}{\PYGZsq{}}\PYG{p}{,} \PYG{l+s+s1}{\PYGZsq{}}\PYG{l+s+s1}{O2}\PYG{l+s+s1}{\PYGZsq{}}\PYG{p}{]}
\PYG{n}{concs}\PYG{o}{.}\PYG{n}{index}\PYG{o}{.}\PYG{n}{name} \PYG{o}{=} \PYG{l+s+s1}{\PYGZsq{}}\PYG{l+s+s1}{Hours since noon}\PYG{l+s+s1}{\PYGZsq{}}
\PYG{n}{concs}\PYG{o}{.}\PYG{n}{plot}\PYG{p}{(}\PYG{n}{ylim}\PYG{o}{=}\PYG{p}{[}\PYG{l+m+mf}{1.e8}\PYG{p}{,} \PYG{k+kc}{None}\PYG{p}{]}\PYG{p}{,} \PYG{n}{logy}\PYG{o}{=}\PYG{k+kc}{True}\PYG{p}{,} \PYG{n}{y}\PYG{o}{=}\PYG{p}{[}\PYG{l+s+s1}{\PYGZsq{}}\PYG{l+s+s1}{O3}\PYG{l+s+s1}{\PYGZsq{}}\PYG{p}{,} \PYG{l+s+s1}{\PYGZsq{}}\PYG{l+s+s1}{NO}\PYG{l+s+s1}{\PYGZsq{}}\PYG{p}{,} \PYG{l+s+s1}{\PYGZsq{}}\PYG{l+s+s1}{NO2}\PYG{l+s+s1}{\PYGZsq{}}\PYG{p}{]}\PYG{p}{,} \PYG{n}{grid}\PYG{o}{=}\PYG{k+kc}{True}\PYG{p}{)}
\PYG{n}{plt}\PYG{o}{.}\PYG{n}{savefig}\PYG{p}{(}\PYG{l+s+s1}{\PYGZsq{}}\PYG{l+s+s1}{test1\PYGZus{}time.png}\PYG{l+s+s1}{\PYGZsq{}}\PYG{p}{)}
\end{Verbatim}

\begin{notice}{note}{Note:}
KPP has a small issue with formatting and sometimes prints a number that can't be read because
some strings are missing. For example, printing \sphinxcode{3.4562-313}. This can't be normally read
and it's supposed to be \sphinxcode{3.4562E-313} and this (apparently) only happens when the number
is close to machine-precision (which we would interpret as zero). The program above takes
this issue into consideration when reading the file, but you should pay attention when trying
to read with by other means.
\end{notice}

If you have ever seen python before, this code should be pretty intuitive. If
you haven't you can still use it easily (maybe you just have to download
python's \sphinxcode{pandas} package).  This code generates the following plot of the
concentrations:
\phantomsection\label{running:test1-time}\begin{figure}[htbp]
\centering

\noindent\scalebox{0.800000}{\sphinxincludegraphics{{test1_time}.png}}
\label{running:test1-time}\end{figure}

We can see that the NOx concentrations follow the solar cycle, which is
indicative that the model is indeed working properly. However we see that the
O3 concentrations still haven't stabilized. This tells us that we need to run
the model for longer. Let us take this chance to modify the \sphinxcode{small\_strato}
example a bit, try and make the O3 concentrations stabilize and learn how to
alter/create models.


\chapter{Modifying and improving the example}
\label{improving:modifying-and-improving-the-example}\label{improving::doc}

\section{Increasing the length of the simulation}
\label{improving:increasing-the-length-of-the-simulation}
There are two ways to make modifications on the model. The first, which works
for simple changes, is to modify the Fortran code itself. The second is to
change the KPP model itself (the \sphinxcode{.def} files etc.) before it gets compiled.
This latter method is more general, so this is the one we will focus on this
guide.  Since all we want for now is to increase the total time, we will base
ourselves in the original \sphinxcode{small\_strato} model and only modify this
parameter.

First, create another directory (anywhere you want) called \sphinxcode{test2} (with
\sphinxcode{mkdir test2}) and enter it (with \sphinxcode{cd test2}). Now create a file called
\sphinxcode{my\_strato.kpp} (with \sphinxcode{notepad++ my\_strato.kpp} or \sphinxcode{gedit my\_strato.kpp}
or whichever text editor you ended up using) and paste the following lines in
the file

\begin{Verbatim}[commandchars=\\\{\}]
\PYG{c+c1}{\PYGZsh{}MODEL      my\PYGZus{}strato}
\PYG{c+c1}{\PYGZsh{}LANGUAGE   Fortran90}
\PYG{c+c1}{\PYGZsh{}INTEGRATOR rosenbrock}
\end{Verbatim}

At this point if you run \sphinxcode{kpp my\_strato.kpp} you should get an error saying
\sphinxcode{"Fatal error : my\_strato.def: Can't read file"}. Which appears because we
instructed KPP to search for the file \sphinxcode{my\_strato.def}, which doesn't exist in
the models directory. So we first must create the \sphinxcode{my\_strato.def} file, which
ultimately defines the \sphinxcode{my\_strato} model.

Let us define our model based on \sphinxcode{small\_strato}, since for now all we want to
do is to modify the time length of the simulation. In order to preserve the
original \sphinxcode{small\_strato.def} we'll copy it and call it \sphinxcode{my\_strato.def}, this
way we can do any modification on \sphinxcode{my\_strato} and the original
\sphinxcode{small\_strato} will be safe. You can copy the file in the \sphinxcode{models}
directory by issuing the following command:

\begin{Verbatim}[commandchars=\\\{\}]
cp \PYGZdl{}KPP\PYGZus{}HOME/models/small\PYGZus{}strato.def \PYGZdl{}KPP\PYGZus{}HOME/models/my\PYGZus{}strato.def
\end{Verbatim}

Now you should open the file we just created (for example with \sphinxcode{notepad++})
and find the lines that look like

\begin{Verbatim}[commandchars=\\\{\}]
\PYG{c+c1}{\PYGZsh{}INLINE F90\PYGZus{}INIT}
        \PYG{n}{TSTART} \PYG{o}{=} \PYG{p}{(}\PYG{l+m+mi}{12}\PYG{o}{*}\PYG{l+m+mi}{3600}\PYG{p}{)}
        \PYG{n}{TEND} \PYG{o}{=} \PYG{n}{TSTART} \PYG{o}{+} \PYG{p}{(}\PYG{l+m+mi}{3}\PYG{o}{*}\PYG{l+m+mi}{24}\PYG{o}{*}\PYG{l+m+mi}{3600}\PYG{p}{)}
        \PYG{n}{DT} \PYG{o}{=} \PYG{l+m+mf}{0.25}\PYG{o}{*}\PYG{l+m+mi}{3600}
        \PYG{n}{TEMP} \PYG{o}{=} \PYG{l+m+mi}{270}
\PYG{c+c1}{\PYGZsh{}ENDINLINE}
\end{Verbatim}

These are the lines that define the start, end, time step and global
temperature of the model. For now, we'll change only the end time. Notice that
they are given in seconds. For example, \sphinxcode{3*24*3600} is the amount of seconds
in 3 days, meaning that at the moment the simulation is set to run for 3 days,
which we saw is not enough.  Let us then replace \sphinxcode{3} with \sphinxcode{30}, meaning we
will run it for 30 days, to be sure that equilibrium is reached. Now that line
should read \sphinxcode{TEND = TSTART + (30*24*3600)}.

After this small change we are ready to test run the model. Run \sphinxcode{kpp
my\_strato.kpp}, which should end with a \sphinxcode{succesfully created the model}
message. Now, just like with the previous example, open again the file
\sphinxcode{Makefile\_my\_strato} and uncomment the line that says \sphinxcode{COMPILER =
GFORTRAN}, so we can use Gfortran instead of Intel. After this is done compile
the model with \sphinxcode{make -f Makefile\_my\_strato}. Again, if everything goes well,
the \sphinxcode{my\_strato.exe} file should be created.

We can now run the model with \sphinxcode{./my\_strato.exe}, which should now take 10
times longer to complete, since we are running it for 10 times as long as
before.  We can use the Python code given in the last section to read the
results. We just need to adjust the name of the file inside the code from
\sphinxcode{small\_strato} to \sphinxcode{my\_strato}.

The plot of the results is
\phantomsection\label{improving:test2-time}\begin{figure}[htbp]
\centering

\noindent\scalebox{0.800000}{\sphinxincludegraphics{{test2_time}.png}}
\label{improving:test2-time}\end{figure}

Now we can see that the solution stabilizes after roughly 200 hours! This was a
minor change in the model. Let us now change some other things and see how the
model reacts.


\section{Change in the initial conditions}
\label{improving:change-in-the-initial-conditions}
We can use the same model as before (\sphinxcode{my\_strato}). Let's open the \sphinxcode{.def} file
with located at \sphinxcode{\$KPP\_HOME/models/my\_strato.def} and consider an atmosphere with
more NO (simulating a polluted condition). Where it says \sphinxcode{NO  = 8.725E+08 ;}, we make it
read \sphinxcode{NO  = 9.00E+09}, which is roughly a 10 times increase in the NO concentration. Let us
also change the O3 initial condition and make the O3 line read \sphinxcode{O3  = 5.00E+10 ;}, simulating
an atmosphere that has a lower initial O3 concentration.

Let us also change the line that reads

\begin{Verbatim}[commandchars=\\\{\}]
\PYG{c+c1}{\PYGZsh{}LOOKATALL          \PYGZob{}File Output\PYGZcb{}}
\end{Verbatim}

We will make it read

\begin{Verbatim}[commandchars=\\\{\}]
\PYG{c+c1}{\PYGZsh{}LOOKAT O3; NO; NO2; \PYGZob{}File Output\PYGZcb{}}
\end{Verbatim}

This latter change makes the program write only time, O3, NO and NO2 into the
output file, in that order. This simplifies the reading process and saves
space, since we are assuming that we're not interested in the other species.

We again go through the same steps: run it with \sphinxcode{kpp my\_strato.kpp}, change the compiler
to gfortran, compile if with \sphinxcode{make -f Makefile\_my\_strato} and run it with \sphinxcode{./my\_strato.exe}.
This time, since the output file changed, we have to change the code to read it correctly:

\begin{Verbatim}[commandchars=\\\{\}]
\PYG{k+kn}{import} \PYG{n+nn}{pandas} \PYG{k}{as} \PYG{n+nn}{pd}
\PYG{k+kn}{from} \PYG{n+nn}{matplotlib} \PYG{k}{import} \PYG{n}{pyplot} \PYG{k}{as} \PYG{n}{plt}
\PYG{n}{concs} \PYG{o}{=} \PYG{n}{pd}\PYG{o}{.}\PYG{n}{read\PYGZus{}csv}\PYG{p}{(}\PYG{l+s+s1}{\PYGZsq{}}\PYG{l+s+s1}{my\PYGZus{}strato.dat}\PYG{l+s+s1}{\PYGZsq{}}\PYG{p}{,} \PYG{n}{index\PYGZus{}col}\PYG{o}{=}\PYG{l+m+mi}{0}\PYG{p}{,} \PYG{n}{delim\PYGZus{}whitespace}\PYG{o}{=}\PYG{k+kc}{True}\PYG{p}{,} \PYG{n}{header}\PYG{o}{=}\PYG{k+kc}{None}\PYG{p}{,} \PYG{n}{dtype}\PYG{o}{=}\PYG{k+kc}{None}\PYG{p}{)}\PYG{o}{.}\PYG{n}{apply}\PYG{p}{(}\PYG{n}{pd}\PYG{o}{.}\PYG{n}{to\PYGZus{}numeric}\PYG{p}{,} \PYG{n}{errors}\PYG{o}{=}\PYG{l+s+s1}{\PYGZsq{}}\PYG{l+s+s1}{coerce}\PYG{l+s+s1}{\PYGZsq{}}\PYG{p}{)}
\PYG{n}{concs}\PYG{o}{.}\PYG{n}{columns} \PYG{o}{=} \PYG{p}{[}\PYG{l+s+s1}{\PYGZsq{}}\PYG{l+s+s1}{O3}\PYG{l+s+s1}{\PYGZsq{}}\PYG{p}{,} \PYG{l+s+s1}{\PYGZsq{}}\PYG{l+s+s1}{NO}\PYG{l+s+s1}{\PYGZsq{}}\PYG{p}{,} \PYG{l+s+s1}{\PYGZsq{}}\PYG{l+s+s1}{NO2}\PYG{l+s+s1}{\PYGZsq{}}\PYG{p}{]}
\PYG{n}{concs}\PYG{o}{.}\PYG{n}{index}\PYG{o}{.}\PYG{n}{name} \PYG{o}{=} \PYG{l+s+s1}{\PYGZsq{}}\PYG{l+s+s1}{Hours since noon}\PYG{l+s+s1}{\PYGZsq{}}
\PYG{n}{concs}\PYG{o}{.}\PYG{n}{plot}\PYG{p}{(}\PYG{n}{ylim}\PYG{o}{=}\PYG{p}{[}\PYG{l+m+mf}{1.e8}\PYG{p}{,} \PYG{k+kc}{None}\PYG{p}{]}\PYG{p}{,} \PYG{n}{logy}\PYG{o}{=}\PYG{k+kc}{True}\PYG{p}{,} \PYG{n}{y}\PYG{o}{=}\PYG{p}{[}\PYG{l+s+s1}{\PYGZsq{}}\PYG{l+s+s1}{O3}\PYG{l+s+s1}{\PYGZsq{}}\PYG{p}{,} \PYG{l+s+s1}{\PYGZsq{}}\PYG{l+s+s1}{NO}\PYG{l+s+s1}{\PYGZsq{}}\PYG{p}{,} \PYG{l+s+s1}{\PYGZsq{}}\PYG{l+s+s1}{NO2}\PYG{l+s+s1}{\PYGZsq{}}\PYG{p}{]}\PYG{p}{,} \PYG{n}{grid}\PYG{o}{=}\PYG{k+kc}{True}\PYG{p}{)}
\PYG{n}{plt}\PYG{o}{.}\PYG{n}{savefig}\PYG{p}{(}\PYG{l+s+s1}{\PYGZsq{}}\PYG{l+s+s1}{test21\PYGZus{}time.png}\PYG{l+s+s1}{\PYGZsq{}}\PYG{p}{)}
\end{Verbatim}

This code produces the following plot:
\phantomsection\label{improving:test21-time}\begin{figure}[htbp]
\centering

\noindent\scalebox{0.800000}{\sphinxincludegraphics{{test21_time}.png}}
\label{improving:test21-time}\end{figure}

From which we can see that the concentration of ozone stabilized more quickly
in this case. As you can see, we can play around with the initial conditions
as much as we want and analyse that result of model. In fact, we encourage you
to do so. However, let us focus this guide on the next step and modify some more
fundamental aspects of the model: the reaction rates.


\section{Modifying the reactions}
\label{improving:modifying-the-reactions}
Now we will alter the reaction rates of some reactions in the model. Keep in
mind that these alterations do not are not realistic. They are simply done here
for the sake of learning how the model works.

Begin again by creating a \sphinxcode{test3} directory anywhere and going into it (\sphinxcode{mkdir test3
\&\& cd test3}). In this directory, create a file called \sphinxcode{strato3.kpp} with the following
contents:

\begin{Verbatim}[commandchars=\\\{\}]
\PYG{c+c1}{\PYGZsh{}MODEL      strato3}
\PYG{c+c1}{\PYGZsh{}LANGUAGE   Fortran90}
\PYG{c+c1}{\PYGZsh{}INTEGRATOR rosenbrock}
\PYG{c+c1}{\PYGZsh{}DRIVER     general}
\end{Verbatim}

\begin{notice}{note}{Note:}
Note that in this kpp file we are forced (for some reason) to include another line
defining the driver for the \sphinxcode{strato3} model.
\end{notice}

This file tells KPP to look for the \sphinxcode{strato3.def} file in its \sphinxcode{models}
directory. So let us create this file by copying the \sphinxcode{my\_strato.def} file.
You can do that with \sphinxcode{cp \$KPP\_HOME/models/my\_strato.def \$KPP\_HOME/models/strato3.def}.
Open the file (\sphinxcode{notepad++ \$KPP\_HOME/models/strato3.def}) and find the first two lines
which originally read

\begin{Verbatim}[commandchars=\\\{\}]
\PYG{c+c1}{\PYGZsh{}include small\PYGZus{}strato.spc}
\PYG{c+c1}{\PYGZsh{}include small\PYGZus{}strato.eqn}
\end{Verbatim}

Which still tells KPP to look for the original \sphinxcode{small\_strato} model files
when defining the species (\sphinxcode{.spc}) and chemical equations (\sphinxcode{.eqn}). You
should modify these lines to the following:

\begin{Verbatim}[commandchars=\\\{\}]
\PYG{c+c1}{\PYGZsh{}include strato3.spc}
\PYG{c+c1}{\PYGZsh{}include strato3.eqn}
\end{Verbatim}

If you try to run KPP now you'll again get an error because those
files still don't exist. Let's create them by copying the original \sphinxcode{small\_strato}
files, which can the following commands:

\begin{Verbatim}[commandchars=\\\{\}]
cp \PYGZdl{}KPP\PYGZus{}HOME/models/small\PYGZus{}strato.spc \PYGZdl{}KPP\PYGZus{}HOME/models/strato3.spc
cp \PYGZdl{}KPP\PYGZus{}HOME/models/small\PYGZus{}strato.eqn \PYGZdl{}KPP\PYGZus{}HOME/models/strato3.eqn
\end{Verbatim}

If you check the \sphinxcode{strato3.spc} file you'll see that it only the definitions
of the species used, which wouldn't make much sense to change for now, so we
will leave it how it is. Now we focus on the \sphinxcode{strato3.eqn} file. If you
open it you'll find the following lines:

\begin{Verbatim}[commandchars=\\\{\}]
\PYG{c+c1}{\PYGZsh{}EQUATIONS \PYGZob{} Small Stratospheric Mechanism \PYGZcb{}}

\PYG{o}{\PYGZlt{}}\PYG{n}{R1}\PYG{o}{\PYGZgt{}}  \PYG{n}{O2}   \PYG{o}{+} \PYG{n}{hv} \PYG{o}{=} \PYG{l+m+mi}{2}\PYG{n}{O}            \PYG{p}{:} \PYG{p}{(}\PYG{l+m+mf}{2.643E\PYGZhy{}10}\PYG{p}{)} \PYG{o}{*} \PYG{n}{SUN}\PYG{o}{*}\PYG{n}{SUN}\PYG{o}{*}\PYG{n}{SUN}\PYG{p}{;}
\PYG{o}{\PYGZlt{}}\PYG{n}{R2}\PYG{o}{\PYGZgt{}}  \PYG{n}{O}    \PYG{o}{+} \PYG{n}{O2} \PYG{o}{=} \PYG{n}{O3}            \PYG{p}{:} \PYG{p}{(}\PYG{l+m+mf}{8.018E\PYGZhy{}17}\PYG{p}{)}\PYG{p}{;}
\PYG{o}{\PYGZlt{}}\PYG{n}{R3}\PYG{o}{\PYGZgt{}}  \PYG{n}{O3}   \PYG{o}{+} \PYG{n}{hv} \PYG{o}{=} \PYG{n}{O}   \PYG{o}{+} \PYG{n}{O2}      \PYG{p}{:} \PYG{p}{(}\PYG{l+m+mf}{6.120E\PYGZhy{}04}\PYG{p}{)} \PYG{o}{*} \PYG{n}{SUN}\PYG{p}{;}
\PYG{o}{\PYGZlt{}}\PYG{n}{R4}\PYG{o}{\PYGZgt{}}  \PYG{n}{O}    \PYG{o}{+} \PYG{n}{O3} \PYG{o}{=} \PYG{l+m+mi}{2}\PYG{n}{O2}           \PYG{p}{:} \PYG{p}{(}\PYG{l+m+mf}{1.576E\PYGZhy{}15}\PYG{p}{)}\PYG{p}{;}
\PYG{o}{\PYGZlt{}}\PYG{n}{R5}\PYG{o}{\PYGZgt{}}  \PYG{n}{O3}   \PYG{o}{+} \PYG{n}{hv} \PYG{o}{=} \PYG{n}{O1D} \PYG{o}{+} \PYG{n}{O2}      \PYG{p}{:} \PYG{p}{(}\PYG{l+m+mf}{1.070E\PYGZhy{}03}\PYG{p}{)} \PYG{o}{*} \PYG{n}{SUN}\PYG{o}{*}\PYG{n}{SUN}\PYG{p}{;}
\PYG{o}{\PYGZlt{}}\PYG{n}{R6}\PYG{o}{\PYGZgt{}}  \PYG{n}{O1D}  \PYG{o}{+} \PYG{n}{M}  \PYG{o}{=} \PYG{n}{O}   \PYG{o}{+} \PYG{n}{M}       \PYG{p}{:} \PYG{p}{(}\PYG{l+m+mf}{7.110E\PYGZhy{}11}\PYG{p}{)}\PYG{p}{;}
\PYG{o}{\PYGZlt{}}\PYG{n}{R7}\PYG{o}{\PYGZgt{}}  \PYG{n}{O1D}  \PYG{o}{+} \PYG{n}{O3} \PYG{o}{=} \PYG{l+m+mi}{2}\PYG{n}{O2}           \PYG{p}{:} \PYG{p}{(}\PYG{l+m+mf}{1.200E\PYGZhy{}10}\PYG{p}{)}\PYG{p}{;}
\PYG{o}{\PYGZlt{}}\PYG{n}{R8}\PYG{o}{\PYGZgt{}}  \PYG{n}{NO}   \PYG{o}{+} \PYG{n}{O3} \PYG{o}{=} \PYG{n}{NO2} \PYG{o}{+} \PYG{n}{O2}      \PYG{p}{:} \PYG{p}{(}\PYG{l+m+mf}{6.062E\PYGZhy{}15}\PYG{p}{)}\PYG{p}{;}
\PYG{o}{\PYGZlt{}}\PYG{n}{R9}\PYG{o}{\PYGZgt{}}  \PYG{n}{NO2}  \PYG{o}{+} \PYG{n}{O}  \PYG{o}{=} \PYG{n}{NO}  \PYG{o}{+} \PYG{n}{O2}      \PYG{p}{:} \PYG{p}{(}\PYG{l+m+mf}{1.069E\PYGZhy{}11}\PYG{p}{)}\PYG{p}{;}
\PYG{o}{\PYGZlt{}}\PYG{n}{R10}\PYG{o}{\PYGZgt{}} \PYG{n}{NO2}  \PYG{o}{+} \PYG{n}{hv} \PYG{o}{=} \PYG{n}{NO}  \PYG{o}{+} \PYG{n}{O}       \PYG{p}{:} \PYG{p}{(}\PYG{l+m+mf}{1.289E\PYGZhy{}02}\PYG{p}{)} \PYG{o}{*} \PYG{n}{SUN}\PYG{p}{;}
\end{Verbatim}

Just for the sake of learning, let us change the photolysis rate (last
reaction) to make it a lot slower. We will make the last line read:

\begin{Verbatim}[commandchars=\\\{\}]
\PYG{o}{\PYGZlt{}}\PYG{n}{R10}\PYG{o}{\PYGZgt{}} \PYG{n}{NO2}  \PYG{o}{+} \PYG{n}{hv} \PYG{o}{=} \PYG{n}{NO}  \PYG{o}{+} \PYG{n}{O}       \PYG{p}{:} \PYG{p}{(}\PYG{l+m+mf}{1.289E\PYGZhy{}06}\PYG{p}{)} \PYG{o}{*} \PYG{n}{SUN}\PYG{p}{;}
\end{Verbatim}

..note

\begin{Verbatim}[commandchars=\\\{\}]
This is 4 orders of magnitude slower than it previously was and is not realistic!
We only made this change for the sake of illustration, so that the output change
is easier to see.
\end{Verbatim}

Now we go through the same steps of running \sphinxcode{kpp strato3.kpp}, changing the
compiler to gfortran and running \sphinxcode{make -f Makefile\_strato3}. If everything
goes well, we'll see the \sphinxcode{strato3.exe} created. After running
\sphinxcode{./strato3.exe} sure enough \sphinxcode{strato3.dat} is created, which we can plot
with the same python code from the last example (only changing the name of the
file of course):
\phantomsection\label{improving:test3-time}\begin{figure}[htbp]
\centering

\noindent\scalebox{0.800000}{\sphinxincludegraphics{{test3_time}.png}}
\label{improving:test3-time}\end{figure}

We can see that once again the final result changed. This time, since NO2 is
photolizing a lot slower, we see less NO in comparison with the previous plot.

Now that we have modified the \sphinxcode{small\_strato} example in (almost) every way
possible, let us create a new model from scratch.


\section{Creating a model from scratch}
\label{improving:creating-a-model-from-scratch}
Now we do our last step and create a completely new model with our own
reactions.  Basically for our new model to be complete we should give it the
initial conditions, numerical constraints, species and reactions list. Let us
start with the latter: the reactions.

We will try to simulate a very small tropospheric model, which we will call
\sphinxcode{ttropo} (meaning tiny tropo; let's write it like that just because it's
easier). First we create an equations (reactions) file in KPP's \sphinxcode{models}
directory. We do that with \sphinxcode{notepad++ \$KPP\_HOME/models/ttropo.eqn} (or
\sphinxcode{gedit \$KPP\_HOME/models/ttropo.eqn} or whatever editor you choose). Now in
that file we will put our reactions following the syntax that we saw in the
previous example. The reactions we choose are:

\begin{Verbatim}[commandchars=\\\{\}]
\PYG{c+c1}{\PYGZsh{}EQUATIONS \PYGZob{} Tiny Tropospheric Mechanism \PYGZcb{}}

\PYG{o}{\PYGZlt{}}\PYG{n}{R1}\PYG{o}{\PYGZgt{}}  \PYG{n}{NO2}  \PYG{o}{+} \PYG{n}{hv}  \PYG{o}{=} \PYG{n}{NO}  \PYG{o}{+} \PYG{n}{O}       \PYG{p}{:} \PYG{p}{(}\PYG{l+m+mf}{8.3E\PYGZhy{}03}\PYG{p}{)} \PYG{o}{*} \PYG{n}{SUN}\PYG{p}{;}
\PYG{o}{\PYGZlt{}}\PYG{n}{R2}\PYG{o}{\PYGZgt{}}  \PYG{n}{O}    \PYG{o}{+} \PYG{n}{O2}  \PYG{o}{=} \PYG{n}{O3}            \PYG{p}{:} \PYG{p}{(}\PYG{l+m+mf}{8.018E\PYGZhy{}17}\PYG{p}{)}\PYG{p}{;}
\PYG{o}{\PYGZlt{}}\PYG{n}{R3}\PYG{o}{\PYGZgt{}}  \PYG{n}{NO}   \PYG{o}{+} \PYG{n}{O3}  \PYG{o}{=} \PYG{n}{NO2} \PYG{o}{+} \PYG{n}{O2}      \PYG{p}{:} \PYG{p}{(}\PYG{l+m+mf}{6.062E\PYGZhy{}15}\PYG{p}{)}\PYG{p}{;}
\PYG{o}{\PYGZlt{}}\PYG{n}{R41}\PYG{o}{\PYGZgt{}} \PYG{n}{O3}   \PYG{o}{+} \PYG{n}{hv}  \PYG{o}{=} \PYG{n}{O}   \PYG{o}{+} \PYG{n}{O2}      \PYG{p}{:} \PYG{p}{(}\PYG{l+m+mf}{6.120E\PYGZhy{}04}\PYG{p}{)} \PYG{o}{*} \PYG{n}{SUN}\PYG{p}{;}
\PYG{o}{\PYGZlt{}}\PYG{n}{R42}\PYG{o}{\PYGZgt{}} \PYG{n}{O3}   \PYG{o}{+} \PYG{n}{hv}  \PYG{o}{=} \PYG{n}{O1D} \PYG{o}{+} \PYG{n}{O2}      \PYG{p}{:} \PYG{p}{(}\PYG{l+m+mf}{1.070E\PYGZhy{}03}\PYG{p}{)} \PYG{o}{*} \PYG{n}{SUN}\PYG{o}{*}\PYG{n}{SUN}\PYG{p}{;}
\PYG{o}{\PYGZlt{}}\PYG{n}{R5}\PYG{o}{\PYGZgt{}}  \PYG{n}{O1D}  \PYG{o}{+} \PYG{n}{M}   \PYG{o}{=} \PYG{n}{O}   \PYG{o}{+} \PYG{n}{M}       \PYG{p}{:} \PYG{p}{(}\PYG{l+m+mf}{7.110E\PYGZhy{}11}\PYG{p}{)}\PYG{p}{;}
\PYG{o}{\PYGZlt{}}\PYG{n}{R6}\PYG{o}{\PYGZgt{}}  \PYG{n}{O1D} \PYG{o}{+} \PYG{n}{H2O}  \PYG{o}{=} \PYG{l+m+mi}{2}\PYG{n}{OH}           \PYG{p}{:} \PYG{p}{(}\PYG{l+m+mf}{2.2E\PYGZhy{}10}\PYG{p}{)}\PYG{p}{;}
\PYG{o}{\PYGZlt{}}\PYG{n}{R7}\PYG{o}{\PYGZgt{}}  \PYG{n}{CO}\PYG{o}{+} \PYG{n}{OH}\PYG{o}{+} \PYG{n}{M}  \PYG{o}{=} \PYG{n}{CO2} \PYG{o}{+} \PYG{n}{H} \PYG{o}{+} \PYG{n}{M}   \PYG{p}{:} \PYG{p}{(}\PYG{l+m+mf}{2.2E\PYGZhy{}13}\PYG{p}{)}\PYG{p}{;}
\PYG{o}{\PYGZlt{}}\PYG{n}{R8}\PYG{o}{\PYGZgt{}}  \PYG{n}{H} \PYG{o}{+} \PYG{n}{O2} \PYG{o}{+} \PYG{n}{M} \PYG{o}{=} \PYG{n}{HO2} \PYG{o}{+} \PYG{n}{M}       \PYG{p}{:} \PYG{p}{(}\PYG{p}{)}\PYG{p}{;}
\PYG{o}{\PYGZlt{}}\PYG{n}{R9}\PYG{o}{\PYGZgt{}}  \PYG{n}{HO2}  \PYG{o}{+} \PYG{n}{NO}  \PYG{o}{=} \PYG{n}{OH}  \PYG{o}{+} \PYG{n}{NO2}     \PYG{p}{:} \PYG{p}{(}\PYG{l+m+mf}{8.3E\PYGZhy{}12}\PYG{p}{)}\PYG{p}{;}
\PYG{o}{\PYGZlt{}}\PYG{n}{R10}\PYG{o}{\PYGZgt{}} \PYG{n}{OH}  \PYG{o}{+} \PYG{n}{NO2}  \PYG{o}{=} \PYG{n}{HNO3}          \PYG{p}{:} \PYG{p}{(}\PYG{l+m+mf}{1.1E\PYGZhy{}11}\PYG{p}{)}\PYG{p}{;}
\PYG{o}{\PYGZlt{}}\PYG{n}{R11}\PYG{o}{\PYGZgt{}} \PYG{n}{HO2} \PYG{o}{+} \PYG{n}{HO2}  \PYG{o}{=} \PYG{n}{H2O2}          \PYG{p}{:} \PYG{p}{(}\PYG{l+m+mf}{5.6E\PYGZhy{}12}\PYG{p}{)}\PYG{p}{;}
\PYG{o}{\PYGZlt{}}\PYG{n}{R12}\PYG{o}{\PYGZgt{}} \PYG{n}{O3}  \PYG{o}{+} \PYG{n}{HO2}  \PYG{o}{=} \PYG{n}{OH} \PYG{o}{+} \PYG{l+m+mi}{2}\PYG{n}{O2}      \PYG{p}{:} \PYG{p}{(}\PYG{l+m+mf}{2.0E\PYGZhy{}15}\PYG{p}{)}\PYG{p}{;}
\PYG{o}{\PYGZlt{}}\PYG{n}{R13}\PYG{o}{\PYGZgt{}} \PYG{n}{H2O2} \PYG{o}{+} \PYG{n}{hv}  \PYG{o}{=} \PYG{l+m+mi}{2}\PYG{n}{OH}           \PYG{p}{:} \PYG{p}{(}\PYG{l+m+mf}{1.366E\PYGZhy{}5}\PYG{p}{)} \PYG{o}{*} \PYG{n}{SUN}\PYG{p}{;}
\PYG{o}{\PYGZlt{}}\PYG{n}{R14}\PYG{o}{\PYGZgt{}} \PYG{n}{H2O2}       \PYG{o}{=} \PYG{n}{H2O2aq}        \PYG{p}{:} \PYG{p}{(}\PYG{l+m+mf}{3.3000e\PYGZhy{}01}\PYG{p}{)}\PYG{p}{;}
\PYG{o}{\PYGZlt{}}\PYG{n}{R15}\PYG{o}{\PYGZgt{}} \PYG{n}{HNO3}       \PYG{o}{=} \PYG{n}{HNO3aq}        \PYG{p}{:} \PYG{p}{(}\PYG{l+m+mf}{2.4000e\PYGZhy{}01}\PYG{p}{)}\PYG{p}{;}
\end{Verbatim}

Now we create the species file which bla bla bla

After these are create, we create the \sphinxcode{.def} with \sphinxcode{notepad++ \$KPP\_HOME/models/ttropo.def}.
In that file you will write the followinf lines:

\begin{Verbatim}[commandchars=\\\{\}]
\PYG{c+c1}{\PYGZsh{}include ttropo.spc}
\PYG{c+c1}{\PYGZsh{}include ttropo.eqn}

\PYG{c+c1}{\PYGZsh{}JACOBIAN SPARSE\PYGZus{}LU\PYGZus{}ROW      \PYGZob{}Use Sparse DATA STRUCTURES\PYGZcb{}}
\PYG{c+c1}{\PYGZsh{}DRIVER general}

\PYG{c+c1}{\PYGZsh{}LOOKAT O3; NO; NO2;           \PYGZob{}File Output\PYGZcb{}}
\PYG{c+c1}{\PYGZsh{}MONITOR O3;N;O2;O;NO;O1D;NO2; \PYGZob{}Screen Output\PYGZcb{}}

\PYG{c+c1}{\PYGZsh{}CHECK O; N;                   \PYGZob{}Check Mass Balance\PYGZcb{}}

\PYG{c+c1}{\PYGZsh{}INITVALUES                    \PYGZob{}Initial Values\PYGZcb{}}

\PYG{n}{CFACTOR} \PYG{o}{=} \PYG{l+m+mf}{1.}    \PYG{p}{;}              \PYG{p}{\PYGZob{}}\PYG{n}{Conversion} \PYG{n}{Factor}\PYG{p}{\PYGZcb{}}
\PYG{n}{O1D} \PYG{o}{=} \PYG{l+m+mf}{9.906E+01} \PYG{p}{;}
\PYG{n}{O}   \PYG{o}{=} \PYG{l+m+mf}{6.624E+08} \PYG{p}{;}
\PYG{n}{O3}  \PYG{o}{=} \PYG{l+m+mf}{5.00E+10} \PYG{p}{;}
\PYG{n}{O2}  \PYG{o}{=} \PYG{l+m+mf}{1.697E+19} \PYG{p}{;}
\PYG{n}{NO}  \PYG{o}{=} \PYG{l+m+mf}{9.00E+09} \PYG{p}{;}
\PYG{n}{NO2} \PYG{o}{=} \PYG{l+m+mf}{2.240E+08} \PYG{p}{;}
\PYG{n}{M}   \PYG{o}{=} \PYG{l+m+mf}{8.120E+16} \PYG{p}{;}
\PYG{n}{H2O2}   \PYG{o}{=}\PYG{p}{;}
\PYG{n}{H2O2aq} \PYG{o}{=}\PYG{p}{;}
\PYG{n}{HNO3}   \PYG{o}{=}\PYG{p}{;}
\PYG{n}{HNO3aq} \PYG{o}{=}\PYG{p}{;}
\PYG{n}{HO2}    \PYG{o}{=}\PYG{p}{;}
\PYG{n}{OH}     \PYG{o}{=}\PYG{p}{;}
\PYG{n}{CO}     \PYG{o}{=} \PYG{p}{;}
\PYG{n}{H2O}    \PYG{o}{=} \PYG{l+m+mf}{3.9E17} \PYG{p}{;}
\PYG{n}{CO2}    \PYG{o}{=} \PYG{p}{;}


\PYG{c+c1}{\PYGZsh{}INLINE F90\PYGZus{}INIT}
       \PYG{n}{TSTART} \PYG{o}{=} \PYG{p}{(}\PYG{l+m+mi}{12}\PYG{o}{*}\PYG{l+m+mi}{3600}\PYG{p}{)}
       \PYG{n}{TEND} \PYG{o}{=} \PYG{n}{TSTART} \PYG{o}{+} \PYG{p}{(}\PYG{l+m+mi}{30}\PYG{o}{*}\PYG{l+m+mi}{24}\PYG{o}{*}\PYG{l+m+mi}{3600}\PYG{p}{)}
       \PYG{n}{DT} \PYG{o}{=} \PYG{l+m+mf}{0.25}\PYG{o}{*}\PYG{l+m+mi}{3600}
       \PYG{n}{TEMP} \PYG{o}{=} \PYG{l+m+mi}{270}
\PYG{c+c1}{\PYGZsh{}ENDINLINE}
\end{Verbatim}


\chapter{Indices and tables}
\label{index:indices-and-tables}\begin{itemize}
\item {} 
\DUrole{xref,std,std-ref}{genindex}

\item {} 
\DUrole{xref,std,std-ref}{modindex}

\item {} 
\DUrole{xref,std,std-ref}{search}

\end{itemize}



\renewcommand{\indexname}{Index}
\printindex
\end{document}
