% Generated by Sphinx.
\def\sphinxdocclass{report}
\newif\ifsphinxKeepOldNames \sphinxKeepOldNamestrue
\documentclass[letterpaper,10pt,openany,oneside]{sphinxmanual}
\usepackage{iftex}

\ifPDFTeX
  \usepackage[utf8]{inputenc}
\fi
\ifdefined\DeclareUnicodeCharacter
  \DeclareUnicodeCharacter{00A0}{\nobreakspace}
\fi
\usepackage{cmap}
\usepackage[T1]{fontenc}
\usepackage{amsmath,amssymb,amstext}
\usepackage[english]{babel}
\usepackage{times}
\usepackage[Bjarne]{fncychap}
\usepackage{longtable}
\usepackage{sphinx}
\usepackage{multirow}
\usepackage{eqparbox}


\addto\captionsenglish{\renewcommand{\figurename}{Fig.\@ }}
\addto\captionsenglish{\renewcommand{\tablename}{Table }}
\SetupFloatingEnvironment{literal-block}{name=Listing }

\addto\extrasenglish{\def\pageautorefname{page}}

\setcounter{tocdepth}{1}


\title{An easy guide for KPP}
\date{Dec 07, 2016}
\release{1.0}
\author{Tomas Chor}
\newcommand{\sphinxlogo}{\sphinxincludegraphics{wheat_field.jpg}\par}
\renewcommand{\releasename}{Release}
\makeindex

\makeatletter
\def\PYG@reset{\let\PYG@it=\relax \let\PYG@bf=\relax%
    \let\PYG@ul=\relax \let\PYG@tc=\relax%
    \let\PYG@bc=\relax \let\PYG@ff=\relax}
\def\PYG@tok#1{\csname PYG@tok@#1\endcsname}
\def\PYG@toks#1+{\ifx\relax#1\empty\else%
    \PYG@tok{#1}\expandafter\PYG@toks\fi}
\def\PYG@do#1{\PYG@bc{\PYG@tc{\PYG@ul{%
    \PYG@it{\PYG@bf{\PYG@ff{#1}}}}}}}
\def\PYG#1#2{\PYG@reset\PYG@toks#1+\relax+\PYG@do{#2}}

\expandafter\def\csname PYG@tok@gd\endcsname{\def\PYG@tc##1{\textcolor[rgb]{0.63,0.00,0.00}{##1}}}
\expandafter\def\csname PYG@tok@gu\endcsname{\let\PYG@bf=\textbf\def\PYG@tc##1{\textcolor[rgb]{0.50,0.00,0.50}{##1}}}
\expandafter\def\csname PYG@tok@gt\endcsname{\def\PYG@tc##1{\textcolor[rgb]{0.00,0.27,0.87}{##1}}}
\expandafter\def\csname PYG@tok@gs\endcsname{\let\PYG@bf=\textbf}
\expandafter\def\csname PYG@tok@gr\endcsname{\def\PYG@tc##1{\textcolor[rgb]{1.00,0.00,0.00}{##1}}}
\expandafter\def\csname PYG@tok@cm\endcsname{\let\PYG@it=\textit\def\PYG@tc##1{\textcolor[rgb]{0.25,0.50,0.56}{##1}}}
\expandafter\def\csname PYG@tok@vg\endcsname{\def\PYG@tc##1{\textcolor[rgb]{0.73,0.38,0.84}{##1}}}
\expandafter\def\csname PYG@tok@vi\endcsname{\def\PYG@tc##1{\textcolor[rgb]{0.73,0.38,0.84}{##1}}}
\expandafter\def\csname PYG@tok@mh\endcsname{\def\PYG@tc##1{\textcolor[rgb]{0.13,0.50,0.31}{##1}}}
\expandafter\def\csname PYG@tok@cs\endcsname{\def\PYG@tc##1{\textcolor[rgb]{0.25,0.50,0.56}{##1}}\def\PYG@bc##1{\setlength{\fboxsep}{0pt}\colorbox[rgb]{1.00,0.94,0.94}{\strut ##1}}}
\expandafter\def\csname PYG@tok@ge\endcsname{\let\PYG@it=\textit}
\expandafter\def\csname PYG@tok@vc\endcsname{\def\PYG@tc##1{\textcolor[rgb]{0.73,0.38,0.84}{##1}}}
\expandafter\def\csname PYG@tok@il\endcsname{\def\PYG@tc##1{\textcolor[rgb]{0.13,0.50,0.31}{##1}}}
\expandafter\def\csname PYG@tok@go\endcsname{\def\PYG@tc##1{\textcolor[rgb]{0.20,0.20,0.20}{##1}}}
\expandafter\def\csname PYG@tok@cp\endcsname{\def\PYG@tc##1{\textcolor[rgb]{0.00,0.44,0.13}{##1}}}
\expandafter\def\csname PYG@tok@gi\endcsname{\def\PYG@tc##1{\textcolor[rgb]{0.00,0.63,0.00}{##1}}}
\expandafter\def\csname PYG@tok@gh\endcsname{\let\PYG@bf=\textbf\def\PYG@tc##1{\textcolor[rgb]{0.00,0.00,0.50}{##1}}}
\expandafter\def\csname PYG@tok@ni\endcsname{\let\PYG@bf=\textbf\def\PYG@tc##1{\textcolor[rgb]{0.84,0.33,0.22}{##1}}}
\expandafter\def\csname PYG@tok@nl\endcsname{\let\PYG@bf=\textbf\def\PYG@tc##1{\textcolor[rgb]{0.00,0.13,0.44}{##1}}}
\expandafter\def\csname PYG@tok@nn\endcsname{\let\PYG@bf=\textbf\def\PYG@tc##1{\textcolor[rgb]{0.05,0.52,0.71}{##1}}}
\expandafter\def\csname PYG@tok@no\endcsname{\def\PYG@tc##1{\textcolor[rgb]{0.38,0.68,0.84}{##1}}}
\expandafter\def\csname PYG@tok@na\endcsname{\def\PYG@tc##1{\textcolor[rgb]{0.25,0.44,0.63}{##1}}}
\expandafter\def\csname PYG@tok@nb\endcsname{\def\PYG@tc##1{\textcolor[rgb]{0.00,0.44,0.13}{##1}}}
\expandafter\def\csname PYG@tok@nc\endcsname{\let\PYG@bf=\textbf\def\PYG@tc##1{\textcolor[rgb]{0.05,0.52,0.71}{##1}}}
\expandafter\def\csname PYG@tok@nd\endcsname{\let\PYG@bf=\textbf\def\PYG@tc##1{\textcolor[rgb]{0.33,0.33,0.33}{##1}}}
\expandafter\def\csname PYG@tok@ne\endcsname{\def\PYG@tc##1{\textcolor[rgb]{0.00,0.44,0.13}{##1}}}
\expandafter\def\csname PYG@tok@nf\endcsname{\def\PYG@tc##1{\textcolor[rgb]{0.02,0.16,0.49}{##1}}}
\expandafter\def\csname PYG@tok@si\endcsname{\let\PYG@it=\textit\def\PYG@tc##1{\textcolor[rgb]{0.44,0.63,0.82}{##1}}}
\expandafter\def\csname PYG@tok@s2\endcsname{\def\PYG@tc##1{\textcolor[rgb]{0.25,0.44,0.63}{##1}}}
\expandafter\def\csname PYG@tok@nt\endcsname{\let\PYG@bf=\textbf\def\PYG@tc##1{\textcolor[rgb]{0.02,0.16,0.45}{##1}}}
\expandafter\def\csname PYG@tok@nv\endcsname{\def\PYG@tc##1{\textcolor[rgb]{0.73,0.38,0.84}{##1}}}
\expandafter\def\csname PYG@tok@s1\endcsname{\def\PYG@tc##1{\textcolor[rgb]{0.25,0.44,0.63}{##1}}}
\expandafter\def\csname PYG@tok@ch\endcsname{\let\PYG@it=\textit\def\PYG@tc##1{\textcolor[rgb]{0.25,0.50,0.56}{##1}}}
\expandafter\def\csname PYG@tok@m\endcsname{\def\PYG@tc##1{\textcolor[rgb]{0.13,0.50,0.31}{##1}}}
\expandafter\def\csname PYG@tok@gp\endcsname{\let\PYG@bf=\textbf\def\PYG@tc##1{\textcolor[rgb]{0.78,0.36,0.04}{##1}}}
\expandafter\def\csname PYG@tok@sh\endcsname{\def\PYG@tc##1{\textcolor[rgb]{0.25,0.44,0.63}{##1}}}
\expandafter\def\csname PYG@tok@ow\endcsname{\let\PYG@bf=\textbf\def\PYG@tc##1{\textcolor[rgb]{0.00,0.44,0.13}{##1}}}
\expandafter\def\csname PYG@tok@sx\endcsname{\def\PYG@tc##1{\textcolor[rgb]{0.78,0.36,0.04}{##1}}}
\expandafter\def\csname PYG@tok@bp\endcsname{\def\PYG@tc##1{\textcolor[rgb]{0.00,0.44,0.13}{##1}}}
\expandafter\def\csname PYG@tok@c1\endcsname{\let\PYG@it=\textit\def\PYG@tc##1{\textcolor[rgb]{0.25,0.50,0.56}{##1}}}
\expandafter\def\csname PYG@tok@o\endcsname{\def\PYG@tc##1{\textcolor[rgb]{0.40,0.40,0.40}{##1}}}
\expandafter\def\csname PYG@tok@kc\endcsname{\let\PYG@bf=\textbf\def\PYG@tc##1{\textcolor[rgb]{0.00,0.44,0.13}{##1}}}
\expandafter\def\csname PYG@tok@c\endcsname{\let\PYG@it=\textit\def\PYG@tc##1{\textcolor[rgb]{0.25,0.50,0.56}{##1}}}
\expandafter\def\csname PYG@tok@mf\endcsname{\def\PYG@tc##1{\textcolor[rgb]{0.13,0.50,0.31}{##1}}}
\expandafter\def\csname PYG@tok@err\endcsname{\def\PYG@bc##1{\setlength{\fboxsep}{0pt}\fcolorbox[rgb]{1.00,0.00,0.00}{1,1,1}{\strut ##1}}}
\expandafter\def\csname PYG@tok@mb\endcsname{\def\PYG@tc##1{\textcolor[rgb]{0.13,0.50,0.31}{##1}}}
\expandafter\def\csname PYG@tok@ss\endcsname{\def\PYG@tc##1{\textcolor[rgb]{0.32,0.47,0.09}{##1}}}
\expandafter\def\csname PYG@tok@sr\endcsname{\def\PYG@tc##1{\textcolor[rgb]{0.14,0.33,0.53}{##1}}}
\expandafter\def\csname PYG@tok@mo\endcsname{\def\PYG@tc##1{\textcolor[rgb]{0.13,0.50,0.31}{##1}}}
\expandafter\def\csname PYG@tok@kd\endcsname{\let\PYG@bf=\textbf\def\PYG@tc##1{\textcolor[rgb]{0.00,0.44,0.13}{##1}}}
\expandafter\def\csname PYG@tok@mi\endcsname{\def\PYG@tc##1{\textcolor[rgb]{0.13,0.50,0.31}{##1}}}
\expandafter\def\csname PYG@tok@kn\endcsname{\let\PYG@bf=\textbf\def\PYG@tc##1{\textcolor[rgb]{0.00,0.44,0.13}{##1}}}
\expandafter\def\csname PYG@tok@cpf\endcsname{\let\PYG@it=\textit\def\PYG@tc##1{\textcolor[rgb]{0.25,0.50,0.56}{##1}}}
\expandafter\def\csname PYG@tok@kr\endcsname{\let\PYG@bf=\textbf\def\PYG@tc##1{\textcolor[rgb]{0.00,0.44,0.13}{##1}}}
\expandafter\def\csname PYG@tok@s\endcsname{\def\PYG@tc##1{\textcolor[rgb]{0.25,0.44,0.63}{##1}}}
\expandafter\def\csname PYG@tok@kp\endcsname{\def\PYG@tc##1{\textcolor[rgb]{0.00,0.44,0.13}{##1}}}
\expandafter\def\csname PYG@tok@w\endcsname{\def\PYG@tc##1{\textcolor[rgb]{0.73,0.73,0.73}{##1}}}
\expandafter\def\csname PYG@tok@kt\endcsname{\def\PYG@tc##1{\textcolor[rgb]{0.56,0.13,0.00}{##1}}}
\expandafter\def\csname PYG@tok@sc\endcsname{\def\PYG@tc##1{\textcolor[rgb]{0.25,0.44,0.63}{##1}}}
\expandafter\def\csname PYG@tok@sb\endcsname{\def\PYG@tc##1{\textcolor[rgb]{0.25,0.44,0.63}{##1}}}
\expandafter\def\csname PYG@tok@k\endcsname{\let\PYG@bf=\textbf\def\PYG@tc##1{\textcolor[rgb]{0.00,0.44,0.13}{##1}}}
\expandafter\def\csname PYG@tok@se\endcsname{\let\PYG@bf=\textbf\def\PYG@tc##1{\textcolor[rgb]{0.25,0.44,0.63}{##1}}}
\expandafter\def\csname PYG@tok@sd\endcsname{\let\PYG@it=\textit\def\PYG@tc##1{\textcolor[rgb]{0.25,0.44,0.63}{##1}}}

\def\PYGZbs{\char`\\}
\def\PYGZus{\char`\_}
\def\PYGZob{\char`\{}
\def\PYGZcb{\char`\}}
\def\PYGZca{\char`\^}
\def\PYGZam{\char`\&}
\def\PYGZlt{\char`\<}
\def\PYGZgt{\char`\>}
\def\PYGZsh{\char`\#}
\def\PYGZpc{\char`\%}
\def\PYGZdl{\char`\$}
\def\PYGZhy{\char`\-}
\def\PYGZsq{\char`\'}
\def\PYGZdq{\char`\"}
\def\PYGZti{\char`\~}
% for compatibility with earlier versions
\def\PYGZat{@}
\def\PYGZlb{[}
\def\PYGZrb{]}
\makeatother

\renewcommand\PYGZsq{\textquotesingle}

\begin{document}

\maketitle
\tableofcontents
\phantomsection\label{index::doc}



\chapter{Introduction}
\label{README:introduction}\label{README:easy-guide-to-compiling-and-running-kpp}\label{README::doc}
This is an unofficial guide aimed at providing additional information not
covered by the official KPP manual. Specifically, we aim to make things more
comprehensive for the user with few or no experience running software via
command lines. We focus on the latest version of the software, release 2.2.3,
which can be freely downloaded at its \href{http://people.cs.vt.edu/~asandu/Software/Kpp/}{official webpage}. We recommend any person
reading this guide to keep a copy of the original manual since this guide is
not meant to replace it, but as a supplement to it.

In this guide our aim is to teach an inexperienced user to download, compile,
and run KPP. Furthermore, besides teaching how to run the example case in the
manual, we also instruct the user on how to modify any already-existent model
and to create new models.

The source for the documentation can be found at its \href{https://github.com/tomchor/ezkpp}{github page} and the online html version can be
accessed \href{https://tomchor.github.io/ezkpp/}{here}. You can also download this
guide in pdf \href{https://github.com/tomchor/ezkpp/raw/gh-pages/ezkpp.pdf}{here}.
All of the codes and scripts created here as examples are available in the
Github repository.

The directions in this guide have been tested in as broad a range of
operational systems as was possible, but some errors are likely to arise when
applying them to other systems. If that is the case, feedback is encouraged,
either by email, in person, or by creating a Github issue in our page.

This guide was primarily typed and uploaded by Tomas Chor, but has had
substantial help from Prof. Suzanne Paulson and Dr. Paul Griffiths,
particularly for chemistry-related issues.


\section{Contributing}
\label{README:contributing}
You can also contribute to this guide yourself. If you find the need to
correct, improve or add something, feel free to download/fork the project on
Github and modify it. We appreciate if the projects could then be merged back
after that (preferably with an updated version tag), but that is entirely up
to you. To contribute, you have to install and use \href{http://sphinx-doc.org/}{Sphinx}, which is a very handy and easy-to-use tool designed
to build multi-platform documentation effectively.

If you are not familiar with Sphinxs, you should probably read a tutorial (it's
very easy!), but a quick way to start is to download the whole thing from
Github, add some text to any file with a \sphinxcode{.rst} extension and run \sphinxcode{make
html}. That will create a webpage (like this one) with your modifications that
you can open in your browser (open the files with a \sphinxcode{.html} extension that
were created). So you can just go from there and infer the syntax from what's
already written. You can actually learn Sphinx by yourself by doing this and
might not even need to read a tutorial.


\chapter{About bash}
\label{bash:about-bash}\label{bash::doc}
The official KPP manual is entirely based on Unix Shell, which are command
languages that Linux distributions use to interact with the system without a
Graphical User Interface. The manual, however, assumes a non-trivial knowledge
of this tool, which makes it difficult for users that are not experienced with
terminals and command line interfaces (which includes bash, C shell, MS-DOS,
PowerShell, ksh etc.) to install and run the simulations effectively. The
approach adopted in this guide will be to go through the steps necessary to
compile and run KPP, as stated in the manual, but taking the time to explain
them a little better how to do them, and what exactly it is that they do.

We will first go over a few basic notions necessary to understand what is going
to be done in the guide. If you are familiar with the concepts of system shells,
you may skip the next sections.


\section{What is Bash?}
\label{bash:what-is-bash}
First things first: what is a shell? A system shell is the name that computer
engineers use to refer to the outer layer of an Operational System (OS). It is
said outer layer because it separates the user (you) from the core of your OS.
So it separates you from the intricate group of codes that ultimately governs
your machine and lets you interact with your computer using a human-readable
language (and not, for example, binary!). Basically, a shell is a bridge
between you and your machine.

These shells can be either graphical shells \footnote[1]{\sphinxAtStartFootnote%
A note here is that the GUI isn't generally considered a shell, but
that is technically correct given the definition of a shell.
} (called Graphical User Interface,
GUI, just like what you use during mundane tasks such as browsing the web and
reading a PDF document) or text shells (also called terminals or Command Line
Interface, CLI). Graphical shells are easier and extremely intuitive (most
people use the mouse in a GUI and never needed to be told how to do it), but
they are very limited. Basically all you can do is click on buttons that were
previously programmed to to some task and input text.

Texts shells (terminals), however, are extremely powerful. You can do virtually
anything with your computer using them. That comes to the cost of terminals not
being intuitive at all. Since KPP is a complicated code for which there is no
graphical interface, we need to use a terminal to compile (``install'') and run
it, simply because this task requires a more powerful tool then your mouse.

Bash (acronym for Bourne Again Shell) is a kind of Unix Shell used by most of
the Linux systems and some Mac OSs. Some other shells can be used to perform
the same tasks (the KPP manual itself also gives some commands in C Shell,
which is another Unix Shell), but we focus on Bash here because it is the most
common and most easily accessible. Besides its popularity among Linux
distributions and Mac OSs, it is the only shell (as far as my knowledge goes by
the time of writing) that can be natively installed into Windows, as we will
explain in the next section.


\section{Accessing Bash}
\label{bash:accessing-bash}
To access and use Bash, you either need a Bash emulator or to be in an
operational system that supports it natively. Various emulators exist (Cygwin,
cmder, MinGW, etc.) but they are not recommended because some of them contain
many bugs. If you would like to try those anyway, chances are that it'll work,
since we're going to be doing simple tasks and they tend to work well for that.
However, running it natively is always a guarantee of no bugs, so (in the
spirit of keeping it general) we adopt this option throughout this guide, only
occasionally giving some remarks on other shells.

We will briefly go through your options for each of the 3 most common
operational systems.


\subsection{From Windows}
\label{bash:from-windows}
Windows doesn't support Unix Shells natively by default, so here are the
options.

If you're using Windows 10, you can natively install the Ubuntu 14.04 inside
your Windows machine with the Windows 10 anniversary update, which is available
for every \textbf{up-to-date Windows 10} computer. Directions to do this are very
simple and are given in many places (such as \href{http://www.howtogeek.com/249966/how-to-install-and-use-the-linux-bash-shell-on-windows-10/}{here})
so for now we will not explain them in detail. Keep in mind that since this
will give you Bash running natively (even though you're on Windows), you'll be
able to use all the commands that can be used on Linux, such as package
installation commands (\sphinxcode{apt install}).

If you a version of Windows older than 10, you can either install one of the
many Bash emulators for Windows or you can install a Linux virtual machine
inside your Windows computer. If you choose the latter (which we strongly
recommend instead of the emulator), you can do it using \href{https://www.virtualbox.org/wiki/Downloads}{Virtual Box} and installing a Ubuntu-based
distribution (we recommend installing either a recent version of Ubuntu or
Linux Mint 18 (or greater), since these two are most suited for beginners in
Linux). Again, directions on how to do this are straightforward and exist all
over the internet, so we will not spend time on steps on how to do that.


\subsection{From Mac OS}
\label{bash:from-mac-os}
If you have a Mac, you might already have Bash natively installed, since all
Macs are based on Unix. To find out what your shell is, you need to open a
terminal application (generally under utilities). Then type the command \sphinxcode{echo
\$SHELL} and press enter. If the output of the shell is something ending in
Bash, like \sphinxcode{/bin/bash}, then you're already running Bash. If it ends in
something else, like \sphinxcode{/bin/ksh}, then you're running a different Unix Shell.
If you want to use this different shell (and you can) most commands should be
the same, but you might have to translate a few (which should be easy Googling
\sphinxcode{bash yourcommand in csh}, for example).

If you're running another terminal and would like to try Bash, you can either
get an Bash emulator for Mac, install a Linux virtual machine (as described in
the Windows section) or change your terminal to Bash. The most recommended here
is to change your Shell to Bash. Instructions on how to do this are easy and
can again be found in many places, including \href{http://osxdaily.com/2012/03/21/change-shell-mac-os-x/}{here}, and generally use
one very simple command called \sphinxcode{chsh}.


\subsection{From Linux}
\label{bash:from-linux}
If you're running Linux you can open a terminal and run the command \sphinxcode{echo
\$SHELL} to find out if you're running Bash or not. If you're not you can try to
keep going with your Shell (some commands may need to be translated) or you can
change your default Shell with the \sphinxcode{chsh} command. You can find more detailed
information on that in many places, such as \href{http://stackoverflow.com/questions/13046192/changing-default-shell-in-linux}{here}.


\chapter{Compiling KPP}
\label{compiling::doc}\label{compiling:compiling-kpp}
We say we compile a program when we convert all the human-readable code that
directly wrote into a binary executable that your machine can actually run
directly. In other words, it's like we're installing a program into the
computer. In this chapter we detail how to successfully download and compile
KPP on your system under the Bash environment.

If from now you encounter any bug that isn't described directly in the text, we
refer you to the {\hyperref[bugs:bugs]{\sphinxcrossref{\DUrole{std,std-ref}{Possible bug fixes}}}} section. If you find a fix for an error that isn't
anywhere in the guide, we also encourage you to contribute to the guide and
include that bug fix in that section. (You can check the Introduction section
on how to contribute.)


\section{Downloading into your folder}
\label{compiling:downloading-into-your-folder}
One of the first things to be said is: most of the commands we will use will
only work if you're in the right directory (which we will always tell what it
is). So when you open a terminal, that terminal is ``running'' in some directory
in your computer.  You can find out which directory that is by entering the
command \sphinxcode{pwd} which stands for ``Print Working Directory''. That will show you
exactly where you are on your computer.

\begin{notice}{note}{Note:}
You can also use the \sphinxcode{ls} command, which will ``list'' everything you have on
that directory to get a better sense of where you are. Also, you can use the
command \sphinxcode{tree -d {}`pwd{}`}, which shows you your current directory on top, and
the subdirectories in it in a tree-like structure. Try it! This can also be used
to make you get a sense of where you are and what directories are ``around you''.
\end{notice}

To change directories, you can
use the command \sphinxcode{cd}, which stands for ``Change Directory''. So if you want to
go to your downloads directory, you can type \sphinxcode{cd Downloads}, or \sphinxcode{cd
/home/myuser/Downloads} depending on where you are on your terminal (the first
is a relative path (to your current location) and the second is an absolute or
full path; you can read more about relative and absolute paths \href{https://jeremywsherman.com/blog/2011/09/26/absolute-and-relative-paths/}{here}).

\begin{notice}{note}{Note:}
Throughout this document, we'll generally use \sphinxcode{myuser} to refer to
your username in the system. This generally comes right after \sphinxcode{/home/}
and you should change according to your case. So if your user name is \sphinxcode{john}
you'd replace \sphinxcode{/home/myuser} with \sphinxcode{/home/john} in every occasion.
\end{notice}

If you prefer to download KPP through its website manually and unpack it
somewhere, you'll have to go there with your terminal. So, if I unpack it in my
home directory, as soon as I open my terminal I'll have to use \sphinxcode{cd
/home/myuser/kpp-2.2.3}. This command will only work if the path is correct
(it might not work on Windows emulators, for example, which may place the
\sphinxcode{/home} directory elsewhere (you can always just google).

However, if you're insecure with navigating your directories using your
terminal, it's best to do everything via this second, more straightforward,
option. It uses solely commands but it's easier. First, with the terminal open
somewhere (anywhere in this case) run the following commands:

\begin{Verbatim}[commandchars=\\\{\},numbers=left,firstnumber=1,stepnumber=1]
\PYG{n+nb}{cd} \PYG{n+nv}{\PYGZdl{}HOME}
wget http://people.cs.vt.edu/\PYGZti{}asandu/Software/Kpp/Download/kpp\PYGZhy{}2.2.3\PYGZus{}Nov.2012.zip
unzip kpp\PYGZhy{}2.2.3\PYGZus{}Nov.2012.zip
\PYG{n+nb}{cd} kpp\PYGZhy{}2.2.3
\end{Verbatim}

Line one will go to your home directory, and line two will automatically download
the software there, while line three will unpack it. This will create a new
directory with all the contents of the \sphinxcode{.zip} file, so the last command line
will move to the recently-created directory, which is now the KPP directory.

\begin{notice}{note}{Note:}
This last set of commands can be run from any directory because we first moved to
the home directory (in the first line) before downloading and unpacking everything.
This was done just to make things easier and more compact, but KPP can be downloaded
and run from anywhere in your system, so if for some reason you want to download,
unpack and install it somewhere else, feel free to change the first line accordingly.
\end{notice}

Make sure you're in the correct directory by entering \sphinxcode{pwd}, which should show
you the full path to the \sphinxcode{kpp-2.2.3} directory. You can also type \sphinxcode{ls}, which should
show you a list of everything that was in the zip file:

\begin{Verbatim}[commandchars=\\\{\}]
\PYG{n}{cflags}        \PYG{n}{drv}       \PYG{n+nb}{int}                 \PYG{n}{Makefile}\PYG{o}{.}\PYG{n}{defs}  \PYG{n}{site}\PYG{o}{\PYGZhy{}}\PYG{n}{lisp}
\PYG{n}{cflags}\PYG{o}{.}\PYG{n}{guess}  \PYG{n}{examples}  \PYG{n+nb}{int}\PYG{o}{.}\PYG{n}{modified\PYGZus{}WCOPY}  \PYG{n}{models}         \PYG{n}{src}
\PYG{n}{doc}           \PYG{n}{gpl}       \PYG{n}{Makefile}            \PYG{n}{readme}         \PYG{n}{util}
\end{Verbatim}


\section{Making sure dependecies are installed}
\label{compiling:making-sure-dependecies-are-installed}
Now we are going to set-up the environment to compile KPP. The first step is to
make sure that you have the necessary software. These are called the
dependencies of a program: it is everything the program needs to be available
in the system (softwares, libraries, etc.) before it's installed.

Be sure that FLEX (which is a public domain \href{https://en.wikipedia.org/wiki/Lexical\_analysis}{lexical analizer}) is installed on your
machine. You can run \sphinxcode{flex -{-}version} and if it is installed you should see
something like \sphinxcode{flex 2.6.0}. If instead you see something like \sphinxcode{flex:
command not found} then it means that it is not installed and you're going to
have to install it by running \sphinxcode{sudo apt update \&\& sudo apt install flex} if
you're running Linux natively (depending on your Linux distribution) or by
manually downloading and installing the file if you're emulating (with Cygwin,
for example). A quick google search should tell you how to install it easily.
Note: if \sphinxcode{flex} isn't available for you, you might need to install the
Flex-dev package with \sphinxcode{sudo apt install flex-devel.x86\_64} or something
similar.

Be also sure that \sphinxcode{yacc} and \sphinxcode{sed} are installed by typing \sphinxcode{which yacc}
and \sphinxcode{which sed}. If you see something like \sphinxcode{/usr/bin/sed} or
\sphinxcode{/usr/lib/yacc} then they are installed. If you see an error message, then
you're also going to have to install it manually. Again, a quick google search
should tell you how to do it, although it is very rare that these packages
aren't installed.

\begin{notice}{note}{Note:}
\sphinxcode{flex} and \sphinxcode{yacc} have to do with \href{https://en.wikipedia.org/wiki/Lexical\_analysis}{lexical analysis} and it's not specially
important to know exactly what they do. Suffices to say that they are used
internally by the compiler to generate the executable file, but you will never
have to use them directly when compiling/using KPP.  On the other hand,
\sphinxcode{sed}, is a very useful \href{https://en.wikipedia.org/wiki/Sed}{text manipulation tool} that you might benefit from learning, but
you also won't need to use it while running KPP, so feel free to disregard it for now.
\end{notice}


\section{Telling your system where KPP is}
\label{compiling:telling-your-system-where-kpp-is}
Now that we have the dependencies installed, we need to make sure that your
computer knows where KPP is in your system. We do that by altering a file
called \sphinxcode{.bashrc}. This file is a simple text file (so can you easily open and
read it, as you'll see) with some very simple commands. Every time you start a
terminal that file is ``read'' internally by the terminal and executed. So inside
that file you can put any command that you could type in the terminal. Thus,
generally, if you want to change something in your terminal so that the change
takes place every time you start it (so you don't have to re-set that change
over and over again every time), that's the place to do it. For our purposes we
simply need to add a couple of lines. We'll do that step by step.

\begin{notice}{note}{Note:}
If you're using a terminal other than Bash the \sphinxcode{.bashrc} file will probably
have a slightly different name (like \sphinxcode{.cshrc} e.g.) and the commands might
also differ a bit, but the process and the ideas are the same! You'll just have
to probably do some quick googling.
\end{notice}

Now you need to open and edit \sphinxcode{.bashrc} from the terminal which can be done
with many programs, it really depends on what is installed for you (or what you
would like to install). The best options would be an editor that runs with a
GUI. For Windows users the best option is probably \sphinxcode{notepad++}, while for
Linux users \sphinxcode{gedit} is generally the default GUI option. You can try these
(and any other GUI plain text editors you know) with the commands \sphinxcode{gedit
\textasciitilde{}/.bashrc}, or \sphinxcode{notepad++ \textasciitilde{}/.bashrc} and so forth with the others.

If any of those work, great!, you can edit the file in an intuitive GUI editor.
If not, you're either going to have to find yourself a text editor with a GUI,
or use Nano by running the command \sphinxcode{nano \textasciitilde{}/.bashrc}. Nano is a very handy
text editor which runs on the terminal itself, however, it's not as
eye-pleasing and not as intuitive as the GUI-based ones.

\begin{notice}{note}{Note:}
If you're forced to use Nano, you should probably read this very quick
\href{http://www.howtogeek.com/howto/42980/the-beginners-guide-to-nano-the-linux-command-line-text-editor/}{tutorial}
to learn how to open, save and close files. It's not as intuitive, but it's
very easy.
\end{notice}
\begin{figure}[htbp]
\centering
\capstart

\noindent\scalebox{0.900000}{\sphinxincludegraphics{{nano_example}.png}}
\caption{.bashrc example.}\label{compiling:nano-ex}\label{compiling:id1}\end{figure}

Once you open \sphinxcode{.bashrc}, you're going to see something like Fig.
{\hyperref[compiling:nano\string-ex]{\sphinxcrossref{\DUrole{std,std-ref}{.bashrc example.}}}} (in this case open with Nano). Don't worry about the lines of
code. They're probably going to be different for you and that's OK; it really
varies a lot from system to system. You can ignore all those codes and jump to
the last line of the file. After the last line you'll include the following

\begin{Verbatim}[commandchars=\\\{\}]
\PYG{n+nb}{export} \PYG{n+nv}{KPP\PYGZus{}HOME}\PYG{o}{=}\PYG{n+nv}{\PYGZdl{}HOME}/kpp\PYGZhy{}2.2.3
\PYG{n+nb}{export} \PYG{n+nv}{PATH}\PYG{o}{=}\PYG{n+nv}{\PYGZdl{}PATH}:\PYG{n+nv}{\PYGZdl{}KPP\PYGZus{}HOME}/bin
\end{Verbatim}

That will work if you followed exactly the previous commands and installed KPP
in the home directory. If you didn't you should replace \sphinxcode{\$HOME/kpp-2.2.3}
with the output of your \sphinxcode{pwd} command which you just saved.

After this is done, you are going to save and exit. If you're using any option
with a GUI this should be straightforward. With Nano you can save and exit by
pressing control X, choosing the ``yes'' option (by hitting the ``y'' key) when it
asks you to save, and then pressing enter when asked to confirm to name of the
file to save to.

Now your terminal will know where KPP is the next times you start it. But for
the changes to make effect you need to close this terminal and open another
one. So just close the terminal you were working with, open a new one. Now, if
everything worked properly, you should be able to type \sphinxcode{cd \$KPP\_HOME} and go
automatically to your KPP directory. If this worked, we are ready for the next
step, which is telling your system how to compile KPP.


\section{Specifying how to compile}
\label{compiling:specifying-how-to-compile}
Now we actually compile (which is a way of installing) KPP. First, type
\sphinxcode{locate libfl.a} and save the output. If there is no output, use \sphinxcode{locate
libfl.sh} and save the output of that. These commands tell you where the Flex
library is, which we assured was installed somewhere in the system during the
last section. In my case the output was \sphinxcode{/usr/lib/x86\_64-linux-gnu/libfl.a}.

Now in your KPP directory, use the same text editor as before to open a file
called \sphinxcode{Makefile.defs}, which sets how Bash is going to make the executable
code for KPP (i.e., it only gives instructions to your computer on how to
compile it). So type \sphinxcode{gedit Makefile.defs}, or \sphinxcode{nano Makefile.defs} and so
on, depending on the editor you're using.

Once again, you'll see a lot of lines with comments, and the only lines that
matter are those that don't start with \sphinxcode{\#}. There should be 5 lines like this
in this file. The first one starts with \sphinxcode{CC}, which sets the C Compiler. In
this guide we will use the Gnu Compiler Collection, \sphinxcode{gcc}. So make sure that
the line which starts with \sphinxcode{CC} reads \sphinxcode{CC=gcc}.

\begin{notice}{note}{Note:}
If you prefer to use another compiler, put that one there instead of gcc.
\end{notice}

Next, since we made sure that Flex was installed, make sure the next important
line reads \sphinxcode{FLEX=flex}. On the third step, set the next variable
(\sphinxcode{FLEX\_LIB\_DIR}) with the output we just saved without the last part. So in
my case the output saved was \sphinxcode{/usr/lib/x86\_64-linux-gnu/libfl.a}, so the line
will read \sphinxcode{FLEX\_LIB\_DIR=/usr/lib/x86\_64-linux-gnu}. You should, of course,
replace your line accordingly.

The next two items define some possible extra options for the compilation and
extra directories also to include in the compilation. We don't have to
worry about those, unless maybe if we need to debug the program for some
reason. Now you can save and close/exit the file.

If we did everything correctly we can compile KPP simply by running:

\begin{Verbatim}[commandchars=\\\{\}]
\PYG{n}{make}
\end{Verbatim}

on the terminal. Many warnings are going to appear on the screen (that's
normal), but as long as no error appears, the compilation will be successful.
You can be sure it was successful by once again running \sphinxcode{ls} and seeing that
there is now one extra entry on the KPP directory called \sphinxcode{bin}:

\begin{Verbatim}[commandchars=\\\{\}]
bin           doc       gpl                 Makefile       readme     util
cflags        drv       int                 Makefile.defs  site\PYGZhy{}lisp
cflags.guess  examples  int.modified\PYGZus{}WCOPY  models         src
\end{Verbatim}

Now let's test it by running the following command:

\begin{Verbatim}[commandchars=\\\{\}]
\PYG{n}{kpp} \PYG{n}{test}
\end{Verbatim}

If the output after this command is something like

\begin{Verbatim}[commandchars=\\\{\}]
\PYG{n}{This} \PYG{o+ow}{is} \PYG{n}{KPP}\PYG{o}{\PYGZhy{}}\PYG{l+m+mf}{2.2}\PYG{o}{.}\PYG{l+m+mf}{3.}

\PYG{n}{KPP} \PYG{o+ow}{is} \PYG{n}{parsing} \PYG{n}{the} \PYG{n}{equation} \PYG{n}{file}\PYG{o}{.}
\PYG{n}{Fatal} \PYG{n}{error} \PYG{p}{:} \PYG{n}{test}\PYG{p}{:} \PYG{n}{File} \PYG{o+ow}{not} \PYG{n}{found}
\PYG{n}{Program} \PYG{n}{aborted}
\end{Verbatim}

then we know it worked. This tells you the version of KPP and that it couldn't
find any file to work with, which is fine because we didn't give it any yet. If
this works, you can skip to the next section.

If, however you get an output similar to \sphinxcode{kpp: command not found...} then
chances are that \sphinxcode{bin} is a binary executable file, while it should be a
directory containing the binary file. This should not happen, according to the
manual, but for some reason it (very) often does. We need simply to rename that
executable file and put it a directory called \sphinxcode{bin}. This can be done with
the following command:

\begin{Verbatim}[commandchars=\\\{\}]
mv bin kpp \PYG{o}{\PYGZam{}\PYGZam{}} mkdir bin \PYG{o}{\PYGZam{}\PYGZam{}} mv kpp bin
\end{Verbatim}

Try this command and then try \sphinxcode{kpp test} again. You should get the correct
output this time, meaning that the system could find KPP successfully.


\chapter{Running KPP}
\label{running::doc}\label{running:running-kpp}
Now that KPP is properly compiled, we proceed to running the first test case
to make sure it works! It's advised to have the official KPP manual along with
you during this section.


\section{The first test case}
\label{running:the-first-test-case}
We now follow the manual and begin running the Chapman stratospheric mechanism
as a test case. This will allow us to illustrate some key features when running
KPP.

In order to run a simulation on KPP, it needs three things:
\begin{itemize}
\item {} 
a \sphinxcode{.kpp} file (type \sphinxcode{ls \$KPP\_HOME/examples} to see some examples of those)

\item {} 
a \sphinxcode{.spc} file (type \sphinxcode{ls \$KPP\_HOME/models} to see some examples of those)

\item {} 
a \sphinxcode{.eqn} file (type \sphinxcode{ls \$KPP\_HOME/models} to see some examples of those)

\end{itemize}

We begin by creating a directory to run this first test. Let's call this
directory \sphinxcode{test1}. We can create this directory anywhere: even inside KPP's home
directory, although, for the sake of simplicity, let's create it in your home directory:

\begin{Verbatim}[commandchars=\\\{\}]
cd \PYGZdl{}HOME
mkdir test1
\end{Verbatim}

Now let's go to that directory with \sphinxcode{cd test1}. Following the manual, let us
create a file called \sphinxcode{small\_strato.kpp} with the following contents:

\begin{Verbatim}[commandchars=\\\{\}]
\PYG{c+c1}{\PYGZsh{}MODEL      small\PYGZus{}strato}
\PYG{c+c1}{\PYGZsh{}LANGUAGE   Fortran90}
\PYG{c+c1}{\PYGZsh{}INTEGRATOR rosenbrock}
\PYG{c+c1}{\PYGZsh{}DRIVER     general}
\end{Verbatim}

You can do this by typing \sphinxcode{notepad++ small\_strato.kpp} in the \sphinxcode{test1}
directory, if using Notepad++, or by using another editor of your choice
(replace \sphinxcode{notepad++} with \sphinxcode{gedit} for example). Then just paste the content
above in the file, save and exit it.

This file tells KPP what model to use (\sphinxcode{small\_strato.def}) and how to process
it (most importantly for us here, it tells KPP to generate a Fortran 90 code,
although it can also generate C and Matlab code). Many other options can be
added to this file and you can learn more about them in the KPP manual.

If our changes to \sphinxcode{.bashrc} are correct, then KPP should be able to find the
correct model, since the \sphinxcode{small\_strato} model (given by \sphinxcode{small\_strato.def})
is located in the \sphinxcode{models} directory, in the KPP home directory. We test this
by running KPP on our recently created file with

\begin{Verbatim}[commandchars=\\\{\}]
\PYG{n}{kpp} \PYG{n}{small\PYGZus{}strato}\PYG{o}{.}\PYG{n}{kpp}
\end{Verbatim}

You should see the following lines on your screen:

\begin{Verbatim}[commandchars=\\\{\}]
\PYG{n}{This} \PYG{o+ow}{is} \PYG{n}{KPP}\PYG{o}{\PYGZhy{}}\PYG{l+m+mf}{2.2}\PYG{o}{.}\PYG{l+m+mf}{3.}

\PYG{n}{KPP} \PYG{o+ow}{is} \PYG{n}{parsing} \PYG{n}{the} \PYG{n}{equation} \PYG{n}{file}\PYG{o}{.}
\PYG{n}{KPP} \PYG{o+ow}{is} \PYG{n}{computing} \PYG{n}{Jacobian} \PYG{n}{sparsity} \PYG{n}{structure}\PYG{o}{.}
\PYG{n}{KPP} \PYG{o+ow}{is} \PYG{n}{starting} \PYG{n}{the} \PYG{n}{code} \PYG{n}{generation}\PYG{o}{.}
\PYG{n}{KPP} \PYG{o+ow}{is} \PYG{n}{initializing} \PYG{n}{the} \PYG{n}{code} \PYG{n}{generation}\PYG{o}{.}
\PYG{n}{KPP} \PYG{o+ow}{is} \PYG{n}{generating} \PYG{n}{the} \PYG{n}{monitor} \PYG{n}{data}\PYG{p}{:}
    \PYG{o}{\PYGZhy{}} \PYG{n}{small\PYGZus{}strato\PYGZus{}Monitor}
\PYG{n}{KPP} \PYG{o+ow}{is} \PYG{n}{generating} \PYG{n}{the} \PYG{n}{utility} \PYG{n}{data}\PYG{p}{:}
    \PYG{o}{\PYGZhy{}} \PYG{n}{small\PYGZus{}strato\PYGZus{}Util}
\PYG{n}{KPP} \PYG{o+ow}{is} \PYG{n}{generating} \PYG{n}{the} \PYG{k}{global} \PYG{n}{declarations}\PYG{p}{:}
    \PYG{o}{\PYGZhy{}} \PYG{n}{small\PYGZus{}strato\PYGZus{}Main}
\PYG{n}{KPP} \PYG{o+ow}{is} \PYG{n}{generating} \PYG{n}{the} \PYG{n}{ODE} \PYG{n}{function}\PYG{p}{:}
    \PYG{o}{\PYGZhy{}} \PYG{n}{small\PYGZus{}strato\PYGZus{}Function}
\PYG{n}{KPP} \PYG{o+ow}{is} \PYG{n}{generating} \PYG{n}{the} \PYG{n}{ODE} \PYG{n}{Jacobian}\PYG{p}{:}
    \PYG{o}{\PYGZhy{}} \PYG{n}{small\PYGZus{}strato\PYGZus{}Jacobian}
    \PYG{o}{\PYGZhy{}} \PYG{n}{small\PYGZus{}strato\PYGZus{}JacobianSP}
\PYG{n}{KPP} \PYG{o+ow}{is} \PYG{n}{generating} \PYG{n}{the} \PYG{n}{linear} \PYG{n}{algebra} \PYG{n}{routines}\PYG{p}{:}
    \PYG{o}{\PYGZhy{}} \PYG{n}{small\PYGZus{}strato\PYGZus{}LinearAlgebra}
\PYG{n}{KPP} \PYG{o+ow}{is} \PYG{n}{generating} \PYG{n}{the} \PYG{n}{Hessian}\PYG{p}{:}
    \PYG{o}{\PYGZhy{}} \PYG{n}{small\PYGZus{}strato\PYGZus{}Hessian}
    \PYG{o}{\PYGZhy{}} \PYG{n}{small\PYGZus{}strato\PYGZus{}HessianSP}
\PYG{n}{KPP} \PYG{o+ow}{is} \PYG{n}{generating} \PYG{n}{the} \PYG{n}{utility} \PYG{n}{functions}\PYG{p}{:}
    \PYG{o}{\PYGZhy{}} \PYG{n}{small\PYGZus{}strato\PYGZus{}Util}
\PYG{n}{KPP} \PYG{o+ow}{is} \PYG{n}{generating} \PYG{n}{the} \PYG{n}{rate} \PYG{n}{laws}\PYG{p}{:}
    \PYG{o}{\PYGZhy{}} \PYG{n}{small\PYGZus{}strato\PYGZus{}Rates}
\PYG{n}{KPP} \PYG{o+ow}{is} \PYG{n}{generating} \PYG{n}{the} \PYG{n}{parameters}\PYG{p}{:}
    \PYG{o}{\PYGZhy{}} \PYG{n}{small\PYGZus{}strato\PYGZus{}Parameters}
\PYG{n}{KPP} \PYG{o+ow}{is} \PYG{n}{generating} \PYG{n}{the} \PYG{k}{global} \PYG{n}{data}\PYG{p}{:}
    \PYG{o}{\PYGZhy{}} \PYG{n}{small\PYGZus{}strato\PYGZus{}Global}
\PYG{n}{KPP} \PYG{o+ow}{is} \PYG{n}{generating} \PYG{n}{the} \PYG{n}{stoichiometric} \PYG{n}{description} \PYG{n}{files}\PYG{p}{:}
    \PYG{o}{\PYGZhy{}} \PYG{n}{small\PYGZus{}strato\PYGZus{}Stoichiom}
    \PYG{o}{\PYGZhy{}} \PYG{n}{small\PYGZus{}strato\PYGZus{}StoichiomSP}
\PYG{n}{KPP} \PYG{o+ow}{is} \PYG{n}{generating} \PYG{n}{the} \PYG{n}{driver} \PYG{k+kn}{from} \PYG{n+nn}{none}\PYG{n+nn}{.}\PYG{n+nn}{f90}\PYG{p}{:}
    \PYG{o}{\PYGZhy{}} \PYG{n}{small\PYGZus{}strato\PYGZus{}Main}
\PYG{n}{KPP} \PYG{o+ow}{is} \PYG{n}{starting} \PYG{n}{the} \PYG{n}{code} \PYG{n}{post}\PYG{o}{\PYGZhy{}}\PYG{n}{processing}\PYG{o}{.}

\PYG{n}{KPP} \PYG{n}{has} \PYG{n}{succesfully} \PYG{n}{created} \PYG{n}{the} \PYG{n}{model} \PYG{l+s+s2}{\PYGZdq{}}\PYG{l+s+s2}{small\PYGZus{}strato}\PYG{l+s+s2}{\PYGZdq{}}\PYG{o}{.}
\end{Verbatim}

\begin{notice}{note}{Note:}
If you get an error message here, go back a few steps and make sure the \sphinxcode{\$KPP\_HOME}
and the \sphinxcode{\$PATH} variable are set correctly, and be sure that both KPP can be
found and the correct model files \sphinxcode{small\_strato} are in \sphinxcode{\$KPP\_HOME/models}.
\end{notice}

If indeed you see this output (or something very similar) it means you were
successful in creating the model. Now if you list your files with the \sphinxcode{ls}
command, you'll see many new files:

\begin{Verbatim}[commandchars=\\\{\}]
\PYG{n}{Makefile\PYGZus{}small\PYGZus{}strato}           \PYG{n}{small\PYGZus{}strato}\PYG{o}{.}\PYG{n}{map}
\PYG{n}{small\PYGZus{}strato\PYGZus{}Function}\PYG{o}{.}\PYG{n}{f90}       \PYG{n}{small\PYGZus{}strato\PYGZus{}mex\PYGZus{}Fun}\PYG{o}{.}\PYG{n}{f90}
\PYG{n}{small\PYGZus{}strato\PYGZus{}Global}\PYG{o}{.}\PYG{n}{f90}         \PYG{n}{small\PYGZus{}strato\PYGZus{}mex\PYGZus{}Hessian}\PYG{o}{.}\PYG{n}{f90}
\PYG{n}{small\PYGZus{}strato\PYGZus{}Hessian}\PYG{o}{.}\PYG{n}{f90}        \PYG{n}{small\PYGZus{}strato\PYGZus{}mex\PYGZus{}Jac\PYGZus{}SP}\PYG{o}{.}\PYG{n}{f90}
\PYG{n}{small\PYGZus{}strato\PYGZus{}HessianSP}\PYG{o}{.}\PYG{n}{f90}      \PYG{n}{small\PYGZus{}strato\PYGZus{}Model}\PYG{o}{.}\PYG{n}{f90}
\PYG{n}{small\PYGZus{}strato\PYGZus{}Initialize}\PYG{o}{.}\PYG{n}{f90}     \PYG{n}{small\PYGZus{}strato\PYGZus{}Monitor}\PYG{o}{.}\PYG{n}{f90}
\PYG{n}{small\PYGZus{}strato\PYGZus{}Integrator}\PYG{o}{.}\PYG{n}{f90}     \PYG{n}{small\PYGZus{}strato\PYGZus{}Parameters}\PYG{o}{.}\PYG{n}{f90}
\PYG{n}{small\PYGZus{}strato\PYGZus{}Jacobian}\PYG{o}{.}\PYG{n}{f90}       \PYG{n}{small\PYGZus{}strato\PYGZus{}Precision}\PYG{o}{.}\PYG{n}{f90}
\PYG{n}{small\PYGZus{}strato\PYGZus{}JacobianSP}\PYG{o}{.}\PYG{n}{f90}     \PYG{n}{small\PYGZus{}strato\PYGZus{}Rates}\PYG{o}{.}\PYG{n}{f90}
\PYG{n}{small\PYGZus{}strato}\PYG{o}{.}\PYG{n}{kpp}                \PYG{n}{small\PYGZus{}strato\PYGZus{}Stoichiom}\PYG{o}{.}\PYG{n}{f90}
\PYG{n}{small\PYGZus{}strato\PYGZus{}LinearAlgebra}\PYG{o}{.}\PYG{n}{f90}  \PYG{n}{small\PYGZus{}strato\PYGZus{}StoichiomSP}\PYG{o}{.}\PYG{n}{f90}
\PYG{n}{small\PYGZus{}strato\PYGZus{}Main}\PYG{o}{.}\PYG{n}{f90}           \PYG{n}{small\PYGZus{}strato\PYGZus{}Util}\PYG{o}{.}\PYG{n}{f90}
\end{Verbatim}

Most of them end with a \sphinxcode{.f90} extension, which tells us they are Fortran 90
codes. These codes have to be compiled into an executable file which is what
will actually process and run the kinetic model. So the next step is to compile
every one of those code together into one executable and run it. To do that,
let's focus for now on the \sphinxcode{Makefile\_small\_strato}. This is a text file that
tells your computer which Fortran compiler to use to compile, which files to
use, etc. We need to modify it, so open the \sphinxcode{Makefile\_small\_strato} file
(again using your preferred editor) and find where it says

\begin{Verbatim}[commandchars=\\\{\}]
\PYG{c+c1}{\PYGZsh{}COMPILER = G95}
\PYG{c+c1}{\PYGZsh{}COMPILER = LAHEY}
\PYG{n+nv}{COMPILER} \PYG{o}{=} INTEL
\PYG{c+c1}{\PYGZsh{}COMPILER = PGF}
\PYG{c+c1}{\PYGZsh{}COMPILER = HPUX}
\PYG{c+c1}{\PYGZsh{}COMPILER = GFORTRAN}
\end{Verbatim}

Each of the lines is a different Fortran compiler, and your computer is only
going to see the line that doesn't start with a \sphinxcode{\#} (we say that the lines
with \sphinxcode{\#} are commented and therefore the computer doesn't ``see'' them). So,
currently, these lines are telling the computer to use the Intel Fortran
compiler, \sphinxcode{ifort}.

If you are using \sphinxcode{ifort}, you should leave it as it is. However, \sphinxcode{ifort} is
paid, so chances are you are using another compiler. If this is the case, put
the \sphinxcode{\#} in front of the \sphinxcode{INTEL} options and take it out of the line which
has the name of your compiler. If you don't know which compiler you have,
chances are you have gfortran, which is free and the one we will use here. You
can install gfortran with \sphinxcode{sudo apt install gfortran} (or the equivalent
installation command for your system).
So, for gfortran, you should make the above lines of code look like
the following:

\begin{Verbatim}[commandchars=\\\{\}]
\PYG{c+c1}{\PYGZsh{}COMPILER = G95}
\PYG{c+c1}{\PYGZsh{}COMPILER = LAHEY}
\PYG{c+c1}{\PYGZsh{}COMPILER = INTEL}
\PYG{c+c1}{\PYGZsh{}COMPILER = PGF}
\PYG{c+c1}{\PYGZsh{}COMPILER = HPUX}
\PYG{n+nv}{COMPILER} \PYG{o}{=} GFORTRAN
\end{Verbatim}

When doing that we say that we ``uncommented'' the gfortran line, since every
line that starts with a \sphinxcode{\#} is commented and not read by the system. You can
save and exit the file.

Now all you have to do is run the following command:

\begin{Verbatim}[commandchars=\\\{\}]
\PYG{n}{make} \PYG{o}{\PYGZhy{}}\PYG{n}{f} \PYG{n}{Makefile\PYGZus{}small\PYGZus{}strato} \PYG{n}{which} \PYG{n}{will} \PYG{n+nb}{compile} \PYG{n}{your} \PYG{n}{Fortran} \PYG{n}{code} \PYG{n}{into} \PYG{n}{an}
\end{Verbatim}

executable file (\sphinxcode{.exe}) using the options we just set. You should see a lot
of lines appearing on screen starting with \sphinxcode{gfortran}, maybe some warnings,
and if no error messages appear the compilation was successful.

Now you'll see many more new files, including one called \sphinxcode{small\_strato.exe},
which is your executable file (run \sphinxcode{ls} again list everything and see that).
This is the executable that will actually calculate the concentrations using
the model.

To test if it works, run the following command:

\begin{Verbatim}[commandchars=\\\{\}]
\PYG{o}{.}\PYG{o}{/}\PYG{n}{small\PYGZus{}strato}\PYG{o}{.}\PYG{n}{exe}
\end{Verbatim}

which will run the executable.
You should see some output on the screen with concentrations, like Fig. {\hyperref[running:test1\string-output]{\sphinxcrossref{\DUrole{std,std-ref}{Output concentrations of the first test case.}}}}
\begin{figure}[htbp]
\centering
\capstart

\noindent\scalebox{0.900000}{\sphinxincludegraphics{{test1_output}.png}}
\caption{Output concentrations of the first test case.}\label{running:test1-output}\label{running:id1}\end{figure}

If this is the case, then your run was successful and everything worked well!
You just calculated the concentrations of the compounds in the \sphinxcode{small\_strato}
model with the pre-defined initial conditions.


\section{Understanding the test case}
\label{running:understanding-the-test-case}
Now let's understand why our run of \sphinxcode{small\_strato.exe} was successful and what
happened. First, by running \sphinxcode{kpp small\_strato}, what we did was to tell KPP
to open a file called \sphinxcode{small\_strato.kpp}, in the current directory and do what that
file tells it to do. In the first line of the file there is the command

\begin{Verbatim}[commandchars=\\\{\}]
\PYG{c+c1}{\PYGZsh{}MODEL      small\PYGZus{}strato}
\end{Verbatim}

which tells KPP to look for a file called \sphinxcode{small\_strato.def}. Since the file
is in KPP's models directory (at \sphinxcode{\$HOME\_KPP/models}), KPP had no problems
finding it. This file has the initial concentrations you want to use in the
model, the time step, etc. It also links two other files (\sphinxcode{small\_strato.spc}
and \sphinxcode{small\_strato.eqn}), which tell KPP with chemical species and chemical
equations to use (effectively defining the mechanism).

After receiving all that information, KPP finally creates a Fortran 90 code
(because it says so in the \sphinxcode{small\_strato.kpp} we created) with our small
stratospheric model containing our pre-defined initial conditions, time step,
chemical reactions and so on.

The code, however, has to be compiled before run, so that is why we issued the
command \sphinxcode{make}, which compiles the code according to the file
\sphinxcode{Makefile\_small\_strato} (which is where we specified the Fortran compiler).
This step creates an executable file, which has the extension \sphinxcode{.exe} and is
ready to be run. By running the \sphinxcode{.exe} file we ran a program that got our
initial concentrations of the species we defined and, based on the chemical
reactions, calculated, step by step, their concentrations in each time step.

At each step, the model is not only printing the concentrations on screen, but
it is also writing them into a file called \sphinxcode{small\_strato.dat}, which is a
column-separated text file. This file can be used to see, plot, make
calculations with the data and so on. However, you should be careful because
the order of the concentrations that appear on screen isn't the same order KPP
uses for the \sphinxcode{.dat} file. You can learn about the ordering at page 7 of the
KPP manual, but a good rule of thumb is to check the file with a \sphinxcode{.map}
extension (in this case, \sphinxcode{small\_strato.map}) and take a look at the
\sphinxcode{species} section. The file output order is the ordering of the variable
species followed by the species on the fixed species.

In the case of \sphinxcode{small\_strato} the order printed on the file (you can check it on
\sphinxcode{small\_strato.map}) is

\begin{Verbatim}[commandchars=\\\{\}]
\PYG{n}{time}\PYG{p}{,} \PYG{n}{O1D}\PYG{p}{,} \PYG{n}{O}\PYG{p}{,} \PYG{n}{O3}\PYG{p}{,} \PYG{n}{NO}\PYG{p}{,} \PYG{n}{NO2}\PYG{p}{,} \PYG{n}{M}\PYG{p}{,} \PYG{n}{O2}
\end{Verbatim}

The time is always going to be the first column, and it is always going to be
in hours since the start of the simulation. Since the solar forcing matters
here, we need to keep track of the time of day that the simulation started. In
this case it was at noon, because that's the way the \sphinxcode{.def} file is set (we
will talk about this in more detail in the sections to come).

We can read that data in many ways. I present below a quick python script
to plot the concentrations as a function of the hour of the day

\begin{Verbatim}[commandchars=\\\{\},numbers=left,firstnumber=1,stepnumber=1]
\PYG{k+kn}{import} \PYG{n+nn}{pandas} \PYG{k}{as} \PYG{n+nn}{pd}
\PYG{k+kn}{from} \PYG{n+nn}{matplotlib} \PYG{k}{import} \PYG{n}{pyplot} \PYG{k}{as} \PYG{n}{plt}
\PYG{n}{concs} \PYG{o}{=} \PYG{n}{pd}\PYG{o}{.}\PYG{n}{read\PYGZus{}csv}\PYG{p}{(}\PYG{l+s+s1}{\PYGZsq{}}\PYG{l+s+s1}{small\PYGZus{}strato.dat}\PYG{l+s+s1}{\PYGZsq{}}\PYG{p}{,} \PYG{n}{index\PYGZus{}col}\PYG{o}{=}\PYG{l+m+mi}{0}\PYG{p}{,} \PYG{n}{delim\PYGZus{}whitespace}\PYG{o}{=}\PYG{k+kc}{True}\PYG{p}{,} \PYG{n}{header}\PYG{o}{=}\PYG{k+kc}{None}\PYG{p}{)}\PYG{o}{.}\PYG{n}{apply}\PYG{p}{(}\PYG{n}{pd}\PYG{o}{.}\PYG{n}{to\PYGZus{}numeric}\PYG{p}{,} \PYG{n}{errors}\PYG{o}{=}\PYG{l+s+s1}{\PYGZsq{}}\PYG{l+s+s1}{coerce}\PYG{l+s+s1}{\PYGZsq{}}\PYG{p}{)}
\PYG{n}{concs}\PYG{o}{.}\PYG{n}{columns} \PYG{o}{=} \PYG{p}{[}\PYG{l+s+s1}{\PYGZsq{}}\PYG{l+s+s1}{O1D}\PYG{l+s+s1}{\PYGZsq{}}\PYG{p}{,} \PYG{l+s+s1}{\PYGZsq{}}\PYG{l+s+s1}{O}\PYG{l+s+s1}{\PYGZsq{}}\PYG{p}{,} \PYG{l+s+s1}{\PYGZsq{}}\PYG{l+s+s1}{O3}\PYG{l+s+s1}{\PYGZsq{}}\PYG{p}{,} \PYG{l+s+s1}{\PYGZsq{}}\PYG{l+s+s1}{NO}\PYG{l+s+s1}{\PYGZsq{}}\PYG{p}{,} \PYG{l+s+s1}{\PYGZsq{}}\PYG{l+s+s1}{NO2}\PYG{l+s+s1}{\PYGZsq{}}\PYG{p}{,} \PYG{l+s+s1}{\PYGZsq{}}\PYG{l+s+s1}{M}\PYG{l+s+s1}{\PYGZsq{}}\PYG{p}{,} \PYG{l+s+s1}{\PYGZsq{}}\PYG{l+s+s1}{O2}\PYG{l+s+s1}{\PYGZsq{}}\PYG{p}{]}
\PYG{n}{concs}\PYG{o}{.}\PYG{n}{index}\PYG{o}{.}\PYG{n}{name} \PYG{o}{=} \PYG{l+s+s1}{\PYGZsq{}}\PYG{l+s+s1}{Hours since noon}\PYG{l+s+s1}{\PYGZsq{}}
\PYG{n}{concs}\PYG{o}{.}\PYG{n}{plot}\PYG{p}{(}\PYG{n}{ylim}\PYG{o}{=}\PYG{p}{[}\PYG{l+m+mf}{1.e8}\PYG{p}{,} \PYG{k+kc}{None}\PYG{p}{]}\PYG{p}{,} \PYG{n}{logy}\PYG{o}{=}\PYG{k+kc}{True}\PYG{p}{,} \PYG{n}{y}\PYG{o}{=}\PYG{p}{[}\PYG{l+s+s1}{\PYGZsq{}}\PYG{l+s+s1}{O3}\PYG{l+s+s1}{\PYGZsq{}}\PYG{p}{,} \PYG{l+s+s1}{\PYGZsq{}}\PYG{l+s+s1}{NO}\PYG{l+s+s1}{\PYGZsq{}}\PYG{p}{,} \PYG{l+s+s1}{\PYGZsq{}}\PYG{l+s+s1}{NO2}\PYG{l+s+s1}{\PYGZsq{}}\PYG{p}{]}\PYG{p}{,} \PYG{n}{grid}\PYG{o}{=}\PYG{k+kc}{True}\PYG{p}{)}
\PYG{n}{plt}\PYG{o}{.}\PYG{n}{savefig}\PYG{p}{(}\PYG{l+s+s1}{\PYGZsq{}}\PYG{l+s+s1}{test1\PYGZus{}time.png}\PYG{l+s+s1}{\PYGZsq{}}\PYG{p}{)}
\end{Verbatim}

\begin{notice}{note}{Note:}
KPP has a small issue with formatting and sometimes prints a number that can't
be read because some strings are missing. For example, printing \sphinxcode{3.4562-313}.
This can't be normally read and it's supposed to be \sphinxcode{3.4562E-313} and this
(apparently) only happens when the number is close to machine-precision (which
we would interpret as zero). The program above takes this issue into
consideration (in line 3) when reading the file, but you should pay attention to that when
trying to read with by other means.
\end{notice}

If you have ever seen python before, this code should be pretty intuitive. If
you haven't you can still use it easily (depending on how you got python, you
might have to install python's \sphinxcode{pandas} package).  This code generates the
following plot of the concentrations:
\phantomsection\label{running:test1-time}\begin{figure}[htbp]
\centering

\noindent\scalebox{0.800000}{\sphinxincludegraphics{{test1_time}.png}}
\label{running:test1-time}\end{figure}

We can see that the NOx concentrations follow the solar cycle, which is
indicative that the model is indeed working properly. However we see that the
O3 concentrations still haven't stabilized. This tells us that we need to run
the model for longer. Let us take this chance to modify the \sphinxcode{small\_strato}
example a bit, try and make the O3 concentrations stabilize and learn how to
alter/create models.


\chapter{Modifying and improving the example}
\label{improving:modifying-and-improving-the-example}\label{improving::doc}

\section{Increasing the length of the simulation}
\label{improving:increasing-the-length-of-the-simulation}
There are two ways to make modifications on the model. The first, which works
for simple changes, is to modify the Fortran code itself. The second is to
change the KPP model itself (the \sphinxcode{.def} files etc.) before it gets compiled.
This latter method is more general, so this is the one we will focus on this
guide.  Since all we want for now is to increase the total time, we will base
ourselves in the original \sphinxcode{small\_strato} model and only modify this
parameter.

First, create another directory (anywhere you want) called \sphinxcode{test2} (with
\sphinxcode{mkdir test2}) and enter it (with \sphinxcode{cd test2}). Now create a file called
\sphinxcode{my\_strato.kpp} (with \sphinxcode{notepad++ my\_strato.kpp} or \sphinxcode{gedit my\_strato.kpp}
or whichever text editor you ended up using) and paste the following lines in
the file:

\begin{Verbatim}[commandchars=\\\{\}]
\PYG{c+c1}{\PYGZsh{}MODEL      my\PYGZus{}strato}
\PYG{c+c1}{\PYGZsh{}LANGUAGE   Fortran90}
\PYG{c+c1}{\PYGZsh{}INTEGRATOR rosenbrock}
\PYG{c+c1}{\PYGZsh{}DRIVER     general}
\end{Verbatim}

At this point if you run \sphinxcode{kpp my\_strato.kpp} you should get an error saying
\sphinxcode{"Fatal error : my\_strato.def: Can't read file"}. Which appears because we
instructed KPP to search for the file \sphinxcode{my\_strato.def}, which doesn't exist
anywhere. So we first must create the \sphinxcode{my\_strato.def} file, which ultimately
defines the \sphinxcode{my\_strato} model.

Let us define our model based on \sphinxcode{small\_strato}, since for now all we want to
do is to modify the time length of the simulation. In order to preserve the
original \sphinxcode{small\_strato.def} we'll copy it and call it \sphinxcode{my\_strato.def}, this
way we can do any modification on \sphinxcode{my\_strato} and the original
\sphinxcode{small\_strato} will be safe. You can copy the file from the \sphinxcode{models}
directory into our working directory (\sphinxcode{test2}) by issuing the following command:

\begin{Verbatim}[commandchars=\\\{\}]
cp \PYGZdl{}KPP\PYGZus{}HOME/models/small\PYGZus{}strato.def my\PYGZus{}strato.def
\end{Verbatim}

\begin{notice}{note}{Note:}
When we run KPP from any directory (say, \sphinxcode{test2}), KPP will first look for
the files in the current directory and then in its home directory. So can
either put our model files in our KPP ``models'' directory or in our current
directory. In this guide we'll always prefer to keep the model we create/modify
in the current directory.
\end{notice}

Now you should open the file we just created (for example with \sphinxcode{notepad++})
and find the lines that look like:

\begin{Verbatim}[commandchars=\\\{\}]
\PYG{c+c1}{\PYGZsh{}INLINE F90\PYGZus{}INIT}
        \PYG{n}{TSTART} \PYG{o}{=} \PYG{p}{(}\PYG{l+m+mi}{12}\PYG{o}{*}\PYG{l+m+mi}{3600}\PYG{p}{)}
        \PYG{n}{TEND} \PYG{o}{=} \PYG{n}{TSTART} \PYG{o}{+} \PYG{p}{(}\PYG{l+m+mi}{3}\PYG{o}{*}\PYG{l+m+mi}{24}\PYG{o}{*}\PYG{l+m+mi}{3600}\PYG{p}{)}
        \PYG{n}{DT} \PYG{o}{=} \PYG{l+m+mf}{0.25}\PYG{o}{*}\PYG{l+m+mi}{3600}
        \PYG{n}{TEMP} \PYG{o}{=} \PYG{l+m+mi}{270}
\PYG{c+c1}{\PYGZsh{}ENDINLINE}
\end{Verbatim}

These are the lines that define the start, end, time step and global
temperature of the model. For now, we'll change only the end time. Notice that
they are given in seconds. For example, \sphinxcode{3*24*3600} is the amount of seconds
in 3 days, meaning that at the moment the simulation is set to run for 3 days,
which we saw is not enough.  Let us then replace \sphinxcode{3} with \sphinxcode{30}, meaning we
will run it for 30 days, to be sure that equilibrium is reached. Now that line
should read \sphinxcode{TEND = TSTART + (30*24*3600)}. Please also note that \sphinxcode{TSTART}
is set to \sphinxcode{12*3600}, or 12 hours, which means that the starting time of the
simulation is at noon.

After this small change we are ready to test run the model. Run \sphinxcode{kpp
my\_strato.kpp}, which should end with a \sphinxcode{succesfully created the model}
message. Now, just like with the previous example, open again the file
\sphinxcode{Makefile\_my\_strato} and uncomment the line that says \sphinxcode{COMPILER =
GFORTRAN}, so we can use gfortran instead of Intel. After this is done compile
the model with:

\begin{Verbatim}[commandchars=\\\{\}]
\PYG{n}{make} \PYG{o}{\PYGZhy{}}\PYG{n}{f} \PYG{n}{Makefile\PYGZus{}my\PYGZus{}strato}
\end{Verbatim}

Again, if everything goes well, the \sphinxcode{my\_strato.exe} file should be created.
You can run \sphinxcode{ls -tr} (which lists every file on your directory ordered by
creation time) and see that the last file listed should be the \sphinxcode{.exe} file.

We can now run the model with \sphinxcode{./my\_strato.exe}, which should now take 10
times longer to complete, since we are running it for 10 times as long as
before.  We can use the Python code given in the last section to read the
results. We just need to adjust the name inside the code from \sphinxcode{small\_strato}
to \sphinxcode{my\_strato}.

The plot of the results is
\phantomsection\label{improving:test2-time}\begin{figure}[htbp]
\centering

\noindent\scalebox{0.800000}{\sphinxincludegraphics{{test2_time}.png}}
\label{improving:test2-time}\end{figure}

Now we can see that the solution reaches equilibrium after roughly 200 hours.
This, however, was only a minor change in the model. Let us now learn how to
change some other parameters and examine how the model reacts.


\section{Change in the initial conditions}
\label{improving:change-in-the-initial-conditions}
We try this next change in the same model as before (\sphinxcode{my\_strato}). Let's open
\sphinxcode{my\_strato.def} and consider an atmosphere with more NO (simulating a more
polluted condition). Where it says \sphinxcode{NO = 8.725E+08;}, we make it read \sphinxcode{NO  =
9.00E+09}, which is roughly a 10 times increase in the NO concentration. Let
us also change the O3 initial condition and make the O3 line read \sphinxcode{O3  =
5.00E+10;}, simulating an atmosphere that has a lower initial O3
concentration.

Let us also change the line that reads:

\begin{Verbatim}[commandchars=\\\{\}]
\PYG{c+c1}{\PYGZsh{}LOOKATALL          \PYGZob{}File Output\PYGZcb{}}
\end{Verbatim}

We will make it read

\begin{Verbatim}[commandchars=\\\{\}]
\PYG{c+c1}{\PYGZsh{}LOOKAT O3; NO; NO2; \PYGZob{}File Output\PYGZcb{}}
\end{Verbatim}

This latter change makes the program write only time, O3, NO and NO2 into the
output file. This simplifies the reading process and saves space, but by doing
that we are assuming that we're not interested in the other species.

\begin{notice}{note}{Note:}
This will change the order of the output in the file. But again, checking the
\sphinxcode{.map} will give you the correct order. This time you'll have to consider
only the species you specified to be on the output, so, e.g., in this case the
order in the \sphinxcode{.map} file is: 1 = O1D, 2 = O, 3 = O3, 4 = NO, 5 = NO2. But since we are writing
only O3, NO and NO2, our output file will have the order (time,) O3, NO, NO2
(which are numbers 3, 4, 5, respectively).  This will have to be done every
time the \sphinxcode{\#LOOKAT} parameter is used.
\end{notice}

We again go through the same steps: run it with \sphinxcode{kpp my\_strato.kpp}, change
the compiler to gfortran, compile if with \sphinxcode{make -f Makefile\_my\_strato} and
run it with \sphinxcode{./my\_strato.exe}.  This time, since the output file changed, we
have to change the code to read it correctly:

\begin{Verbatim}[commandchars=\\\{\}]
\PYG{k+kn}{import} \PYG{n+nn}{pandas} \PYG{k}{as} \PYG{n+nn}{pd}
\PYG{k+kn}{from} \PYG{n+nn}{matplotlib} \PYG{k}{import} \PYG{n}{pyplot} \PYG{k}{as} \PYG{n}{plt}
\PYG{n}{concs} \PYG{o}{=} \PYG{n}{pd}\PYG{o}{.}\PYG{n}{read\PYGZus{}csv}\PYG{p}{(}\PYG{l+s+s1}{\PYGZsq{}}\PYG{l+s+s1}{my\PYGZus{}strato.dat}\PYG{l+s+s1}{\PYGZsq{}}\PYG{p}{,} \PYG{n}{index\PYGZus{}col}\PYG{o}{=}\PYG{l+m+mi}{0}\PYG{p}{,} \PYG{n}{delim\PYGZus{}whitespace}\PYG{o}{=}\PYG{k+kc}{True}\PYG{p}{,} \PYG{n}{header}\PYG{o}{=}\PYG{k+kc}{None}\PYG{p}{,} \PYG{n}{dtype}\PYG{o}{=}\PYG{k+kc}{None}\PYG{p}{)}\PYG{o}{.}\PYG{n}{apply}\PYG{p}{(}\PYG{n}{pd}\PYG{o}{.}\PYG{n}{to\PYGZus{}numeric}\PYG{p}{,} \PYG{n}{errors}\PYG{o}{=}\PYG{l+s+s1}{\PYGZsq{}}\PYG{l+s+s1}{coerce}\PYG{l+s+s1}{\PYGZsq{}}\PYG{p}{)}
\PYG{n}{concs}\PYG{o}{.}\PYG{n}{columns} \PYG{o}{=} \PYG{p}{[}\PYG{l+s+s1}{\PYGZsq{}}\PYG{l+s+s1}{O3}\PYG{l+s+s1}{\PYGZsq{}}\PYG{p}{,} \PYG{l+s+s1}{\PYGZsq{}}\PYG{l+s+s1}{NO}\PYG{l+s+s1}{\PYGZsq{}}\PYG{p}{,} \PYG{l+s+s1}{\PYGZsq{}}\PYG{l+s+s1}{NO2}\PYG{l+s+s1}{\PYGZsq{}}\PYG{p}{]}
\PYG{n}{concs}\PYG{o}{.}\PYG{n}{index}\PYG{o}{.}\PYG{n}{name} \PYG{o}{=} \PYG{l+s+s1}{\PYGZsq{}}\PYG{l+s+s1}{Hours since noon}\PYG{l+s+s1}{\PYGZsq{}}
\PYG{n}{concs}\PYG{o}{.}\PYG{n}{plot}\PYG{p}{(}\PYG{n}{ylim}\PYG{o}{=}\PYG{p}{[}\PYG{l+m+mf}{1.e8}\PYG{p}{,} \PYG{k+kc}{None}\PYG{p}{]}\PYG{p}{,} \PYG{n}{logy}\PYG{o}{=}\PYG{k+kc}{True}\PYG{p}{,} \PYG{n}{y}\PYG{o}{=}\PYG{p}{[}\PYG{l+s+s1}{\PYGZsq{}}\PYG{l+s+s1}{O3}\PYG{l+s+s1}{\PYGZsq{}}\PYG{p}{,} \PYG{l+s+s1}{\PYGZsq{}}\PYG{l+s+s1}{NO}\PYG{l+s+s1}{\PYGZsq{}}\PYG{p}{,} \PYG{l+s+s1}{\PYGZsq{}}\PYG{l+s+s1}{NO2}\PYG{l+s+s1}{\PYGZsq{}}\PYG{p}{]}\PYG{p}{,} \PYG{n}{grid}\PYG{o}{=}\PYG{k+kc}{True}\PYG{p}{)}
\PYG{n}{plt}\PYG{o}{.}\PYG{n}{savefig}\PYG{p}{(}\PYG{l+s+s1}{\PYGZsq{}}\PYG{l+s+s1}{test21\PYGZus{}time.png}\PYG{l+s+s1}{\PYGZsq{}}\PYG{p}{)}
\end{Verbatim}

This code produces the following plot:
\phantomsection\label{improving:test21-time}\begin{figure}[htbp]
\centering

\noindent\scalebox{0.800000}{\sphinxincludegraphics{{test21_time}.png}}
\label{improving:test21-time}\end{figure}

From which we can see that the concentration of ozone stabilized more quickly
in this case. As you can see, we can play around with the initial conditions
as much as we want and analyse that result of model. In fact, we encourage you
to do so. However, let us focus this guide on the next step and modify some more
fundamental aspects of the model: the reaction rates.


\section{Modifying the reactions}
\label{improving:modifying-the-reactions}
Now we will alter the reaction rates of some reactions in the model. Keep in
mind that these alterations are not meant to be realistic. They are simply done
here for the sake of learning how the model works.

Begin again by creating a \sphinxcode{test3} directory anywhere and going into it
(\sphinxcode{mkdir test3 \&\& cd test3}). In this directory, create a file called
\sphinxcode{strato3.kpp} with the following contents:

\begin{Verbatim}[commandchars=\\\{\}]
\PYG{c+c1}{\PYGZsh{}MODEL      strato3}
\PYG{c+c1}{\PYGZsh{}LANGUAGE   Fortran90}
\PYG{c+c1}{\PYGZsh{}INTEGRATOR rosenbrock}
\PYG{c+c1}{\PYGZsh{}DRIVER     general}
\end{Verbatim}

This file tells KPP to look for the \sphinxcode{strato3.def} file. So let us create this
file by again copying the \sphinxcode{small\_strato.def} file to our current working directory.
You can do that with:

\begin{Verbatim}[commandchars=\\\{\}]
cp \PYGZdl{}KPP\PYGZus{}HOME/models/small\PYGZus{}strato.def strato3.def
\end{Verbatim}

Open the file (\sphinxcode{notepad++ strato3.def}) and find the first two lines
which originally read

\begin{Verbatim}[commandchars=\\\{\}]
\PYG{c+c1}{\PYGZsh{}include small\PYGZus{}strato.spc}
\PYG{c+c1}{\PYGZsh{}include small\PYGZus{}strato.eqn}
\end{Verbatim}

Which still tells KPP to look for the original \sphinxcode{small\_strato} model files
when defining the species (\sphinxcode{.spc}) and chemical equations (\sphinxcode{.eqn}). You
should modify these lines to the following:

\begin{Verbatim}[commandchars=\\\{\}]
\PYG{c+c1}{\PYGZsh{}include strato3.spc}
\PYG{c+c1}{\PYGZsh{}include strato3.eqn}
\end{Verbatim}

Also, you should do same modification we did in the last example. That is to change the
length of the run from 3 to 30 days by modifying the line that reads \sphinxcode{TEND =
TSTART + (3*24*3600)} to make it read \sphinxcode{TEND = TSTART + (30*24*3600)}, and to change
the line that reads \sphinxcode{\#LOOKATALL} to \sphinxcode{\#LOOKAT O3; NO; NO2;}.

If you try to run KPP now you'll again get an error because those
files still don't exist. Let's create them by copying the original \sphinxcode{small\_strato}
files, which can be done with the following commands:

\begin{Verbatim}[commandchars=\\\{\}]
cp \PYGZdl{}KPP\PYGZus{}HOME/models/small\PYGZus{}strato.spc strato3.spc
cp \PYGZdl{}KPP\PYGZus{}HOME/models/small\PYGZus{}strato.eqn strato3.eqn
\end{Verbatim}

Now if you try running KPP it should work. But this is still not what we want;
this is just the \sphinxcode{small\_strato} mechanism with another name, so let us move to
the actual changes.

If you check the \sphinxcode{strato3.spc} file you'll see that it only the definitions
of the species used, which wouldn't make much sense to change for now since
we'll be using the same species, so we will leave it how it is. Now we focus on
the \sphinxcode{strato3.eqn} file. If you open it you'll find the following lines:

\begin{Verbatim}[commandchars=\\\{\}]
\PYG{c+c1}{\PYGZsh{}EQUATIONS \PYGZob{} Small Stratospheric Mechanism \PYGZcb{}}

\PYG{o}{\PYGZlt{}}\PYG{n}{R1}\PYG{o}{\PYGZgt{}}  \PYG{n}{O2}   \PYG{o}{+} \PYG{n}{hv} \PYG{o}{=} \PYG{l+m+mi}{2}\PYG{n}{O}            \PYG{p}{:} \PYG{p}{(}\PYG{l+m+mf}{2.643E\PYGZhy{}10}\PYG{p}{)} \PYG{o}{*} \PYG{n}{SUN}\PYG{o}{*}\PYG{n}{SUN}\PYG{o}{*}\PYG{n}{SUN}\PYG{p}{;}
\PYG{o}{\PYGZlt{}}\PYG{n}{R2}\PYG{o}{\PYGZgt{}}  \PYG{n}{O}    \PYG{o}{+} \PYG{n}{O2} \PYG{o}{=} \PYG{n}{O3}            \PYG{p}{:} \PYG{p}{(}\PYG{l+m+mf}{8.018E\PYGZhy{}17}\PYG{p}{)}\PYG{p}{;}
\PYG{o}{\PYGZlt{}}\PYG{n}{R3}\PYG{o}{\PYGZgt{}}  \PYG{n}{O3}   \PYG{o}{+} \PYG{n}{hv} \PYG{o}{=} \PYG{n}{O}   \PYG{o}{+} \PYG{n}{O2}      \PYG{p}{:} \PYG{p}{(}\PYG{l+m+mf}{6.120E\PYGZhy{}04}\PYG{p}{)} \PYG{o}{*} \PYG{n}{SUN}\PYG{p}{;}
\PYG{o}{\PYGZlt{}}\PYG{n}{R4}\PYG{o}{\PYGZgt{}}  \PYG{n}{O}    \PYG{o}{+} \PYG{n}{O3} \PYG{o}{=} \PYG{l+m+mi}{2}\PYG{n}{O2}           \PYG{p}{:} \PYG{p}{(}\PYG{l+m+mf}{1.576E\PYGZhy{}15}\PYG{p}{)}\PYG{p}{;}
\PYG{o}{\PYGZlt{}}\PYG{n}{R5}\PYG{o}{\PYGZgt{}}  \PYG{n}{O3}   \PYG{o}{+} \PYG{n}{hv} \PYG{o}{=} \PYG{n}{O1D} \PYG{o}{+} \PYG{n}{O2}      \PYG{p}{:} \PYG{p}{(}\PYG{l+m+mf}{1.070E\PYGZhy{}03}\PYG{p}{)} \PYG{o}{*} \PYG{n}{SUN}\PYG{o}{*}\PYG{n}{SUN}\PYG{p}{;}
\PYG{o}{\PYGZlt{}}\PYG{n}{R6}\PYG{o}{\PYGZgt{}}  \PYG{n}{O1D}  \PYG{o}{+} \PYG{n}{M}  \PYG{o}{=} \PYG{n}{O}   \PYG{o}{+} \PYG{n}{M}       \PYG{p}{:} \PYG{p}{(}\PYG{l+m+mf}{7.110E\PYGZhy{}11}\PYG{p}{)}\PYG{p}{;}
\PYG{o}{\PYGZlt{}}\PYG{n}{R7}\PYG{o}{\PYGZgt{}}  \PYG{n}{O1D}  \PYG{o}{+} \PYG{n}{O3} \PYG{o}{=} \PYG{l+m+mi}{2}\PYG{n}{O2}           \PYG{p}{:} \PYG{p}{(}\PYG{l+m+mf}{1.200E\PYGZhy{}10}\PYG{p}{)}\PYG{p}{;}
\PYG{o}{\PYGZlt{}}\PYG{n}{R8}\PYG{o}{\PYGZgt{}}  \PYG{n}{NO}   \PYG{o}{+} \PYG{n}{O3} \PYG{o}{=} \PYG{n}{NO2} \PYG{o}{+} \PYG{n}{O2}      \PYG{p}{:} \PYG{p}{(}\PYG{l+m+mf}{6.062E\PYGZhy{}15}\PYG{p}{)}\PYG{p}{;}
\PYG{o}{\PYGZlt{}}\PYG{n}{R9}\PYG{o}{\PYGZgt{}}  \PYG{n}{NO2}  \PYG{o}{+} \PYG{n}{O}  \PYG{o}{=} \PYG{n}{NO}  \PYG{o}{+} \PYG{n}{O2}      \PYG{p}{:} \PYG{p}{(}\PYG{l+m+mf}{1.069E\PYGZhy{}11}\PYG{p}{)}\PYG{p}{;}
\PYG{o}{\PYGZlt{}}\PYG{n}{R10}\PYG{o}{\PYGZgt{}} \PYG{n}{NO2}  \PYG{o}{+} \PYG{n}{hv} \PYG{o}{=} \PYG{n}{NO}  \PYG{o}{+} \PYG{n}{O}       \PYG{p}{:} \PYG{p}{(}\PYG{l+m+mf}{1.289E\PYGZhy{}02}\PYG{p}{)} \PYG{o}{*} \PYG{n}{SUN}\PYG{p}{;}
\end{Verbatim}

Just for the sake of learning, let us change the photolysis rate (last
reaction) to make it a lot slower. We will make the last line read:

\begin{Verbatim}[commandchars=\\\{\}]
\PYG{o}{\PYGZlt{}}\PYG{n}{R10}\PYG{o}{\PYGZgt{}} \PYG{n}{NO2}  \PYG{o}{+} \PYG{n}{hv} \PYG{o}{=} \PYG{n}{NO}  \PYG{o}{+} \PYG{n}{O}       \PYG{p}{:} \PYG{p}{(}\PYG{l+m+mf}{1.289E\PYGZhy{}06}\PYG{p}{)} \PYG{o}{*} \PYG{n}{SUN}\PYG{p}{;}
\end{Verbatim}

\begin{notice}{note}{Note:}
This is 4 orders of magnitude slower than it previously was and it may not be
realistic! We only make this change for the sake of illustration, so that
the output change is easier to see.
\end{notice}

You can actually change not only the reaction rate for the equation, but also
modify the equations itself here and add (or remove) equations. For now we will
leave the equations the way they are.

Now we go through the same steps of running \sphinxcode{kpp strato3.kpp}, changing the
compiler to gfortran and running \sphinxcode{make -f Makefile\_strato3}. If everything
goes well, we'll see the \sphinxcode{strato3.exe} created. After running
\sphinxcode{./strato3.exe} sure enough \sphinxcode{strato3.dat} is created, which we can plot
with the same python code from the last example (only changing the name of the
file of course):
\phantomsection\label{improving:test3-time}\begin{figure}[htbp]
\centering

\noindent\scalebox{0.800000}{\sphinxincludegraphics{{test3_time}.png}}
\label{improving:test3-time}\end{figure}

\begin{notice}{note}{Note:}
This process of running KPP, then changing the Makefile, then compiling, etc.,
is pretty cumbersome and straightforward. So we included a file called
\sphinxcode{updatenrun.sh} in the directory \sphinxcode{test3} that can be found in the \href{https://github.com/tomchor/ezkpp/tree/gh-pages/test3}{github
repo}. This is a bash
script that does these steps automatically. To run it, you enter \sphinxcode{sh
updatenrun.sh modelname}. In this case, for example it shoudl be used with
\sphinxcode{sh updatenrun.sh strato3}.
\end{notice}

We can see that once again the final result changed. This time, since NO2 is
photolizing a lot slower, we see less NO in comparison with the previous plot.
We encourage you to try different reaction rates and initial conditions and see
what is the result in the model. With the \sphinxcode{updatenrun.sh} script (check the
note above) it should be easy!

Now that we have modified the \sphinxcode{small\_strato} example in several ways, let
take it a step further and create a new model from scratch.


\section{Creating a model from scratch}
\label{improving:creating-a-model-from-scratch}
Now we do one more step and create a completely new model with our own
set of reactions. Basically for our new model to be complete we should give it the
initial conditions, numerical constraints, species and reactions list. Let us
start with the KPP file and move on from there.

We will try to simulate a very small tropospheric model, which we will call
\sphinxcode{ttropo} (meaning tiny tropospheric; let's write it like that just because
it's easier). First let's create a new directory for our test with \sphinxcode{mkdir
ttropo} and move to that new directory with \sphinxcode{cd ttropo}. Now we create the
main KPP file with \sphinxcode{notepad++ ttropo.kpp} and put the following lines in it:

\begin{Verbatim}[commandchars=\\\{\}]
\PYG{c+c1}{\PYGZsh{}MODEL      ttropo}
\PYG{c+c1}{\PYGZsh{}LANGUAGE   Fortran90}
\PYG{c+c1}{\PYGZsh{}INTEGRATOR rosenbrock}
\PYG{c+c1}{\PYGZsh{}DRIVER     general}
\end{Verbatim}

Which means tells KPP to look for the \sphinxcode{ttropo.def} file. If you run KPP now
it will finish with an error because it won't find it. But we will create that
later. Let us first define our mechanism, i.e., our chemical reactions.

We create the \sphinxcode{ttropo.eqn} file (e.g. with \sphinxcode{notepad++ ttropo.eqn}). Now we
will put our reactions in that file, following the syntax that we saw in the
previous example. We choose a simplified set of tropospheric reactions that can
be written as:

\begin{Verbatim}[commandchars=\\\{\}]
\PYG{c+c1}{\PYGZsh{}EQUATIONS \PYGZob{} Tiny Tropospheric Mechanism \PYGZcb{}}
\PYG{o}{\PYGZlt{}}\PYG{n}{R2}\PYG{o}{\PYGZgt{}}  \PYG{n}{O}    \PYG{o}{+} \PYG{n}{O2}  \PYG{o}{=} \PYG{n}{O3}            \PYG{p}{:} \PYG{p}{(}\PYG{l+m+mf}{8.018E\PYGZhy{}17}\PYG{p}{)}\PYG{p}{;}
\PYG{o}{\PYGZlt{}}\PYG{n}{R1}\PYG{o}{\PYGZgt{}}  \PYG{n}{NO2}  \PYG{o}{+} \PYG{n}{hv}  \PYG{o}{=} \PYG{n}{NO}  \PYG{o}{+} \PYG{n}{O}       \PYG{p}{:} \PYG{p}{(}\PYG{l+m+mf}{1.289E\PYGZhy{}02}\PYG{p}{)} \PYG{o}{*} \PYG{n}{SUN}\PYG{p}{;}
\PYG{o}{\PYGZlt{}}\PYG{n}{R3}\PYG{o}{\PYGZgt{}}  \PYG{n}{NO}   \PYG{o}{+} \PYG{n}{O3}  \PYG{o}{=} \PYG{n}{NO2} \PYG{o}{+} \PYG{n}{O2}      \PYG{p}{:} \PYG{p}{(}\PYG{l+m+mf}{6.062E\PYGZhy{}15}\PYG{p}{)}\PYG{p}{;}
\PYG{o}{\PYGZlt{}}\PYG{n}{R41}\PYG{o}{\PYGZgt{}} \PYG{n}{O3}   \PYG{o}{+} \PYG{n}{hv}  \PYG{o}{=} \PYG{n}{O}   \PYG{o}{+} \PYG{n}{O2}      \PYG{p}{:} \PYG{p}{(}\PYG{l+m+mf}{5.500E\PYGZhy{}04}\PYG{p}{)} \PYG{o}{*} \PYG{n}{SUN}\PYG{p}{;}
\PYG{o}{\PYGZlt{}}\PYG{n}{R42}\PYG{o}{\PYGZgt{}} \PYG{n}{O3}   \PYG{o}{+} \PYG{n}{hv}  \PYG{o}{=} \PYG{n}{O1D} \PYG{o}{+} \PYG{n}{O2}      \PYG{p}{:} \PYG{p}{(}\PYG{l+m+mf}{6.000E\PYGZhy{}05}\PYG{p}{)} \PYG{o}{*} \PYG{n}{SUN}\PYG{o}{*}\PYG{n}{SUN}\PYG{p}{;}
\PYG{o}{\PYGZlt{}}\PYG{n}{R5}\PYG{o}{\PYGZgt{}}  \PYG{n}{O1D}  \PYG{o}{+} \PYG{n}{M}   \PYG{o}{=} \PYG{n}{O}   \PYG{o}{+} \PYG{n}{M}       \PYG{p}{:} \PYG{p}{(}\PYG{l+m+mf}{7.110E\PYGZhy{}11}\PYG{p}{)}\PYG{p}{;}
\PYG{o}{\PYGZlt{}}\PYG{n}{R6}\PYG{o}{\PYGZgt{}}  \PYG{n}{O1D} \PYG{o}{+} \PYG{n}{H2O}  \PYG{o}{=} \PYG{l+m+mi}{2}\PYG{n}{OH}           \PYG{p}{:} \PYG{p}{(}\PYG{l+m+mf}{2.2E\PYGZhy{}10}\PYG{p}{)}\PYG{p}{;}
\PYG{o}{\PYGZlt{}}\PYG{n}{R7}\PYG{o}{\PYGZgt{}}  \PYG{n}{CO}   \PYG{o}{+} \PYG{n}{OH}  \PYG{o}{=} \PYG{n}{CO2} \PYG{o}{+} \PYG{n}{HO2}     \PYG{p}{:} \PYG{p}{(}\PYG{l+m+mf}{2.2E\PYGZhy{}13}\PYG{p}{)}\PYG{p}{;}
\PYG{o}{\PYGZlt{}}\PYG{n}{R9}\PYG{o}{\PYGZgt{}}  \PYG{n}{HO2}  \PYG{o}{+} \PYG{n}{NO}  \PYG{o}{=} \PYG{n}{OH}  \PYG{o}{+} \PYG{n}{NO2}     \PYG{p}{:} \PYG{p}{(}\PYG{l+m+mf}{8.3E\PYGZhy{}12}\PYG{p}{)}\PYG{p}{;}
\PYG{o}{\PYGZlt{}}\PYG{n}{R10}\PYG{o}{\PYGZgt{}} \PYG{n}{OH}  \PYG{o}{+} \PYG{n}{NO2}  \PYG{o}{=} \PYG{n}{HNO3}          \PYG{p}{:} \PYG{p}{(}\PYG{l+m+mf}{1.1E\PYGZhy{}11}\PYG{p}{)}\PYG{p}{;}
\PYG{o}{\PYGZlt{}}\PYG{n}{R11}\PYG{o}{\PYGZgt{}} \PYG{n}{HO2} \PYG{o}{+} \PYG{n}{HO2}  \PYG{o}{=} \PYG{n}{H2O2}          \PYG{p}{:} \PYG{p}{(}\PYG{l+m+mf}{5.6E\PYGZhy{}12}\PYG{p}{)}\PYG{p}{;}
\PYG{o}{\PYGZlt{}}\PYG{n}{R12}\PYG{o}{\PYGZgt{}} \PYG{n}{O3}  \PYG{o}{+} \PYG{n}{HO2}  \PYG{o}{=} \PYG{n}{OH} \PYG{o}{+} \PYG{l+m+mi}{2}\PYG{n}{O2}      \PYG{p}{:} \PYG{p}{(}\PYG{l+m+mf}{2.0E\PYGZhy{}15}\PYG{p}{)}\PYG{p}{;}
\PYG{o}{\PYGZlt{}}\PYG{n}{R13}\PYG{o}{\PYGZgt{}} \PYG{n}{H2O2} \PYG{o}{+} \PYG{n}{hv}  \PYG{o}{=} \PYG{l+m+mi}{2}\PYG{n}{OH}           \PYG{p}{:} \PYG{p}{(}\PYG{l+m+mf}{1.366E\PYGZhy{}5}\PYG{p}{)} \PYG{o}{*} \PYG{n}{SUN}\PYG{p}{;}
\PYG{o}{\PYGZlt{}}\PYG{n}{R14}\PYG{o}{\PYGZgt{}} \PYG{n}{H2O2}       \PYG{o}{=} \PYG{n}{H2O2aq}        \PYG{p}{:} \PYG{p}{(}\PYG{l+m+mf}{3.3000e\PYGZhy{}03}\PYG{p}{)}\PYG{p}{;}
\PYG{o}{\PYGZlt{}}\PYG{n}{R15}\PYG{o}{\PYGZgt{}} \PYG{n}{HNO3}       \PYG{o}{=} \PYG{n}{HNO3aq}        \PYG{p}{:} \PYG{p}{(}\PYG{l+m+mf}{2.4000e\PYGZhy{}03}\PYG{p}{)}\PYG{p}{;}
\end{Verbatim}

So copy and paste those lines into \sphinxcode{ttropo.eqn}, save and exit.

\begin{notice}{note}{Note:}
Again, some of these reaction constants might not be exactly accurate for
tropospheric conditions, so please double-check if you plan on using them for
professional means, since the objective here is to only present this as an
example.
\end{notice}

Now we create the species file, which has to have all the species we used in
the reactions above properly defined. We can define a species as being variable
(when its concentration can vary according to the kinetics) or fixed (when its
concentration is a constant). In this case, we define only \sphinxcode{M}, \sphinxcode{H2O} and
\sphinxcode{O2} as fixed quantities and the other ones as variables:
\begin{alltt}
\#include atoms

\#DEFVAR
O   = O;            \{ Oxygen atomic ground state \}      
O1D = O;            \{ Oxygen atomic excited state \}
O3  = O + O + O;    \{ Ozone \}
NO2 = N + O + O;    \{ Nitrogen dioxide \}
NO  = N + O;        \{ Nitric oxide \}         
HNO3 = H + N + O+O+O;
H2O2 = H+H + O+O;
CO2  = C + O + O;
CO   = C + O;
OH   = O + H;
HO2  = H + O + O;
H2O2aq = IGNORE;
HNO3aq = IGNORE;

\#DEFFIX
M   = O + O + N + N;\{ Atmospheric generic molecule \}
O2  = O + O;        \{ Molecular oxygen \}
H2O = H + H + O;    \{ Water \}

\end{alltt}

Again, copy and paste those lines into \sphinxcode{ttropo.spc}, save, exit, and let's
proceed to the \sphinxcode{.def} file. Create \sphinxcode{ttropo.def} with \sphinxcode{notepad++
ttropo.def}.  In that file you will write the following lines:

\begin{Verbatim}[commandchars=\\\{\}]
\PYG{c+c1}{\PYGZsh{}include ttropo.spc}
\PYG{c+c1}{\PYGZsh{}include ttropo.eqn}

\PYG{c+c1}{\PYGZsh{}JACOBIAN SPARSE\PYGZus{}LU\PYGZus{}ROW      \PYGZob{}Use Sparse DATA STRUCTURES\PYGZcb{}}
\PYG{c+c1}{\PYGZsh{}DRIVER general}
\PYG{c+c1}{\PYGZsh{}DOUBLE ON}
\PYG{c+c1}{\PYGZsh{}STOICMAT ON}

\PYG{c+c1}{\PYGZsh{}LOOKATALL;                 \PYGZob{}File Output\PYGZcb{}\PYGZcb{}}
\PYG{c+c1}{\PYGZsh{}MONITOR O3;N;O;NO;O1D;NO2; \PYGZob{}Screen Output\PYGZcb{}}

\PYG{c+c1}{\PYGZsh{}CHECK O; N;                   \PYGZob{}Check Mass Balance\PYGZcb{}}

\PYG{c+c1}{\PYGZsh{}INITVALUES                    \PYGZob{}Initial Values\PYGZcb{}}
\PYG{n}{CFACTOR} \PYG{o}{=} \PYG{l+m+mf}{1.}    \PYG{p}{;}              \PYG{p}{\PYGZob{}}\PYG{n}{Conversion} \PYG{n}{Factor}\PYG{p}{\PYGZcb{}}
\PYG{n}{O3}  \PYG{o}{=} \PYG{l+m+mf}{7.65E+11} \PYG{p}{;} 
\PYG{n}{NO}  \PYG{o}{=} \PYG{l+m+mf}{2.55E+10} \PYG{p}{;}
\PYG{n}{NO2} \PYG{o}{=} \PYG{l+m+mf}{0.0E+09} \PYG{p}{;}
\PYG{n}{O2}  \PYG{o}{=} \PYG{l+m+mf}{1.697E+19} \PYG{p}{;}
\PYG{n}{M}   \PYG{o}{=} \PYG{l+m+mf}{2.550E+19} \PYG{p}{;}
\PYG{n}{H2O} \PYG{o}{=} \PYG{l+m+mf}{3.0E+17}\PYG{p}{;}
\PYG{n}{CO}  \PYG{o}{=} \PYG{l+m+mf}{2.55E13}\PYG{p}{;}

\PYG{n}{OH}     \PYG{o}{=} \PYG{l+m+mf}{0.}\PYG{p}{;}
\PYG{n}{HO2}    \PYG{o}{=} \PYG{l+m+mf}{0.}\PYG{p}{;}
\PYG{n}{H2O2}   \PYG{o}{=} \PYG{l+m+mf}{0.}\PYG{p}{;}
\PYG{n}{H2O2aq} \PYG{o}{=} \PYG{l+m+mf}{0.}\PYG{p}{;}
\PYG{n}{HNO3}   \PYG{o}{=} \PYG{l+m+mf}{0.}\PYG{p}{;}
\PYG{n}{HNO3aq} \PYG{o}{=} \PYG{l+m+mf}{0.}\PYG{p}{;}
\PYG{n}{O1D}    \PYG{o}{=} \PYG{l+m+mf}{0.} \PYG{p}{;}
\PYG{n}{O}      \PYG{o}{=} \PYG{l+m+mf}{0.} \PYG{p}{;}

\PYG{c+c1}{\PYGZsh{}INLINE F90\PYGZus{}INIT}
       \PYG{n}{TSTART} \PYG{o}{=} \PYG{p}{(}\PYG{l+m+mi}{12}\PYG{o}{*}\PYG{l+m+mi}{3600}\PYG{p}{)}
       \PYG{n}{TEND} \PYG{o}{=} \PYG{n}{TSTART} \PYG{o}{+} \PYG{p}{(}\PYG{l+m+mi}{15}\PYG{o}{*}\PYG{l+m+mi}{24}\PYG{o}{*}\PYG{l+m+mi}{3600}\PYG{p}{)}
       \PYG{n}{DT} \PYG{o}{=} \PYG{l+m+mf}{0.2}\PYG{o}{*}\PYG{l+m+mi}{3600}
       \PYG{n}{TEMP} \PYG{o}{=} \PYG{l+m+mi}{270}
\PYG{c+c1}{\PYGZsh{}ENDINLINE}
\end{Verbatim}

You can see that with this set of definitions we chose to run the model for 15
days, with a time step of 0.2 hours and that many of the initial concentrations
are set to zero. Note also that we are again starting the simulation at noon.

With these files we have the complete \sphinxcode{ttropo} model and are ready to run it.
We can use the \sphinxcode{updatenrun.sh} script as \sphinxcode{sh updatenrun.sh ttropo} (you'll
have to copy it to the current directory with \sphinxcode{cp} first). It should run
successfully now. Note that we again have to check out \sphinxcode{ttropo.map} to find
out the order of the species in the output file. We can use the following Python
script to plot the results (the correct output order is already included in
it):

\begin{Verbatim}[commandchars=\\\{\}]
\PYG{k+kn}{import} \PYG{n+nn}{pandas} \PYG{k}{as} \PYG{n+nn}{pd}
\PYG{k+kn}{from} \PYG{n+nn}{matplotlib} \PYG{k}{import} \PYG{n}{pyplot} \PYG{k}{as} \PYG{n}{plt}

\PYG{n}{concs} \PYG{o}{=} \PYG{n}{pd}\PYG{o}{.}\PYG{n}{read\PYGZus{}csv}\PYG{p}{(}\PYG{l+s+s1}{\PYGZsq{}}\PYG{l+s+s1}{ttropo.dat}\PYG{l+s+s1}{\PYGZsq{}}\PYG{p}{,} \PYG{n}{index\PYGZus{}col}\PYG{o}{=}\PYG{l+m+mi}{0}\PYG{p}{,} \PYG{n}{delim\PYGZus{}whitespace}\PYG{o}{=}\PYG{k+kc}{True}\PYG{p}{,} \PYG{n}{header}\PYG{o}{=}\PYG{k+kc}{None}\PYG{p}{,} \PYG{n}{dtype}\PYG{o}{=}\PYG{k+kc}{None}\PYG{p}{)}\PYG{o}{.}\PYG{n}{apply}\PYG{p}{(}\PYG{n}{pd}\PYG{o}{.}\PYG{n}{to\PYGZus{}numeric}\PYG{p}{,} \PYG{n}{errors}\PYG{o}{=}\PYG{l+s+s1}{\PYGZsq{}}\PYG{l+s+s1}{coerce}\PYG{l+s+s1}{\PYGZsq{}}\PYG{p}{)}
\PYG{n}{concs}\PYG{o}{.}\PYG{n}{columns} \PYG{o}{=} \PYG{p}{[}\PYG{l+s+s1}{\PYGZsq{}}\PYG{l+s+s1}{CO2}\PYG{l+s+s1}{\PYGZsq{}}\PYG{p}{,} \PYG{l+s+s1}{\PYGZsq{}}\PYG{l+s+s1}{H2O2aq}\PYG{l+s+s1}{\PYGZsq{}}\PYG{p}{,} \PYG{l+s+s1}{\PYGZsq{}}\PYG{l+s+s1}{HNO3aq}\PYG{l+s+s1}{\PYGZsq{}}\PYG{p}{,} \PYG{l+s+s1}{\PYGZsq{}}\PYG{l+s+s1}{HNO3}\PYG{l+s+s1}{\PYGZsq{}}\PYG{p}{,} \PYG{l+s+s1}{\PYGZsq{}}\PYG{l+s+s1}{H2O2}\PYG{l+s+s1}{\PYGZsq{}}\PYG{p}{,} \PYG{l+s+s1}{\PYGZsq{}}\PYG{l+s+s1}{CO}\PYG{l+s+s1}{\PYGZsq{}}\PYG{p}{,} \PYG{l+s+s1}{\PYGZsq{}}\PYG{l+s+s1}{O1D}\PYG{l+s+s1}{\PYGZsq{}}\PYG{p}{,} \PYG{l+s+s1}{\PYGZsq{}}\PYG{l+s+s1}{O}\PYG{l+s+s1}{\PYGZsq{}}\PYG{p}{,} \PYG{l+s+s1}{\PYGZsq{}}\PYG{l+s+s1}{OH}\PYG{l+s+s1}{\PYGZsq{}}\PYG{p}{,} \PYG{l+s+s1}{\PYGZsq{}}\PYG{l+s+s1}{HO2}\PYG{l+s+s1}{\PYGZsq{}}\PYG{p}{,} \PYG{l+s+s1}{\PYGZsq{}}\PYG{l+s+s1}{O3}\PYG{l+s+s1}{\PYGZsq{}}\PYG{p}{,} \PYG{l+s+s1}{\PYGZsq{}}\PYG{l+s+s1}{NO}\PYG{l+s+s1}{\PYGZsq{}}\PYG{p}{,} \PYG{l+s+s1}{\PYGZsq{}}\PYG{l+s+s1}{NO2}\PYG{l+s+s1}{\PYGZsq{}}\PYG{p}{,} \PYG{l+s+s1}{\PYGZsq{}}\PYG{l+s+s1}{M}\PYG{l+s+s1}{\PYGZsq{}}\PYG{p}{,} \PYG{l+s+s1}{\PYGZsq{}}\PYG{l+s+s1}{O2}\PYG{l+s+s1}{\PYGZsq{}}\PYG{p}{,} \PYG{l+s+s1}{\PYGZsq{}}\PYG{l+s+s1}{H2O}\PYG{l+s+s1}{\PYGZsq{}}\PYG{p}{]}
\PYG{n}{concs}\PYG{o}{.}\PYG{n}{index}\PYG{o}{.}\PYG{n}{name} \PYG{o}{=} \PYG{l+s+s1}{\PYGZsq{}}\PYG{l+s+s1}{Hours since noon}\PYG{l+s+s1}{\PYGZsq{}}
\PYG{n}{concs}\PYG{o}{.}\PYG{n}{plot}\PYG{p}{(}\PYG{n}{ylim}\PYG{o}{=}\PYG{p}{[}\PYG{l+m+mf}{1.e4}\PYG{p}{,} \PYG{k+kc}{None}\PYG{p}{]}\PYG{p}{,} \PYG{n}{logy}\PYG{o}{=}\PYG{k+kc}{True}\PYG{p}{,} \PYG{n}{y}\PYG{o}{=}\PYG{p}{[}\PYG{l+s+s1}{\PYGZsq{}}\PYG{l+s+s1}{O3}\PYG{l+s+s1}{\PYGZsq{}}\PYG{p}{,} \PYG{l+s+s1}{\PYGZsq{}}\PYG{l+s+s1}{NO}\PYG{l+s+s1}{\PYGZsq{}}\PYG{p}{,} \PYG{l+s+s1}{\PYGZsq{}}\PYG{l+s+s1}{NO2}\PYG{l+s+s1}{\PYGZsq{}}\PYG{p}{,} \PYG{l+s+s1}{\PYGZsq{}}\PYG{l+s+s1}{CO}\PYG{l+s+s1}{\PYGZsq{}}\PYG{p}{,} \PYG{l+s+s1}{\PYGZsq{}}\PYG{l+s+s1}{HNO3}\PYG{l+s+s1}{\PYGZsq{}}\PYG{p}{,} \PYG{l+s+s1}{\PYGZsq{}}\PYG{l+s+s1}{OH}\PYG{l+s+s1}{\PYGZsq{}}\PYG{p}{,} \PYG{l+s+s1}{\PYGZsq{}}\PYG{l+s+s1}{HO2}\PYG{l+s+s1}{\PYGZsq{}}\PYG{p}{]}\PYG{p}{,} \PYG{n}{grid}\PYG{o}{=}\PYG{k+kc}{True}\PYG{p}{)}

\PYG{n}{plt}\PYG{o}{.}\PYG{n}{savefig}\PYG{p}{(}\PYG{l+s+s1}{\PYGZsq{}}\PYG{l+s+s1}{test4\PYGZus{}time.png}\PYG{l+s+s1}{\PYGZsq{}}\PYG{p}{,} \PYG{n}{bbox\PYGZus{}inches}\PYG{o}{=}\PYG{l+s+s1}{\PYGZsq{}}\PYG{l+s+s1}{tight}\PYG{l+s+s1}{\PYGZsq{}}\PYG{p}{)}
\PYG{n}{plt}\PYG{o}{.}\PYG{n}{show}\PYG{p}{(}\PYG{p}{)}
\end{Verbatim}

The output of this model can be seen in this figure:
\phantomsection\label{improving:test4-time}\begin{figure}[htbp]
\centering

\noindent\scalebox{0.800000}{\sphinxincludegraphics{{test4_time}.png}}
\label{improving:test4-time}\end{figure}

As you can see, in 15 days we ran the model for, it hasn't reached equilibrium
yet. This system of equations will take longer to reach equilibrium state than
previous models. You can, once again, investigate the effects of different
initial concentrations in the final result, change the reactions and reaction
rates, or even fix some quantities in the \sphinxcode{.spc} file.

The creation of any new model from scratch follows the same paths that we just
described here. So for other models, following these steps should produce the
correct result. For model detailed information not included here, we refer the
user to the official manual for KPP.


\chapter{Possible bug fixes}
\label{bugs:possible-bug-fixes}\label{bugs::doc}\label{bugs:bugs}
\begin{notice}{note}{Note:}
This section is meant to become a list of common bugs that might arise
along with a way to solve them.
\end{notice}


\section{Can't fine bashrc file}
\label{bugs:can-t-fine-bashrc-file}
If you could not find your \sphinxcode{.bashrc} file, it might mean that you either
don't have one, or that it is located somewhere else. If you're using C shell,
then you should be actually looking for \sphinxcode{.cshrc}. If that is indeed the case,
replace \sphinxcode{bashrc} by \sphinxcode{cshrc} everywhere it appears. Furthermore, environment
definitions are made slightly different in C shells. Instead of typing \sphinxcode{export
KPP\_HOME=\$HOME/kpp} for example, you would have to type \sphinxcode{setenv KPP\_HOME
\$HOME/kpp}. For other (less common) shells, we advice you to google these
definitions. They should be easy to find.  When in doubt of which shell you're
using, type \sphinxcode{echo \$SHELL} and check the output.

If you still can't find your \sphinxcode{.bashrc} or \sphinxcode{.cshrc}, chances are you're using
an emulator, and not running natively. If this is the case, google for the
location of the \sphinxcode{.bashrc}-equivalent in your shell emulator (be it, Cygwin,
Mingw, cmder or others).



\renewcommand{\indexname}{Index}
\printindex
\end{document}
